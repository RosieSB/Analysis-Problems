%% Generated by Sphinx.
\def\sphinxdocclass{jupyterBook}
\documentclass[letterpaper,10pt,english]{jupyterBook}
\ifdefined\pdfpxdimen
   \let\sphinxpxdimen\pdfpxdimen\else\newdimen\sphinxpxdimen
\fi \sphinxpxdimen=.75bp\relax
\ifdefined\pdfimageresolution
    \pdfimageresolution= \numexpr \dimexpr1in\relax/\sphinxpxdimen\relax
\fi
%% let collapsible pdf bookmarks panel have high depth per default
\PassOptionsToPackage{bookmarksdepth=5}{hyperref}
%% turn off hyperref patch of \index as sphinx.xdy xindy module takes care of
%% suitable \hyperpage mark-up, working around hyperref-xindy incompatibility
\PassOptionsToPackage{hyperindex=false}{hyperref}
%% memoir class requires extra handling
\makeatletter\@ifclassloaded{memoir}
{\ifdefined\memhyperindexfalse\memhyperindexfalse\fi}{}\makeatother

\PassOptionsToPackage{booktabs}{sphinx}
\PassOptionsToPackage{colorrows}{sphinx}

\PassOptionsToPackage{warn}{textcomp}

\catcode`^^^^00a0\active\protected\def^^^^00a0{\leavevmode\nobreak\ }
\usepackage{cmap}
\usepackage{fontspec}
\defaultfontfeatures[\rmfamily,\sffamily,\ttfamily]{}
\usepackage{amsmath,amssymb,amstext}
\usepackage{polyglossia}
\setmainlanguage{english}



\setmainfont{FreeSerif}[
  Extension      = .otf,
  UprightFont    = *,
  ItalicFont     = *Italic,
  BoldFont       = *Bold,
  BoldItalicFont = *BoldItalic
]
\setsansfont{FreeSans}[
  Extension      = .otf,
  UprightFont    = *,
  ItalicFont     = *Oblique,
  BoldFont       = *Bold,
  BoldItalicFont = *BoldOblique,
]
\setmonofont{FreeMono}[
  Extension      = .otf,
  UprightFont    = *,
  ItalicFont     = *Oblique,
  BoldFont       = *Bold,
  BoldItalicFont = *BoldOblique,
]



\usepackage[Bjarne]{fncychap}
\usepackage[,numfigreset=2,mathnumfig]{sphinx}

\fvset{fontsize=\small}
\usepackage{geometry}


% Include hyperref last.
\usepackage{hyperref}
% Fix anchor placement for figures with captions.
\usepackage{hypcap}% it must be loaded after hyperref.
% Set up styles of URL: it should be placed after hyperref.
\urlstyle{same}


\usepackage{sphinxmessages}



        % Start of preamble defined in sphinx-jupyterbook-latex %
         \usepackage[Latin,Greek]{ucharclasses}
        \usepackage{unicode-math}
        % fixing title of the toc
        \addto\captionsenglish{\renewcommand{\contentsname}{Contents}}
        \hypersetup{
            pdfencoding=auto,
            psdextra
        }
        % End of preamble defined in sphinx-jupyterbook-latex %
        

\title{MAS2004/9 Semester 2 Problems}
\date{Jan 22, 2025}
\release{}
\author{Rosie Shewell Brockway}
\newcommand{\sphinxlogo}{\vbox{}}
\renewcommand{\releasename}{}
\makeindex
\begin{document}

\pagestyle{empty}
\sphinxmaketitle
\pagestyle{plain}
\sphinxtableofcontents
\pagestyle{normal}
\phantomsection\label{\detokenize{intro::doc}}


\sphinxAtStartPar
Maths is best learned by doing many problems.

\sphinxAtStartPar
This problem booklet serves 3 purposes:
\begin{enumerate}
\sphinxsetlistlabels{\arabic}{enumi}{enumii}{}{.}%
\item {} 
\sphinxAtStartPar
\sphinxstylestrong{Problems of the Week.} Each week, I will circulate a selection of problems from this booklet that relate most closely to the material covered in lectures that week. I recommend working on these steadily throughout the semester, and keeping a record of any you would like help with (e.g. in tutorials or \sphinxhref{https://calendar.app.google/1LFwxG3HSMFM8pC17}{office hours}).

\item {} 
\sphinxAtStartPar
\sphinxstylestrong{Tutorials.} The problem set for each tutorial will come from this booklet.

\item {} 
\sphinxAtStartPar
\sphinxstylestrong{Homework}. Written homework will be a selection of problems from the problem booklet.

\end{enumerate}

\sphinxAtStartPar
See Blackboard%
\begin{footnote}[1]\sphinxAtStartFootnote
Links: \sphinxhref{https://vle.shef.ac.uk/webapps/blackboard/content/listContentEditable.jsp?content\_id=\_7918618\_1\&amp;course\_id=\_119813\_1\&amp;mode=reset}{MAS2004}; \sphinxhref{https://vle.shef.ac.uk/ultra/courses/\_119818\_1/cl/outline}{MAS2009}
%
\end{footnote} for more information.

\sphinxAtStartPar
{\hyperref[\detokenize{Solutions-full:sol}]{\sphinxcrossref{\DUrole{std,std-ref}{Solutions}}}} will be made available in installments throughout the semester..


\bigskip\hrule\bigskip


\sphinxstepscope


\section{Problems}
\label{\detokenize{Problems:problems}}\label{\detokenize{Problems:prob}}\label{\detokenize{Problems::doc}}

\subsection{Preliminary problems}
\label{\detokenize{Problems:preliminary-problems}}\phantomsection\label{\detokenize{Problems:p1}}
\sphinxAtStartPar
P1. Which of the following statements is the correct definition of convergence of a real sequence \((x_n)\) to a limit \(l\in\mathbb{R}\)? How would you correct the incorrect statements?

{[}Note: There is more than one correct answer.{]}

\sphinxAtStartPar
(i) \(\varepsilon>0\) \(N\in\mathbb{N}\) then \(|x_n-l|<\varepsilon\).

\sphinxAtStartPar
(ii) \(\forall\varepsilon>0\) \(\exists N\in\mathbb{N}\) s.t.  \(|x_n-l|<\varepsilon\) \(\forall n\geq N\).

\sphinxAtStartPar
(iii) For all \(\varepsilon>0\) and \(N\in\mathbb{N}\), we have \(|x_n-l|<\varepsilon\) whenever \(n\geq N\).

\sphinxAtStartPar
(iv) \(\exists\varepsilon>0\) s.t. \(\forall N\in\mathbb{N}\), \(|x_n-l|<\varepsilon\) \(\forall n\geq N\).

\sphinxAtStartPar
(v) Given any \(\varepsilon>0\), there is \(N\in\mathbb{N}\) such that \(|x_n-l|<\varepsilon\) whenever \(n\geq N\).


\phantomsection\label{\detokenize{Problems:p2}}
\sphinxAtStartPar
P2. Prove using the \((\varepsilon-N)\)\sphinxhyphen{}definition of convergence that the sequence \((x_n)\) given by \(x_n = \frac{2n}{3n-1}\) converges to \(\frac{2}{3}\) as \(n\rightarrow\infty\).


\phantomsection\label{\detokenize{Problems:p3}}
\sphinxAtStartPar
P3. Using your MAS107 notes or another resource, look up and carefully state the Bolzano–Weierstrass theorem.

Its proof combines two other important results about sequences from MAS107. What do these results say?

{[}Note: We will use all three of these theorems at various points this semester. Check you are clear on what each of them is saying, and remember your lecturers and tutors are here to help.{]}


\phantomsection\label{\detokenize{Problems:p4}}
\sphinxAtStartPar
P4.\\
(i) If \(a, b \geq 0\), show that \(\sqrt{a + b} \leq \sqrt{a} + \sqrt{b}\).

{[}Hint: Consider the square of both sides. You may use that the square root function is increasing.{]}

\sphinxAtStartPar
(ii) If \(a,b \in \mathbb{R}\), deduce that \(\left|\sqrt{|a|} - \sqrt{|b|}\right| \leq \sqrt{|a - b|}\).

{[}Hint: Immitate the proof of \sphinxhref{https://rosiesb.github.io/Analysis-Notes/1Rev.html\#cor:tri}{Corollary 1.1} in the lecture notes.{]}

\sphinxAtStartPar
(iii) Hence prove that if the sequence \((a_{n})\) converges to \(l\), then \(\left(\sqrt{|a_{n}|}\right)\) converges to \(\sqrt{|l|}\).


\phantomsection\label{\detokenize{Problems:p5}}
\sphinxAtStartPar
P5. Look up and carefully define the \sphinxstyleemphasis{supremum} and \sphinxstyleemphasis{infimum} of a set \(A\subseteq\mathbb{R}\). What conditions must be met for each of these to exist?


\phantomsection\label{\detokenize{Problems:p6}}
\sphinxAtStartPar
P6. Compute, without proof, the suprema and infima (if they exist) of the following sets:

\sphinxAtStartPar
(i) \(\left\{\frac{m}{n}:m,n\in\mathbb{N} \text{ s.t } m<n\right\}\).

\sphinxAtStartPar
(ii) \(\left\{\frac{(-1)^m}{n}:m,n\in\mathbb{N} \text{ s.t } m<n\right\}\).

\sphinxAtStartPar
(iii) \(\left\{\frac{n}{3n + 1} :n\in\mathbb{N}\right\}\).


\phantomsection\label{\detokenize{Problems:p7}}
\sphinxAtStartPar
P7. Let \(A\) and \(B\) be bounded and non\sphinxhyphen{}empty subsets of \(\mathbb{R}\). Which of the following statements are true, and which are false?

\sphinxAtStartPar
(i) \(\inf A < \sup A\).

\sphinxAtStartPar
(ii) For all \(\varepsilon>0\), there is \(x\in A\) such that \(x-\inf A < \varepsilon\).

\sphinxAtStartPar
(iii) \(A\subseteq B\) implies \(\sup A \leq \sup B\).

\sphinxAtStartPar
(iv) If \(x>0\) for all \(x\in B\), then \(\inf B >0\).

\sphinxAtStartPar
(v) \(\sup(A\cup B) = \max\{\sup A,\sup B\}\).




\subsection{Limits of functions}
\label{\detokenize{Problems:limits-of-functions}}\phantomsection\label{\detokenize{Problems:id1}}\begin{enumerate}
\sphinxsetlistlabels{\arabic}{enumi}{enumii}{}{.}%
\item {} 
\sphinxAtStartPar
For each of the following formulas, what is the largest subset \(X\) of \(\mathbb{R}\)  which may be taken as the domain of a function with that formula?

\sphinxAtStartPar
(i) \(g_{1}(x) = \displaystyle\frac{x^{2} + 2x + 7}{x(x+1)}\),

\sphinxAtStartPar
(ii) \(g_{2}(x) = \displaystyle\frac{(x-1)(x+4)}{x^{3} + 4x^{2} + x - 6}\),

\sphinxAtStartPar
(iii) \(g_{3}(x) = \displaystyle\frac{x+4}{x^{2}+5x + 6}\),

\sphinxAtStartPar
(iv) \(g_{4}(x) = \exp{\left(-\displaystyle\frac{1}{x-1}\right)}\),

\sphinxAtStartPar
(v) \(g_{5}(x) = \cos\left(\displaystyle\frac{1}{\pi x}\right)\).

\end{enumerate}
\phantomsection\label{\detokenize{Problems:id2}}\begin{enumerate}
\sphinxsetlistlabels{\arabic}{enumi}{enumii}{}{.}%
\setcounter{enumi}{1}
\item {} 
\sphinxAtStartPar
For each of the sets \(X\subseteq\mathbb{R}\) below, calculate the associated set \(L\) of its limit points.

\sphinxAtStartPar
(i) \(X=(0,1)\cup[2,3)\cup\{4,5\}\),

\sphinxAtStartPar
(ii) \(X=\mathbb{Z}\),

\sphinxAtStartPar
(iii) \(X=\mathbb{R}\setminus\mathbb{Z}\),

\sphinxAtStartPar
(iv) \(X=\{x\in\mathbb{Q}:0<x<1\}\),

\sphinxAtStartPar
(v) \(X=\displaystyle\left\{\frac{1}{n}:n\in\mathbb{N}\right\}\).

\end{enumerate}
\phantomsection\label{\detokenize{Problems:id3}}\begin{enumerate}
\sphinxsetlistlabels{\arabic}{enumi}{enumii}{}{.}%
\setcounter{enumi}{2}
\item {} 
\sphinxAtStartPar
For each of the following functions \(f\), calculate \(\lim_{x\rightarrow 2}f(x)\), and prove your calculation is correct using the \((\varepsilon-\delta)\) criterion (\sphinxhref{https://rosiesb.github.io/Analysis-Notes/2LoF.html\#functionlimit}{Definition 2.2}). Does \(f\) converge to a limit as \(x\rightarrow 0\)?

\sphinxAtStartPar
(i) \(f:\mathbb{R}\to\mathbb{R}\); \(f(x)=4x+7\).

\sphinxAtStartPar
(ii) \(f:\{0\}\cup[1,3]\to\mathbb{R}\); \(f(x)=3x^2-1\).

\sphinxAtStartPar
(iii) \(f:(0,\infty)\to\mathbb{R}\); \(f(x)=x+\frac{1}{x}\).

\end{enumerate}
\phantomsection\label{\detokenize{Problems:id4}}\begin{enumerate}
\sphinxsetlistlabels{\arabic}{enumi}{enumii}{}{.}%
\setcounter{enumi}{3}
\item {} 
\sphinxAtStartPar
For \(g_2:X\to \mathbb{R}\) as in part (ii) of the {\hyperref[\detokenize{Problems:id1}]{\sphinxcrossref{Problem 1}}}, investigate each of

\end{enumerate}
\begin{equation*}
\begin{split}
\lim_{x \rightarrow 1}g_{2}(x), \hspace{3em} \lim_{x \rightarrow -2}g_{2}(x) \hspace{2em} \text{ and } \hspace{2em} \lim_{x \rightarrow -3}g_{2}(x).
\end{split}
\end{equation*}
\sphinxAtStartPar
Use the sequential criterion (\sphinxhref{https://rosiesb.github.io/Analysis-Notes/2LoF.html\#ed}{Theorem 2.10}) to prove you are right.

\phantomsection\label{\detokenize{Problems:id5}}\begin{enumerate}
\sphinxsetlistlabels{\arabic}{enumi}{enumii}{}{.}%
\setcounter{enumi}{4}
\item {} 
\sphinxAtStartPar
If \(f: \mathbb{R} \rightarrow [0, \infty)\) satisfies \(\displaystyle\lim_{x \rightarrow a} f(x) = l\), where \(l > 0\), show that

\end{enumerate}
\begin{equation*}
\begin{split}
\lim_{x \rightarrow a} \sqrt{f(x)} = \sqrt{l}.
\end{split}
\end{equation*}
\sphinxAtStartPar
Hence calculate \(\displaystyle\lim_{x \rightarrow 1}\sqrt{\displaystyle\frac{x+1}{x^{2}}}\).

{[}Hint: For the first part, use the result of {\hyperref[\detokenize{Problems:p4}]{\sphinxcrossref{\DUrole{std,std-ref}{P4 (iii)}}}} from the preliminary exercises.{]}

\phantomsection\label{\detokenize{Problems:id6}}\begin{enumerate}
\sphinxsetlistlabels{\arabic}{enumi}{enumii}{}{.}%
\setcounter{enumi}{5}
\item {} 
\sphinxAtStartPar
Why are limits of functions unique?

\end{enumerate}
\phantomsection\label{\detokenize{Problems:id7}}\begin{enumerate}
\sphinxsetlistlabels{\arabic}{enumi}{enumii}{}{.}%
\setcounter{enumi}{6}
\item {} 
\sphinxAtStartPar
Verify that \(\text{sgn}(x) = \displaystyle\frac{|x|}{x} = \displaystyle\frac{x}{|x|}\), for \(x \neq 0\) (see Example 2.3  in the notes for the definition).

Show that \(\displaystyle\lim_{x \rightarrow 0}\text{sgn}(x)\) does not exist. {[}Hint: Use a well\sphinxhyphen{}chosen sequence and \sphinxhref{https://rosiesb.github.io/Analysis-Notes/2LoF.html\#ed}{Theorem 2.1}.{]}

Show that both the left and right limits exist at \(x=0\), and find their values.

\end{enumerate}
\phantomsection\label{\detokenize{Problems:id8}}\begin{enumerate}
\sphinxsetlistlabels{\arabic}{enumi}{enumii}{}{.}%
\setcounter{enumi}{7}
\item {} 
\sphinxAtStartPar
For the following functions, each of which is defined on the whole of \(\mathbb{R}\), find every point at which both the left and right limits exist, but are different from each other, and find the values of these limits.

\sphinxAtStartPar
(i) \(f(x) = \begin{cases} 1 -x & \text{if }x < 1\\ x^{2}& \text{if }x \geq 1. \end{cases}\)

\sphinxAtStartPar
(ii) \(g(x) = [x]\), where
\begin{equation}\label{equation:Problems:eq:[x]}
\begin{split}[x] = \left\{\begin{array}{cl} \lfloor x\rfloor & \text{ if } x\geq 0 \\ \lceil x \rceil & \text{ if } x<0 \end{array}\right.\end{split}
\end{equation}
\sphinxAtStartPar
denotes the integer part of \(x\).

\sphinxAtStartPar
(iii) \(h(x) =3 - 5{\bf 1}_{(0, 1]}(x) + 7{\bf 1}_{(1, 2]}(x)\). {[}Here we have used the notation for indicator functions — see Example 2.4 in the notes.{]}

\end{enumerate}
\phantomsection\label{\detokenize{Problems:id9}}\begin{enumerate}
\sphinxsetlistlabels{\arabic}{enumi}{enumii}{}{.}%
\setcounter{enumi}{8}
\item {} 
\sphinxAtStartPar
What is the largest subset \(A\) of \(\mathbb{R}\) for which we can specify a function by \(f(x) = \sin\left(\frac{1}{x}\right)\)? Does \(f:A\to\mathbb{R}\) have a limit at \(x = 0\)? {[}Hint: Consider sequences whose \(n\)th term is \(\frac{1}{\theta + 2n\pi}\), and think about good choices for \(\theta\).{]}

\end{enumerate}
\phantomsection\label{\detokenize{Problems:id10}}\begin{enumerate}
\sphinxsetlistlabels{\arabic}{enumi}{enumii}{}{.}%
\setcounter{enumi}{9}
\item {} 
\sphinxAtStartPar
What is the largest subset \(A\) of \(\mathbb{R}\) for which we can specify a function by \(f(x) = x\sin\left(\frac{1}{x}\right)\)? Does \(f:A\to\mathbb{R}\)  have a limit at \(x = 0\)?

\end{enumerate}
\phantomsection\label{\detokenize{Problems:id11}}\begin{enumerate}
\sphinxsetlistlabels{\arabic}{enumi}{enumii}{}{.}%
\setcounter{enumi}{10}
\item {} 
\sphinxAtStartPar
In the notes, we gave meaning to \(\displaystyle\lim_{x \rightarrow a} f(x)\) where \(f:\mathbb{R} \rightarrow \mathbb{R}\) is a function and \(a \in \mathbb{R}\). In this question, you can investigate what happens when \(x\) tends to \(\infty\) or \(-\infty\).

\sphinxAtStartPar
(i) Formulate a rigorous definition, in terms of \(\varepsilon\)’s and \(\delta\)’s, of what it means for \(\displaystyle\lim_{x \rightarrow \infty}f(x)\) and \(\displaystyle\lim_{x \rightarrow -\infty}f(x)\) to exist.

{[}Hint: For the analogue of \((\varepsilon-\delta)\), think very carefully about how you are going to control the behaviour of \(x\). Remember that you cannot treat \(\infty\) as if it were a number!{]}

\sphinxAtStartPar
(ii) Find an analogue of the sequential criterion for this case, and prove the analogous result to Theorem 2.1.

\sphinxAtStartPar
(iii) Check that you can prove that \(\displaystyle\lim_{x \rightarrow \infty}\frac{1}{x} = \lim_{x \rightarrow -\infty}\frac{1}{x} = 0\), using your criterion.

\end{enumerate}
\phantomsection\label{\detokenize{Problems:id12}}\begin{enumerate}
\sphinxsetlistlabels{\arabic}{enumi}{enumii}{}{.}%
\setcounter{enumi}{11}
\item {} 
\sphinxAtStartPar
(i) Formulate a rigorous definition for \(\displaystyle\lim_{x \rightarrow \infty}f(x) = \infty\) in terms of sequences  and write down an analogue of the \((\varepsilon-\delta)\) criterion. {[}The cases  \(\displaystyle\lim_{x \rightarrow \infty}f(x) = -\infty\) and \(\displaystyle\lim_{x \rightarrow -\infty}f(x) = \pm \infty\) can be treated similarly.{]}

\sphinxAtStartPar
(ii) Let \(f:\mathbb{R} \rightarrow [0, \infty)\) and \(g:\mathbb{R} \rightarrow [0, \infty)\), and suppose that \(\displaystyle\lim_{x \rightarrow \infty}f(x) = \infty\) and \(\displaystyle\lim_{x \rightarrow \infty}g(x) = l\), where \(l > 0\). Define \(h(x) = f(x)g(x)\) for all \(x \in \mathbb{R}\). Show that \(\displaystyle\lim_{x \rightarrow \infty}h(x) = \infty\).

\sphinxAtStartPar
(iii) Let \(p: \mathbb{R} \rightarrow \mathbb{R}\) be an even polynomial of degree \(m\), where the leading coefficient (i.e. the coefficient of \(x^{m}\)) is positive. Show that \(\displaystyle\lim_{x \rightarrow \infty}p(x) = \lim_{x \rightarrow -\infty}p(x) = \infty\). What happens when \(m\) is odd?

\end{enumerate}


\subsection{Continuity}
\label{\detokenize{Problems:continuity}}\phantomsection\label{\detokenize{Problems:id13}}\begin{enumerate}
\sphinxsetlistlabels{\arabic}{enumi}{enumii}{}{.}%
\setcounter{enumi}{12}
\item {} 
\sphinxAtStartPar
Return to {\hyperref[\detokenize{Problems:id1}]{\sphinxcrossref{Problem 1}}}. Consider each function there, defined on the largest subset \(A\) of \(\mathbb{R}\)
which can be taken as its domain, as found in that problem. For each function, what is the maximum subset of \(A\) on which it is continuous?

\end{enumerate}
\phantomsection\label{\detokenize{Problems:id14}}\begin{enumerate}
\sphinxsetlistlabels{\arabic}{enumi}{enumii}{}{.}%
\setcounter{enumi}{13}
\item {} 
\sphinxAtStartPar
Let \(f: \mathbb{R} \rightarrow \mathbb{R}\) be continuous at a point \(a\).
Prove that \(|f|\) is continuous at \(a\), where \(|f|(x) = |f(x)|\), for all \(x \in \mathbb{R}\).

{[}Hint: Use the formulation of continuity with sequences (Theorem 3.1(ii)) and Corollary 1.1 from the \sphinxhref{https://rosiesb.github.io/Analysis-Notes}{notes}.{]}

\end{enumerate}
\phantomsection\label{\detokenize{Problems:id15}}\begin{enumerate}
\sphinxsetlistlabels{\arabic}{enumi}{enumii}{}{.}%
\setcounter{enumi}{14}
\item {} 
\sphinxAtStartPar
Prove Theorem 3.3 in the notes (i.e. that composition of continuous functions is continuous). {[}Hint: Use sequences.{]}

\end{enumerate}
\phantomsection\label{\detokenize{Problems:id16}}\begin{enumerate}
\sphinxsetlistlabels{\arabic}{enumi}{enumii}{}{.}%
\setcounter{enumi}{15}
\item {} 
\sphinxAtStartPar
Define \(f:\mathbb{R} \setminus \{0\}\to \mathbb{R}\) and \(g: \mathbb{R} \rightarrow \mathbb{R}\setminus \{0\}\) by \(f(x) = \frac{1}{x}\), and \(g(x) = 1 + x^{2}\). Write down the functions \(f \circ g\), and \(g \circ f\), giving their domains explicitly. What can you say about continuity of these functions?

\end{enumerate}
\phantomsection\label{\detokenize{Problems:id17}}\begin{enumerate}
\sphinxsetlistlabels{\arabic}{enumi}{enumii}{}{.}%
\setcounter{enumi}{16}
\item {} 
\sphinxAtStartPar
For each of the functions in {\hyperref[\detokenize{Problems:id7}]{\sphinxcrossref{Problem 7}}}:

\sphinxAtStartPar
(i) Find the maximum subset of \(\mathbb{R}\) on which it is continuous.

\sphinxAtStartPar
(ii) Identify those discontinuities which are jumps, and calculate the size of each jump.

\end{enumerate}
\phantomsection\label{\detokenize{Problems:id18}}\begin{enumerate}
\sphinxsetlistlabels{\arabic}{enumi}{enumii}{}{.}%
\setcounter{enumi}{17}
\item {} 
\sphinxAtStartPar
(i) Define \(f:\mathbb{R} \setminus \{0\} \rightarrow \mathbb{R}\) by
\begin{equation*}
\begin{split}
    f(x) = \displaystyle\frac{(1 + x)^{2} - 1}{x}.
    \end{split}
\end{equation*}
\sphinxAtStartPar
Find an extension of \(f\) to the whole of \(\mathbb{R}\) that is continuous there.

\sphinxAtStartPar
(ii) Write down the largest subset of \(\mathbb{R}\) which can be taken as the domain \(A\) of the function \(f\) given by
\begin{equation*}
\begin{split}
    f(x) = \displaystyle\frac{x^{3}-8}{x^{2} - 4}
    \end{split}
\end{equation*}
\sphinxAtStartPar
and explain why \(f\) is continuous at every point of \(A\). The complement \(A^{c}\) of \(A\) in \(\mathbb{R}\) comprises two points. Show that \(f\) may be extended to be continuous at only one of these points, and write down this continuous extension.

\end{enumerate}
\phantomsection\label{\detokenize{Problems:id19}}\begin{enumerate}
\sphinxsetlistlabels{\arabic}{enumi}{enumii}{}{.}%
\setcounter{enumi}{18}
\item {} 
\sphinxAtStartPar
Suppose that the function \(g: \mathbb{R} \rightarrow \mathbb{R}\) is continuous at \(a\) with \(g(a) > 0\). Show that there exists \(\delta > 0\) such that \(g(x) > 0\)  for all \( x \in (a - \delta, a + \delta)\). {[}Hint: Assume \(g(a) > 0\) and try a proof by contradiction.{]}

\end{enumerate}
\phantomsection\label{\detokenize{Problems:id20}}\begin{enumerate}
\sphinxsetlistlabels{\arabic}{enumi}{enumii}{}{.}%
\setcounter{enumi}{19}
\item {} 
\sphinxAtStartPar
(i) For any \(x, y \in \mathbb{R}\) show that
\begin{equation*}
\begin{split}
    \max\{x, y\} = \frac{1}{2}(x + y) + \frac{1}{2}|x - y|.
    \end{split}
\end{equation*}
\sphinxAtStartPar
Hence show that if both \(f:A\to \mathbb{R}\) and \(g: B \rightarrow \mathbb{R}\) are continuous at \(a\), then so is \(\max\{f, g\}\), where for all \(x \in A\cap B\),
\begin{equation*}
\begin{split}
    \max\{f, g\}(x)  = \max\{f(x), g(x)\}.
    \end{split}
\end{equation*}
\sphinxAtStartPar
(ii) Find a similar expression for \(\min\{x, y\}\), and hence prove continuity of \(\min\{f, g\}\) at \(a\).

\end{enumerate}
\phantomsection\label{\detokenize{Problems:id21}}\begin{enumerate}
\sphinxsetlistlabels{\arabic}{enumi}{enumii}{}{.}%
\setcounter{enumi}{20}
\item {} 
\sphinxAtStartPar
The aim of this question is to prove the following: the only functions \(f: \mathbb{R} \rightarrow \mathbb{R}\) which are continuous at zero and satisfy \(f(x+y) = f(x) + f(y)\) for all \(x,y \in \mathbb{R}\) are the linear mappings \(f(x) = kx\) for all \(x \in \mathbb{R}\), where \(k \in \mathbb{R}\) is fixed.

Begin by considering \(f: \mathbb{R} \rightarrow \mathbb{R}\), such that \(f(x+y) = f(x) + f(y)\) for all \(x,y \in \mathbb{R}\).

\sphinxAtStartPar
(i) Prove that \(f(0) = 0\).

\sphinxAtStartPar
(ii) Show that \(f(-x) = -f(x)\) for all \(x \in \mathbb{R}\).

\sphinxAtStartPar
(iii) If \(f\) is continuous at zero, prove that it is continuous at every \(x \in \mathbb{R}\).

\sphinxAtStartPar
(iv) If \(f(1) = k\), prove that \(f(n) = kn\) for all \(n\in \mathbb{Z}\).

\sphinxAtStartPar
(v) If \(f(1) = k\), prove that \(f\left(\frac{p}{q}\right) = k\frac{p}{q}\) for all \(\frac{p}{q} \in \mathbb{Q}\).

\sphinxAtStartPar
(vi) If \(f(1) = k\) and \(f\) is continuous at zero, prove that \(f(x) =kx\) for all \(x \in \mathbb{R}\).

\end{enumerate}
\phantomsection\label{\detokenize{Problems:id22}}\begin{enumerate}
\sphinxsetlistlabels{\arabic}{enumi}{enumii}{}{.}%
\setcounter{enumi}{21}
\item {} 
\sphinxAtStartPar
Show that Dirichlet’s “other” function, as discussed in Example 3.8 in the notes, is discontinuous at every rational point in its domain.

\end{enumerate}
\phantomsection\label{\detokenize{Problems:id23}}\begin{enumerate}
\sphinxsetlistlabels{\arabic}{enumi}{enumii}{}{.}%
\setcounter{enumi}{22}
\item {} 
\sphinxAtStartPar
What can you say about left/right continuity of the function  \({\bf 1}_{(a, b)}:\mathbb{R}\to\mathbb{R}\) at the points \(a\) and \(b\)?

\end{enumerate}
\phantomsection\label{\detokenize{Problems:id24}}\begin{enumerate}
\sphinxsetlistlabels{\arabic}{enumi}{enumii}{}{.}%
\setcounter{enumi}{23}
\item {} 
\sphinxAtStartPar
Prove Corollary 3.1. {[}Hint: Apply the intermediate value theorem to the function \(g(x) = f(x) - \gamma\), for \(x \in [a, b]\).{]}

\end{enumerate}
\phantomsection\label{\detokenize{Problems:id25}}\begin{enumerate}
\sphinxsetlistlabels{\arabic}{enumi}{enumii}{}{.}%
\setcounter{enumi}{24}
\item {} 
\sphinxAtStartPar
Use Corollary 3.1

\sphinxAtStartPar
(i) Every continuous function from \(\mathbb{R}\) to \(\mathbb{Z}\) is constant.

\sphinxAtStartPar
(ii) Every continuous function from \(\mathbb{R}\) to \(\mathbb{Q}\) is constant.

\end{enumerate}
\phantomsection\label{\detokenize{Problems:id26}}\begin{enumerate}
\sphinxsetlistlabels{\arabic}{enumi}{enumii}{}{.}%
\setcounter{enumi}{25}
\item {} 
\sphinxAtStartPar
Prove the following \{\textbackslash{}it fixed point theorem\}: if \(f:[a,b] \rightarrow (a,b)\) is continuous, then there exists \(c \in (a, b)\) such that \(f(c) = c\).

{[}Hint: This is a similar proof to that of {\hyperref[\detokenize{Problems:id24}]{\sphinxcrossref{Problem 24}}}. This time you need to consider a function of the form \(g(x) = f(x) - \) (\sphinxstyleemphasis{something}). What is \sphinxstyleemphasis{something}?{]} Give a counter\sphinxhyphen{}example to demonstrate that the claim is false if the domain of \(f\) is restricted to \((0, 1)\).

\end{enumerate}
\phantomsection\label{\detokenize{Problems:id27}}\begin{enumerate}
\sphinxsetlistlabels{\arabic}{enumi}{enumii}{}{.}%
\setcounter{enumi}{26}
\item {} 
\sphinxAtStartPar
Complete the proof of the extreme value theorem (Theorem 3.5), i.e. prove that a continuous function \(f:[a,b]\to\mathbb{R}\) attains its infimum.

\end{enumerate}
\phantomsection\label{\detokenize{Problems:id28}}\begin{enumerate}
\sphinxsetlistlabels{\arabic}{enumi}{enumii}{}{.}%
\setcounter{enumi}{27}
\item {} 
\sphinxAtStartPar
Show that if \(f:[0,1] \rightarrow \mathbb{R}\) is continuous and \(0\) is not in the range of  \(f\), then the function \(\frac{1}{f}:[0,1]\to \mathbb{R}\) is bounded, where for each \(x \in [0,1], \left(\frac{1}{f}\right)(x) = \frac{1}{f(x)}\).

\end{enumerate}
\phantomsection\label{\detokenize{Problems:id29}}\begin{enumerate}
\sphinxsetlistlabels{\arabic}{enumi}{enumii}{}{.}%
\setcounter{enumi}{28}
\item {} 
\sphinxAtStartPar
Explain why there are no continuous functions having domain \([0, 1]\) and range \(\mathbb{R}\). {[}Hint: Use Theorem 3.5.{]}

\end{enumerate}
\phantomsection\label{\detokenize{Problems:id30}}\begin{enumerate}
\sphinxsetlistlabels{\arabic}{enumi}{enumii}{}{.}%
\setcounter{enumi}{29}
\item {} 
\sphinxAtStartPar
Suppose that the function \(f:[0,1] \rightarrow [0,1]\) is continuous, and fix \(0 < r < 1\). Suppose that we are given a sequence \((x_{n})\) in \([0,1]\) such that \(f(x_{n+1}) \leq rf(x_{n})\) for all \(n\in\mathbb{N}\). Show that there exists \(c \in [0, 1]\) for which \(f(c) = 0\). {[}Hint: Use the Bolzano\sphinxhyphen{}Weierstrass theorem.{]}

\end{enumerate}
\phantomsection\label{\detokenize{Problems:id31}}\begin{enumerate}
\sphinxsetlistlabels{\arabic}{enumi}{enumii}{}{.}%
\setcounter{enumi}{30}
\item {} 
\sphinxAtStartPar
Show that for each \(n\in\mathbb{N}\), the function \(f: [0, \infty)\to\mathbb{R}\) given by \(f(x) =x^{n}\) is strictly monotonic increasing.

\end{enumerate}
\phantomsection\label{\detokenize{Problems:id32}}\begin{enumerate}
\sphinxsetlistlabels{\arabic}{enumi}{enumii}{}{.}%
\setcounter{enumi}{31}
\item {} 
\sphinxAtStartPar
Show that the function \(f(x) = \sin(x)\) has a continuous inverse when restricted to the interval \(\left[-\frac{\pi}{2}, \frac{\pi}{2}\right]\). What goes wrong outside this interval? {[}Hint: Use an appropriate trigonometric identity.{]}

\end{enumerate}
\phantomsection\label{\detokenize{Problems:id33}}\begin{enumerate}
\sphinxsetlistlabels{\arabic}{enumi}{enumii}{}{.}%
\setcounter{enumi}{32}
\item {} 
\sphinxAtStartPar
Find \(\displaystyle\lim_{x \rightarrow 1}\frac{1 - x}{1 - x^{\frac{m}{n}}}\), where \(m, n \in \mathbb{N}\).

\end{enumerate}
\phantomsection\label{\detokenize{Problems:id34}}\begin{enumerate}
\sphinxsetlistlabels{\arabic}{enumi}{enumii}{}{.}%
\setcounter{enumi}{33}
\item {} 
\sphinxAtStartPar
If \(f:[a, b] \rightarrow \mathbb{R}\) is continuous and bijective, and \(f(a) < f(b)\), show that \(f\) is strictly monotonic increasing.

\end{enumerate}
\phantomsection\label{\detokenize{Problems:id35}}\begin{enumerate}
\sphinxsetlistlabels{\arabic}{enumi}{enumii}{}{.}%
\setcounter{enumi}{34}
\item {} 
\sphinxAtStartPar
\(^*\) Show that if \(f, g: \mathbb{R} \rightarrow \mathbb{R}\)  are monotonic increasing, with \(f+g\) continuous, then both \(f\) and \(g\) are continuous.

\end{enumerate}
\phantomsection\label{\detokenize{Problems:id36}}\begin{enumerate}
\sphinxsetlistlabels{\arabic}{enumi}{enumii}{}{.}%
\setcounter{enumi}{35}
\item {} 
\sphinxAtStartPar
\(^*\)

\sphinxAtStartPar
(i) Let \((x_{n})\) be a sequence in \(\mathbb{R}\) for which \(x_{1} = a > 0\) and \(x_{n+1} = \sqrt{x_{n}}\), for all \(n\in\mathbb{N}\). Show that \(\displaystyle\lim_{n \rightarrow \infty} x_{n} = 1\).

\sphinxAtStartPar
(ii) Let \(f: \mathbb{R} \rightarrow \mathbb{R}\) be a continuous function for which \(f(x) = f(x^{2})\) for all \(x \in \mathbb{R}\). Use the result of (i) to show that \(f\) is constant.

\end{enumerate}


\subsection{Differentiation}
\label{\detokenize{Problems:differentiation}}\phantomsection\label{\detokenize{Problems:id37}}\begin{enumerate}
\sphinxsetlistlabels{\arabic}{enumi}{enumii}{}{.}%
\setcounter{enumi}{36}
\item {} 
\sphinxAtStartPar
Let \(f:A\to\mathbb{R}\) and let \(a\in\mathbb{R}\) be such that there is some sequence \((x_n)\) in \(A\) with
\(x_n\neq a\) for all \(n\) and \(\displaystyle\lim_{n\to\infty} x_n=a\). What does it mean to say that
the function \(f\) has limit \(l\) at the point \(a\)? What notation do we use to write this?

{[}This question is asking you to recall (or revise if you can’t recall) Definition 2.2 from the lecture notes.That’s a key definition and we build on it when we define what it means to be differentiable — see the next question.{]}

\end{enumerate}
\phantomsection\label{\detokenize{Problems:id38}}\begin{enumerate}
\sphinxsetlistlabels{\arabic}{enumi}{enumii}{}{.}%
\setcounter{enumi}{37}
\item {} 
\sphinxAtStartPar
(i) Give the definition of what it means for a function \(f:A\to\mathbb{R}\) to be differentiable at \(a \in A\).

\sphinxAtStartPar
(ii) State carefully what this means in terms of limits of sequences.

\sphinxAtStartPar
(iii) State carefully what this means in terms of the \(\varepsilon-\delta\) criterion.
{[}The first part is just asking you to give Definition 4.1 and the other parts are checking that you know what this means.

The key points from Chapter 2 are \sphinxcode{\sphinxupquote{Definition }}2.2 of a limit of a function and Theorem 2.1 giving the sequential criterion. But you need to see how to apply these, not just copy them out. Here they have to be applied to the relevant function for the definition of differentiability, not \(f\) itself. {]}

\end{enumerate}
\phantomsection\label{\detokenize{Problems:id39}}\begin{enumerate}
\sphinxsetlistlabels{\arabic}{enumi}{enumii}{}{.}%
\setcounter{enumi}{38}
\item {} 
\sphinxAtStartPar
Give a rigorous proof that the function \(f:\mathbb{R}\setminus\{0\}\to \mathbb{R}\) given by
\(f(x) = \frac{1}{x}\) is differentiable for all \(x \in \mathbb{R} \setminus \{0\}\) from the definition of the derivative and find \(f'(x)\) explicitly. Can we extend the function so that it is differentiable on the whole of \(\mathbb{R}\) by defining its value at zero to be zero?

\end{enumerate}
\phantomsection\label{\detokenize{Problems:id40}}\begin{enumerate}
\sphinxsetlistlabels{\arabic}{enumi}{enumii}{}{.}%
\setcounter{enumi}{39}
\item {} 
\sphinxAtStartPar
Let \(k\in \mathbb{R}\setminus\{0\}\). Give a rigorous proof from the definition of the derivative that  \(f:\mathbb{R}\to \mathbb{R}\) given by  \(f(x) = e^{kx}\) is differentiable for all \(x \in \mathbb{R}\), and find \(f'(x)\) explicitly.

{[}Hint: For fixed \(k\in \mathbb{R}\), let \(g:\mathbb{R}\to\mathbb{R}\) be given by \(g(h)=e^{kh} - 1 - kh \).
You may use the fact that  \(\displaystyle\lim_{h \rightarrow 0}\frac{g(h)}{h} = 0\). (This follows from the series expansion of the exponential function, which you already know, and we will study more later.) {]}

\end{enumerate}
\phantomsection\label{\detokenize{Problems:id41}}\begin{enumerate}
\sphinxsetlistlabels{\arabic}{enumi}{enumii}{}{.}%
\setcounter{enumi}{40}
\item {} 
\sphinxAtStartPar
Consider the function  \(f:\mathbb{R}\to \mathbb{R}\) given by \(f(x) = \begin{cases} x\sin\left(\frac{1}{x}\right) & \text{if }x \neq 0,\\ 0 & \text{if }x = 0.\end{cases}\)

\sphinxAtStartPar
(i) Show that \(f\) is differentiable at every \(x \neq 0\). (You can use standard derivatives and facts about derivatives, such as the product and chain rules.)

\sphinxAtStartPar
(ii) Show that \(f\) is not differentiable at \(x = 0\). (Use the definition of the derivative as a limit. You may assume that \(\sin\left(\frac{1}{x}\right)\) has no limit as \(x\) tends to \(0\). This was {\hyperref[\detokenize{Problems:id8}]{\sphinxcrossref{Problem 8}}}.)

\end{enumerate}
\phantomsection\label{\detokenize{Problems:id42}}\begin{enumerate}
\sphinxsetlistlabels{\arabic}{enumi}{enumii}{}{.}%
\setcounter{enumi}{41}
\item {} 
\sphinxAtStartPar
Show that the function  \(f:\mathbb{R}\to \mathbb{R}\) given by

\end{enumerate}
\begin{equation*}
\begin{split}
f(x) = \begin{cases} x^2\sin\left(\frac{1}{x}\right) & \text{if }x \neq 0,\\ 0 & \text{if }x = 0.\end{cases}
\end{split}
\end{equation*}
\sphinxAtStartPar
is differentiable at every \(x \in \mathbb{R}\). What can you say about its second derivative?

{[}Hint: You will need to consider the case \(x=0\) separately from \(x\in\mathbb{R}\setminus\{0\}\). You may assume that the limit as \(x\) tends to \(0\) of \(x\sin\left(\frac{1}{x}\right)\) exists and equals \(0\). This was studied in {\hyperref[\detokenize{Problems:id10}]{\sphinxcrossref{\DUrole{std,std-ref}{Problem 10}}}}.{]}

\phantomsection\label{\detokenize{Problems:id43}}\begin{enumerate}
\sphinxsetlistlabels{\arabic}{enumi}{enumii}{}{.}%
\setcounter{enumi}{42}
\item {} 
\sphinxAtStartPar
(i) Sketch the graph of any continuous function \(f:[0,1]\to \mathbb{R}\) which is not differentiable at \(x=\frac{1}{2}\), but is differentiable at all other points. {[}You do \sphinxstyleemphasis{not} need to give a formula.{]}

\sphinxAtStartPar
(ii) Sketch the graph of any continuous function \(f:[0,1]\to \mathbb{R}\) which is not differentiable at \(x=\frac{1}{3}\) and is not differentiable at \(x=\frac{2}{3}\), but is differentiable at all other points. {[}You do \sphinxstyleemphasis{not} need to give a formula.{]}

\end{enumerate}
\phantomsection\label{\detokenize{Problems:id44}}\begin{enumerate}
\sphinxsetlistlabels{\arabic}{enumi}{enumii}{}{.}%
\setcounter{enumi}{43}
\item {} 
\sphinxAtStartPar
Let \(f:\mathbb{R}\to \mathbb{R}\) be given by \(f(x) = x - [x]\) for all \(x \in \mathbb{R}\), where \([x]\) denotes the integer part of \(f\) (see {\hyperref[\detokenize{Problems:equation-eq-x}]{\sphinxcrossref{(1.1)}}}).

Explain carefully at which points \(f\)  is differentiable, and find the value of its derivative there. {[}It will probably help to sketch the graph!{]}

\end{enumerate}
\phantomsection\label{\detokenize{Problems:id45}}\begin{enumerate}
\sphinxsetlistlabels{\arabic}{enumi}{enumii}{}{.}%
\setcounter{enumi}{44}
\item {} 
\sphinxAtStartPar
Show that if \(f:\mathbb{R} \rightarrow \mathbb{R}\) is differentiable at \(a\) then

\end{enumerate}
\begin{equation*}
\begin{split}
\lim_{h \downarrow 0} \displaystyle\frac{f(a + h) - f(a - h)}{2h} = f'(a).
\end{split}
\end{equation*}
\sphinxAtStartPar
By considering \(f:\mathbb{R}\to\mathbb{R}\) given by \(f(x) = |x|\), show that the limit on the left\sphinxhyphen{}hand side may exist, even when \(f\) is not differentiable at \(a\).

\phantomsection\label{\detokenize{Problems:id46}}\begin{enumerate}
\sphinxsetlistlabels{\arabic}{enumi}{enumii}{}{.}%
\setcounter{enumi}{45}
\item {} 
\sphinxAtStartPar
A function \(f:\mathbb{R} \rightarrow \mathbb{R}\) is defined by \(f(x) = \begin{cases} -x^{2} & \text{if }x < 0,\\ x^{2} & \text{if }x \geq 0. \end{cases}\)

Determine whether each of the following is true or false. Justify your answers.

\sphinxAtStartPar
(i) \(f\) is continuous at \(0\).

\sphinxAtStartPar
(ii) \(f'(0)\) exists.

\sphinxAtStartPar
(iii) \(f':\mathbb{R}\to\mathbb{R}\) is continuous at \(0\).

\sphinxAtStartPar
(iv) \(f^{\prime \prime}(0)\) exists.

\end{enumerate}
\phantomsection\label{\detokenize{Problems:id47}}\begin{enumerate}
\sphinxsetlistlabels{\arabic}{enumi}{enumii}{}{.}%
\setcounter{enumi}{46}
\item {} 
\sphinxAtStartPar
(i) Must any differentiable function \(f:[a, b] \rightarrow \mathbb{R}\) have a maximum and minimum value? Why?

\sphinxAtStartPar
(ii) If \(f:[a, b] \rightarrow \mathbb{R}\) is differentiable  and \(f(a) = f(b)\), must \(f\) have a maximum and minimum value in \((a, b)\)?

\end{enumerate}
\phantomsection\label{\detokenize{Problems:id48}}\begin{enumerate}
\sphinxsetlistlabels{\arabic}{enumi}{enumii}{}{.}%
\setcounter{enumi}{47}
\item {} 
\sphinxAtStartPar
Let \(a_{0}, a_{1}, \ldots, a_{n}\) be real numbers such that

\end{enumerate}
\begin{equation*}
\begin{split}
a_{0} + \frac{a_{1}}{2} + \frac{a_{2}}{3} + \cdots + \frac{a_{n}}{n+1} = 0.
\end{split}
\end{equation*}
\sphinxAtStartPar
Consider \(f:\mathbb{R}\to\mathbb{R}\) given by \(f(x) = a_{0}+ a_{1}x + a_{2}x^{2} + \cdots + a_{n}x^{n}\). Show that there is some \(c\in (0,1)\) such that \(f(c)=0\). {[}Hint: Integrate the function \(f\) term–by–term, and think about how to use Rolle’s theorem.{]}

\phantomsection\label{\detokenize{Problems:id49}}\begin{enumerate}
\sphinxsetlistlabels{\arabic}{enumi}{enumii}{}{.}%
\setcounter{enumi}{48}
\item {} 
\sphinxAtStartPar
(i) Use the mean value theorem to show the following. If \(f:\mathbb{R} \rightarrow \mathbb{R}\) is continuous on \([a, b]\) and differentiable on \((a, b)\) with \(f'(c)= 0\) for all \(c \in (a, b)\), then \(f\) is constant on \([a, b]\).

\sphinxAtStartPar
(ii) Use part (i) to show that if \(g, h:\mathbb{R}\to\mathbb{R}\) are both continuous on \([a, b]\) and differentiable on \((a, b)\) with \(h'(x) = g'(x)\) for all \(x \in (a, b)\), then there exists \(k \in \mathbb{R}\) such that \(h(x) = g(x) + k\), for all \(x \in [a, b]\).

\end{enumerate}
\phantomsection\label{\detokenize{Problems:id50}}\begin{enumerate}
\sphinxsetlistlabels{\arabic}{enumi}{enumii}{}{.}%
\setcounter{enumi}{49}
\item {} 
\sphinxAtStartPar
If \(f:\mathbb{R} \rightarrow \mathbb{R}\) is continuous on \([a, b]\) and differentiable on \((a, b)\) and there exist \(m, M \in \mathbb{R}\) such that \(m \leq f'(c) \leq M\), for all \(c \in (a, b)\), show that

\end{enumerate}
\begin{equation*}
\begin{split}
f(a) + m(b - a) \leq f(b) \leq f(a) + M(b-a).
\end{split}
\end{equation*}\phantomsection\label{\detokenize{Problems:id51}}\begin{enumerate}
\sphinxsetlistlabels{\arabic}{enumi}{enumii}{}{.}%
\setcounter{enumi}{50}
\item {} 
\sphinxAtStartPar
If \(r > 0\) and \(q \in \mathbb{R}\) show that the polynomial \(p(x) = x^{3} + rx + q\) has exactly one real zero.
{[} You may assume the result of Corollary 3.2: \(p\) has at least one real root. So you need to show that there can’t be more than one. {]}

\end{enumerate}
\phantomsection\label{\detokenize{Problems:id52}}\begin{enumerate}
\sphinxsetlistlabels{\arabic}{enumi}{enumii}{}{.}%
\setcounter{enumi}{51}
\item {} 
\sphinxAtStartPar
Let \(r>1\) and fix \(y\in (0,1)\). By applying the mean value theorem to the function \(f(x)=x^r\) on \([y,1]\), show that

\end{enumerate}
\begin{equation*}
\begin{split}
1 - y^r < r(1 - y).
\end{split}
\end{equation*}\phantomsection\label{\detokenize{Problems:id53}}\begin{enumerate}
\sphinxsetlistlabels{\arabic}{enumi}{enumii}{}{.}%
\setcounter{enumi}{52}
\item {} 
\sphinxAtStartPar
Let \(f:\mathbb{R} \rightarrow \mathbb{R}\) be twice differentiable at \(a\) with \(f'(a) = 0\). If \(f^{\prime \prime}(a) < 0\), show that \(f\) has a local maximum at \(a\), while if \(f^{\prime \prime}(a) > 0\), show that \(f\) has a local minimum at \(a\).

\end{enumerate}


\subsection{Sequences and series of functions}
\label{\detokenize{Problems:sequences-and-series-of-functions}}\phantomsection\label{\detokenize{Problems:id54}}\begin{enumerate}
\sphinxsetlistlabels{\arabic}{enumi}{enumii}{}{.}%
\setcounter{enumi}{53}
\item {} 
\sphinxAtStartPar
Consider the sequence of functions  \((f_n)\), where \(f_n:[0,\pi ]\to \mathbb{R}\) is defined by \(f_n(x) = \sin^n (x)\) for each \(n\in\mathbb{N}\). Show that the sequence \((f_n)\) converges pointwise. Does the sequence \((f_n)\)  converge uniformly? Justify your answer.

\end{enumerate}
\phantomsection\label{\detokenize{Problems:id55}}\begin{enumerate}
\sphinxsetlistlabels{\arabic}{enumi}{enumii}{}{.}%
\setcounter{enumi}{54}
\item {} 
\sphinxAtStartPar
For each of the following sequences of functions \((f_n)\) determine the pointwise limit (if it exists), and decide whether \((f_n)\) converges uniformly to this limit.

\sphinxAtStartPar
(i) \(f_n:[0,1]\to\mathbb{R}\), \(f_n (x) = x^{\frac{1}{n}}\).

\sphinxAtStartPar
(ii) \(f_n:\mathbb{R}\to\mathbb{R}\), where \(\displaystyle f_n (x)  = \left\{ \begin{array}{ll} 0 & x\leq n, \\ x-n & x\geq n \\ \end{array} \right.\).

\sphinxAtStartPar
(iii) \(f_n:(1,\infty)\to\mathbb{R}\), \(f_n(x) = \frac{e^x}{x^n}\).

\sphinxAtStartPar
(iv) \(f_n:[-1,1]\to\mathbb{R}\), \(f_n(x) = e^{-nx^2}\).

\sphinxAtStartPar
(v) \(f_n:\mathbb{R}\to\mathbb{R}\), \(f_n(x) = e^{-x^2}{n}\).

\end{enumerate}
\phantomsection\label{\detokenize{Problems:id56}}\begin{enumerate}
\sphinxsetlistlabels{\arabic}{enumi}{enumii}{}{.}%
\setcounter{enumi}{55}
\item {} 
\sphinxAtStartPar
For each of the following sequences of functions \((g_n)\) find the pointwise limit, and determine whether the sequence converges uniformly on \([0,1]\), and on \([0,\infty)\).

\sphinxAtStartPar
(i) \(\displaystyle g_n(x) = \frac{x}{n}\).

\sphinxAtStartPar
(ii) \(\displaystyle g_n(x) = \frac{x^n}{1+x^n}\).

\sphinxAtStartPar
(iii) \(\displaystyle g_n (x) = \frac{x^n}{n+x^n}\).

\end{enumerate}
\phantomsection\label{\detokenize{Problems:id57}}\begin{enumerate}
\sphinxsetlistlabels{\arabic}{enumi}{enumii}{}{.}%
\setcounter{enumi}{56}
\item {} 
\sphinxAtStartPar
For each of the following sequences of functions \((h_n)\), where \(h_n\colon [0,1]\rightarrow \mathbb{R}\), find the pointwise limit, if it exists, and in that case determine whether the sequence converges uniformly.

\sphinxAtStartPar
(i) \(h_n(x) = \left(1-\frac{x}{n}\right)^2\).

\sphinxAtStartPar
(ii) \(h_n(x) = x-x^n\).

\sphinxAtStartPar
(iii) \(h_n (x) = \sum_{k=0}^n x^k\).

\end{enumerate}
\phantomsection\label{\detokenize{Problems:id58}}\begin{enumerate}
\sphinxsetlistlabels{\arabic}{enumi}{enumii}{}{.}%
\setcounter{enumi}{57}
\item {} 
\sphinxAtStartPar
Define \(f_n\colon \mathbb{R}\rightarrow \mathbb{R}\) by \(f_n (x) = \frac{n+\cos x}{2n+\sin^2 x}\).

\sphinxAtStartPar
(i) Find the pointwise limit of the sequence of functions \((f_n)\).

\sphinxAtStartPar
(ii) Show that the sequence \((f_n)\) converges uniformly.

\sphinxAtStartPar
(iii) Calculate \(f_n'\) and show that \(f'(x) = \lim_{n\rightarrow \infty} f_n'(x)\). Does \((f_n)\) satisfy the conditions of Theorem 5.2?

\end{enumerate}
\phantomsection\label{\detokenize{Problems:id59}}\begin{enumerate}
\sphinxsetlistlabels{\arabic}{enumi}{enumii}{}{.}%
\setcounter{enumi}{58}
\item {} 
\sphinxAtStartPar
(i) Let \(n\in \mathbb{N}\). Show that we can define a continuous function \(f_n\colon [0,1]\rightarrow \mathbb{R}\) by
\begin{equation*}
\begin{split}
    f_n(x) = \left\{ \begin{array}{ll}
    \displaystyle 0 & x=0, \\
    \displaystyle\frac{x^{\frac{1}{n}}-1}{\ln x} & 0<x<1, \\
    \displaystyle\frac{1}{n} & x=1. \\
    \end{array} \right.
    \end{split}
\end{equation*}
\sphinxAtStartPar
(Note: you only need check continuity at \(x=0\) and \(x=1\).)

\sphinxAtStartPar
(ii) Does the sequence \((f_n)\) converge  uniformly to a limit? Justify your answer. If you wish, you may assume without proof that each function \(f_n\) is monotone increasing.

\end{enumerate}
\phantomsection\label{\detokenize{Problems:id60}}\begin{enumerate}
\sphinxsetlistlabels{\arabic}{enumi}{enumii}{}{.}%
\setcounter{enumi}{59}
\item {} 
\sphinxAtStartPar
Let \(f_n:[a,b]\to\mathbb{R}\). Suppose that \(\sum_{n=1}^\infty f_n\) converges uniformly to \(f\). Show that the sequence \((f_n)\) converges uniformly to the zero function.

\end{enumerate}
\phantomsection\label{\detokenize{Problems:id61}}\begin{enumerate}
\sphinxsetlistlabels{\arabic}{enumi}{enumii}{}{.}%
\setcounter{enumi}{60}
\item {} 
\sphinxAtStartPar
The functions \(c,s:\mathbb{R}\to\mathbb{R}\) are defined by the infinite series \(s(x) = \sum_{n=0}^\infty\frac{(-1)^nx^{2n+1}}{(2n+1)!}\), \(c(x) = \sum_{n=0}^\infty\frac{(-1)^kx^{2n}}{(2n)!}\).

\sphinxAtStartPar
(i) Prove that these functions are well\sphinxhyphen{}defined as pointwise limits, and that their series converge absolutely.

\sphinxAtStartPar
(ii) Show that when restricted to a closed bounded interval \([-R,R]\), where \(R>0\), both series’ converge uniformly.

\sphinxAtStartPar
(iii) Prove that \(s\) and \(c\) are differentiable everywhere, with \(s'(x)=c(x)\) and \(c'(x)=-s(x)\).
{[}Hint: Mimic the approach of Example 5.7 in the notes.{]}

\sphinxAtStartPar
(iv) Prove that \(c^2+s^2=1\). {[}Hint: Differentiate the function \(f:\mathbb{R}\to\mathbb{R}\) given by \(f(x)=c(x)^2+s(x)^2\) for each \(x\).{]}

\sphinxAtStartPar
(v)* Show that \(\exp(ix)=c(x)+is(x)\) for all \(x\in\mathbb{R}\), where \(\exp\) is the function from Example 5.7. You should justify any rearrangements of terms in infinite series (MAS107 Theorem 4.18 may help).

\end{enumerate}
\phantomsection\label{\detokenize{Problems:id62}}\begin{enumerate}
\sphinxsetlistlabels{\arabic}{enumi}{enumii}{}{.}%
\setcounter{enumi}{61}
\item {} 
\sphinxAtStartPar
By using the Weierstrass \(M\)\sphinxhyphen{}test or otherwise, for each of the following series, determine whether it  converges uniformly on \(\mathbb{R}\) and  whether it converges uniformly on \([0,1]\).

\sphinxAtStartPar
(i) \(\displaystyle\sum_{n=1}^\infty \frac{1}{n^2 +x^2}\)

\sphinxAtStartPar
(ii) \(\displaystyle\sum_{n=0}^\infty \frac{(-1)^nx^{2n+1}}{(2n+1)!}\)

\sphinxAtStartPar
(iii) \(\displaystyle\sum_{n=1}^\infty \sin (nx)\)

\sphinxAtStartPar
(iv)* \(\displaystyle\sum_{n=1}^\infty \frac{\sin (nx)}{n}\)

\end{enumerate}
\phantomsection\label{\detokenize{Problems:id63}}\begin{enumerate}
\sphinxsetlistlabels{\arabic}{enumi}{enumii}{}{.}%
\setcounter{enumi}{62}
\item {} 
\sphinxAtStartPar
(i) Show the series \(\sum_{n=1}^\infty x^n\) converges uniformly for \(x\in [0,a]\) whenever \(0<a<1\).

\sphinxAtStartPar
(ii) Does the series converge uniformly on \([0,1)\)\textasciitilde{}? Explain.

\end{enumerate}
\phantomsection\label{\detokenize{Problems:id64}}\begin{enumerate}
\sphinxsetlistlabels{\arabic}{enumi}{enumii}{}{.}%
\setcounter{enumi}{63}
\item {} 
\sphinxAtStartPar
Prove that there is a function \(f\colon [0,2\pi] \rightarrow \mathbb{R}\) defined by

\end{enumerate}
\begin{equation*}
\begin{split}
f(x) = \sum_{n=1}^\infty \frac{\sin n x }{n^2}
\end{split}
\end{equation*}
\sphinxAtStartPar
and that this function is continuous.

\phantomsection\label{\detokenize{Problems:id65}}
\sphinxAtStartPar
65.* This question is about the controversial series presented by Fourier in 1807, when he first attempted to publish his solution to the diffusion equation (for a rectangular lamina).
\begin{equation}\label{equation:Problems:fourier}
\begin{split}\frac{4}{\pi}\left[\cos\left(\frac{\pi x}{2}\right)-\frac{1}{3}\cos\left(\frac{3\pi x}{2}\right)+\frac{1}{5}\cos\left(\frac{5\pi x}{2}\right)-\frac{1}{7}\cos\left(\frac{7\pi x}{2}\right)+\ldots\right]\end{split}
\end{equation}
\sphinxAtStartPar
While a perfectly good series, mathematical understanding in europe at the time was simply not equipped to make any sense of it. Completing this question may give you a sense of why Fourier’s contemporaries found it so disturbing.

\sphinxAtStartPar
\sphinxstylestrong{Disclaimer:} I am not a historian. For more on this fascinating subject, I recommend reading Chapter 1 of \sphinxhref{https://find.shef.ac.uk/permalink/f/1lephdb/44SFD\_ALMA\_DS21193257230001441}{A Radical Approach to Real Analysis by Bressoud}.

\sphinxAtStartPar
(i) Using \sphinxhref{https://www.desmos.com/calculator/bd3xikfhb0}{Desmos} or otherwise, determine (or make an educated guess) what function \(f:\mathbb{R}\to\mathbb{R}\) is represented by the Fourier series {\hyperref[\detokenize{Problems:equation-fourier}]{\sphinxcrossref{(1.2)}}}, and sketch its graph. What is its period?

\sphinxAtStartPar
(ii) Evaluate the series when \(x\) is an odd integer. What do you notice? Do you need to adjust your answer to (i)?

\sphinxAtStartPar
(iii) Calculate \(\text{Dom}(f')\), and write down a formula for \(f'\).

\sphinxAtStartPar
(iv) Now differentiate the series {\hyperref[\detokenize{Problems:equation-fourier}]{\sphinxcrossref{(1.2)}}} term by term. Does this series converge pointwise for any values of \(x\)? You may find Desmos or a similar graphing tool helpful for this part.


\subsection{Integration}
\label{\detokenize{Problems:integration}}\phantomsection\label{\detokenize{Problems:id66}}\begin{enumerate}
\sphinxsetlistlabels{\arabic}{enumi}{enumii}{}{.}%
\setcounter{enumi}{65}
\item {} 
\sphinxAtStartPar
Let \(f:[1,4]\to\mathbb{R}\); \(\displaystyle f(x)=\frac{1}{x}\). Let \(P\) be the partition consisting of points \(\left\{1,\frac{3}{2},2,4\right\}\).

\sphinxAtStartPar
(i) Compute \(L(f,P)\), \(U(f,P)\) and \(U(f,P)-L(f,P)\).

\sphinxAtStartPar
(ii) What happens to the value of \(U(f,P)-L(f,P)\) when we add the point \(3\) to the partition?

\sphinxAtStartPar
(iii) Find a partition \(P'\) of \([1,4]\) for which \(U(f,P')-L(f,P')<\frac{2}{5}\).

\end{enumerate}
\phantomsection\label{\detokenize{Problems:id67}}\begin{enumerate}
\sphinxsetlistlabels{\arabic}{enumi}{enumii}{}{.}%
\setcounter{enumi}{66}
\item {} 
\sphinxAtStartPar
Let \(g:[0,1]\to\mathbb{R}\); \(\displaystyle g(x)=\left\{\begin{array}{cc} 1 & \text{for } 0\leq x<1 \\ 2 &\text{for } x=1 \end{array}\right.\).

\sphinxAtStartPar
(i) Show that \(L(g,P)=1\) for every partition \(P\) of \([0,1]\).

\sphinxAtStartPar
(ii) Construct a partition \(P\) for which \(U(g,P)<1+\frac{1}{10}\)

\sphinxAtStartPar
(iii) Given \(\varepsilon>0\), construct a partition \(P_\varepsilon\) such that \(U(g,P_\varepsilon)<1+\varepsilon\).

\end{enumerate}
\phantomsection\label{\detokenize{Problems:id68}}\begin{enumerate}
\sphinxsetlistlabels{\arabic}{enumi}{enumii}{}{.}%
\setcounter{enumi}{67}
\item {} 
\sphinxAtStartPar
(Exam question)

\sphinxAtStartPar
(i) Define step functions \(r,s\colon \mathbb{R} \rightarrow \mathbb{R}\) by
\begin{equation*}
\begin{split}
    r = {\bf{1}}_{[0,1)} + e {\bf{1}}_{[1,2)} + e^4 {\bf{1}}_{[2,3]}, \qquad  s = e {\bf{1}}_{[0,1]} + e^4 {\bf{1}}_{(1,2]} + e^9 {\bf{1}}_{(2,3]}.
    \end{split}
\end{equation*}
\sphinxAtStartPar
Evaluate the integrals
\begin{equation*}
\begin{split}
    \int_0^3 r(x)dx, \qquad \int_0^3 s(x)dx.
    \end{split}
\end{equation*}
\sphinxAtStartPar
(ii) Why is the function \(f\colon [0,3]\rightarrow \mathbb{R}\) defined by \(f(x)=e^{x^2}\) Riemann integrable?

\sphinxAtStartPar
(iii) Prove that
\begin{equation*}
\begin{split}
    1+e+e^4 \leq \int_0^3 e^{x^2}\ dx \leq e+e^4+e^9.
    \end{split}
\end{equation*}
\sphinxAtStartPar
{[}Hint: there’s no need to calculate the integral in the middle; use the previous parts to prove the inequalities.{]}

\end{enumerate}
\phantomsection\label{\detokenize{Problems:id69}}\begin{enumerate}
\sphinxsetlistlabels{\arabic}{enumi}{enumii}{}{.}%
\setcounter{enumi}{68}
\item {} 
\sphinxAtStartPar
Let \(f,g:[a,b]\to\mathbb{R}\) be integrable functions.

\sphinxAtStartPar
(i) Show that if \(P\) is a partition of \([a,b]\), then
\begin{equation*}
\begin{split}
    U(f+g,P) \leq U(f,P)+U(g,P)
    \end{split}
\end{equation*}
\sphinxAtStartPar
and
\begin{equation*}
\begin{split}
    L(f+g,P) \geq L(f,P)+L(g,P).
    \end{split}
\end{equation*}
\sphinxAtStartPar
Can you think of a particular example where these inequalities are strict?

\sphinxAtStartPar
(ii) Let \(P_1\) and \(P_2\) be partitions of \([a,b]\). Show that
\begin{equation*}
\begin{split}
    U(f+g,P_1\cup P_2) \leq U(f,P_1) + U(g,P_2)
    \end{split}
\end{equation*}
\sphinxAtStartPar
and
\begin{equation*}
\begin{split}
    L(f+g,P_1\cup P_2) \geq L(f,P_1) + L(g,P_2)
    \end{split}
\end{equation*}
\sphinxAtStartPar
{[}Hint: Lemma 6.1 may be helpful.{]}

\sphinxAtStartPar
(iii) Deduce that
\begin{equation*}
\begin{split}
    L(f+g) \geq \int_a^bf(x)dx + \int_a^bg(x)dx
    \end{split}
\end{equation*}
\sphinxAtStartPar
and
\begin{equation*}
\begin{split}
    U(f+g) \leq \int_a^bf(x)dx + \int_a^bg(x)dx.
    \end{split}
\end{equation*}
\sphinxAtStartPar
Explain why this means \(f+g\) must be integrable, with
\begin{equation*}
\begin{split}
    \int_a^b(f(x)+g(x))dx = \int_a^bf(x)dx + \int_a^bg(x)dx.
    \end{split}
\end{equation*}
\end{enumerate}
\phantomsection\label{\detokenize{Problems:id70}}\begin{enumerate}
\sphinxsetlistlabels{\arabic}{enumi}{enumii}{}{.}%
\setcounter{enumi}{69}
\item {} 
\sphinxAtStartPar
Show that if \(k\in\mathbb{R}\) and \(f:[a,b]\to\mathbb{R}\) is integrable, then so is \(kf:[a,b]\to\mathbb{R}\); \(x\mapsto kf(x)\), and

\end{enumerate}
\begin{equation*}
\begin{split}
\int_a^b kf(x)dx = k\int_a^bf(x).
\end{split}
\end{equation*}
\sphinxAtStartPar
{[}Hint: It may help to treat the cases \(k\geq 0\) and \(k<0\) separately.{]}

\phantomsection\label{\detokenize{Problems:id71}}\begin{enumerate}
\sphinxsetlistlabels{\arabic}{enumi}{enumii}{}{.}%
\setcounter{enumi}{70}
\item {} 
\sphinxAtStartPar
Let \(f\) be bounded on a set \(A\subseteq\mathbb{R}\), let \(M=\sup\{f(x):x\in A\}, \; m=\inf\{f(x):x\in A\}\), and let \(M'=\sup\{|f(x)|:x\in A\}, \; \text{ and } \; m'=\inf\{|f(x)|:x\in A\}\).

\sphinxAtStartPar
(i) Show that \(M-m\geq M'-m'\).

\sphinxAtStartPar
(ii) Show that if \(f\) is integrable \([a,b]\), then
\begin{equation*}
\begin{split}
    U(|f|,P)-L(|f|,P) \leq U(f,P) - L(f,P)
    \end{split}
\end{equation*}
\sphinxAtStartPar
for any partition \(P\) of \([a,b]\).

\sphinxAtStartPar
(iii) Complete the proof of Proposition 6.3(iii) that is, show that \(|f|\) is integrable whenever \(f\) is.

\end{enumerate}
\phantomsection\label{\detokenize{Problems:id72}}\begin{enumerate}
\sphinxsetlistlabels{\arabic}{enumi}{enumii}{}{.}%
\setcounter{enumi}{71}
\item {} 
\sphinxAtStartPar
Let \(f\colon \mathbb{R} \rightarrow \mathbb{R}\) be a continuous function, and let \(a,b\colon \mathbb{R} \rightarrow \mathbb{R}\) be differentiable functions. Prove that

\end{enumerate}
\begin{equation*}
\begin{split}
\frac{d}{dx} \int_{a(x)}^{b(x)} f(t)dt = b'(x)f(b(x))-a'(x)f(a(x)).
\end{split}
\end{equation*}\phantomsection\label{\detokenize{Problems:id73}}\begin{enumerate}
\sphinxsetlistlabels{\arabic}{enumi}{enumii}{}{.}%
\setcounter{enumi}{72}
\item {} 
\sphinxAtStartPar
Define a function \(l:(0,\infty ) \rightarrow \mathbb{R}\) by \(l(x) = \int_1^x \frac{1}{t}dt\).

Show the following directly from the definition of \(l\) via an integral (that is, \sphinxstyleemphasis{without} using any properties of the function \(\ln\)).

\sphinxAtStartPar
(i) \(l\) is differentiable, and \(l'(x) = \frac{1}{x}\).

\sphinxAtStartPar
(ii) \(l(xy) = l(x)+l(y)\) for all \(x,y >0\).

\end{enumerate}
\phantomsection\label{\detokenize{Problems:id74}}
\sphinxAtStartPar
74.\(^*\) Let \(f\colon [a,b]\rightarrow \mathbb{R}\) be Riemann integrable. Prove that there is a number \(x\in [a,b]\) such that
\begin{equation*}
\begin{split}
\int_a^x f(t)dt = \int_x^b f(t)dt.
\end{split}
\end{equation*}
\sphinxAtStartPar
{[}Hint: apply the intermediate value theorem to the function \(F\) given by
\begin{equation*}
\begin{split}
F(x) = \int_a^x f(t)dt - \int_x^b f(t)dt .\qquad ]
\end{split}
\end{equation*}\phantomsection\label{\detokenize{Problems:id75}}\begin{enumerate}
\sphinxsetlistlabels{\arabic}{enumi}{enumii}{}{.}%
\setcounter{enumi}{74}
\item {} 
\sphinxAtStartPar
(i) Let \(f\colon [a,b]\rightarrow \mathbb{R}\) be Riemann integrable. Suppose there are \(m,M\in \mathbb{R}\) such that \(m\leq f(x)\leq M\) for all \(x\in [a,b]\). Prove that there is a number \(\mu \in [m,M]\) such that
\begin{equation*}
\begin{split}
    \int_a^b f(x)\ dx = (b-a)\mu .
    \end{split}
\end{equation*}
\sphinxAtStartPar
(ii) Let \(f\colon [a,b]\rightarrow \mathbb{R}\) be continuous. Prove that there is some \(c \in [a,b]\) such that
\begin{equation*}
\begin{split}
    \int_a^b f(x)\ dx = (b-a)f(c) .
    \end{split}
\end{equation*}
\sphinxAtStartPar
{[}Hint: The intermediate value theorem is useful.{]}

\end{enumerate}
\phantomsection\label{\detokenize{Problems:id76}}\begin{enumerate}
\sphinxsetlistlabels{\arabic}{enumi}{enumii}{}{.}%
\setcounter{enumi}{75}
\item {} 
\sphinxAtStartPar
Let \(f\colon [a,b]\rightarrow \mathbb{R}\) be continuous, and let \(g\colon [a,b]\rightarrow [0,\infty )\) be integrable. Prove that there is some \(c \in [a,b]\) such that

\end{enumerate}
\begin{equation*}
\begin{split}
\int_a^b f(x)g(x)\ dx = f(c) \int_a^b g(x)\ dx .
\end{split}
\end{equation*}
\sphinxAtStartPar
Do we need the assumption \(g(x)\geq 0\)? Justify your answer.

\sphinxstepscope


\section{Solutions}
\label{\detokenize{Solutions-full:solutions}}\label{\detokenize{Solutions-full:sol}}\label{\detokenize{Solutions-full::doc}}

\subsection{Preliminary problems}
\label{\detokenize{Solutions-full:preliminary-problems}}
\sphinxAtStartPar
{\hyperref[\detokenize{Problems:p1}]{\sphinxcrossref{\DUrole{std,std-ref}{P1.}}}}
The correct statements are (ii) and (v).

\sphinxAtStartPar
In statement (iv), the quantifiers \(\forall\) and \(\exists\) are the wrong way round, while in (i) they are missing altogether. Statement (iii) asserts that last part, \(|x_n-l|<\varepsilon\) \(\forall n\geq N\), should hold for all \(\varepsilon\) and for all \(N\) — so again, there is an issue with the quantifiers.

\sphinxAtStartPar
Bonus exercise: prove that (iv) holds if and only if \((x_n)\) is the constant sequence \(x_n=l\) for all \(n\in\mathbb{N}\).


\bigskip\hrule\bigskip


\sphinxAtStartPar
{\hyperref[\detokenize{Problems:p2}]{\sphinxcrossref{\DUrole{std,std-ref}{P2.}}}} (Homework 1 question)

\sphinxAtStartPar
Let \(\varepsilon>0\). According to the definition, we must prove there is an \(N\in\mathbb{N}\) for which \(\left|\frac{2n}{3n-1}-\frac{2}{3}\right|<\varepsilon\) whenever \(n\geq N\).

\sphinxAtStartPar
Now,
\begin{equation*}
\begin{split}
\left|x_n-\frac{2}{3}\right| = \left|\frac{2n}{3n-1}-\frac{2}{3}\right| = \frac{6n-(6n-2)}{3(3n-1)} = \frac{2}{3(3n-1)}.
\end{split}
\end{equation*}
\sphinxAtStartPar
This will be strictly less than \(\varepsilon\) whenever \(3n-1>\frac{2}{3\varepsilon}\)., or in other words, whenever
\begin{equation*}
\begin{split}
n>\frac{1}{3}\left(\frac{2}{3\varepsilon}+1\right)=\frac{2+3\varepsilon}{9\varepsilon}.
\end{split}
\end{equation*}
\sphinxAtStartPar
Let \(N\) be any integer greater than \(\frac{2+3\varepsilon}{9\varepsilon}\). Then \(\left|x_n-\frac{2}{3}\right|<\varepsilon\) for all \(n\geq N\), and we have proven that \(x_n\rightarrow\frac{2}{3}\) as \(n\rightarrow\infty\) using the definition.


\bigskip\hrule\bigskip


\sphinxAtStartPar
{\hyperref[\detokenize{Problems:p3}]{\sphinxcrossref{\DUrole{std,std-ref}{P3.}}}} (Homework 1 question)

\sphinxAtStartPar
The Bolzano–Weierstrass theorem states that every bounded sequence has a convergent subsequence. Its proof combines the following two results:
\begin{itemize}
\item {} 
\sphinxAtStartPar
The monotone convergence theorem (Theorem 3.10 in the MAS107 notes), which states that bounded monotone sequences must converge.  More specifically, every monotone increasing sequence that is bounded above converges to its supremum, and every monotone decreasing sequence bounded below converges to its infimum.

\item {} 
\sphinxAtStartPar
Theorem 3.13 from MAS107: Every sequence has a monotone subsequence.

\end{itemize}

\sphinxAtStartPar
\sphinxstylestrong{Proof of Bolzano–Weierstrass:} 
If \((x_n)\) is a bounded sequence, then it has a monotone subsequence, \((x_{n_k})\), by Theorem 3.13. This subsequence must also be bounded since \((x_n)\) is bounded, and hence it converges by the monotone convergence theorem.


\bigskip\hrule\bigskip


\sphinxAtStartPar
{\hyperref[\detokenize{Problems:p4}]{\sphinxcrossref{\DUrole{std,std-ref}{P4.}}}}(i) Let \(a, b \geq 0\). Then,
\$\(
\left(\sqrt{a+b}\right)^2 = a+b,
\)\$

\sphinxAtStartPar
while,
\begin{equation*}
\begin{split}
\left(\sqrt{a}+\sqrt{b}\right)^2 = a + 2\sqrt{a}\sqrt{b} + b \geq a+b.
\end{split}
\end{equation*}
\sphinxAtStartPar
Therefore
\begin{equation*}
\begin{split}
\left(\sqrt{a}+\sqrt{b}\right)^2 \geq \left(\sqrt{a+b}\right)^2.
\end{split}
\end{equation*}
\sphinxAtStartPar
Since \(\sqrt{a}\), \(\sqrt{b}\) and \(\sqrt{a+b}\) are all non\sphinxhyphen{}negative and the square root function is increasing, we can square root both sides to get
\begin{equation*}
\begin{split}
\sqrt{a}+\sqrt{b}\geq\sqrt{a+b}.
\end{split}
\end{equation*}
\sphinxAtStartPar
(ii)  Note that the inequality we have been asked to prove is symmetric in \(a\) and \(b\), in the sense that exchanging the roles of \(a\) and \(b\) does not affect the value of either side.

\sphinxAtStartPar
This means we can assume without loss of generality that \(\sqrt{|a|}\geq\sqrt{|b|}\). Then,
\begin{equation*}
\begin{split}
\left|\sqrt{|a|} - \sqrt{|b|}\right| = \sqrt{|a|}-\sqrt{|b|},
\end{split}
\end{equation*}
\sphinxAtStartPar
and we need only show that
\begin{equation}\label{equation:Solutions-full:eq:sqrta-sqrtb}
\begin{split}\sqrt{|a|} - \sqrt{|b|} \leq \sqrt{|a - b|}.\end{split}
\end{equation}
\sphinxAtStartPar
We use a trick similar to the proof of Corollary 1.1, and write
\begin{equation*}
\begin{split}
\sqrt{|a|} = \sqrt{\big|(a-b) + b\big|}.
\end{split}
\end{equation*}
\sphinxAtStartPar
By the triangle inequality, we have that \(|a| = \big|(a-b) + b\big| \leq |a-b|+|b|\).

\sphinxAtStartPar
Since the square root function is increasing, we can square root both sides to get
\begin{equation*}
\begin{split}
\sqrt{|a|} \leq \sqrt{|a-b|+|b|}.
\end{split}
\end{equation*}
\sphinxAtStartPar
Applying the result from (i) to the right\sphinxhyphen{}hand side, it follows that
\begin{equation*}
\begin{split}
\sqrt{|a|} \leq \sqrt{|a-b|+|b|} \leq \sqrt{|a-b|}+\sqrt{|b|}.
\end{split}
\end{equation*}
\sphinxAtStartPar
Equation {\hyperref[\detokenize{Solutions-full:equation-eq-sqrta-sqrtb}]{\sphinxcrossref{(2.1)}}} now follows by subtracting \(\sqrt{|b|}\) from both sides.

\sphinxAtStartPar
(iii) Let \((a_n)\) be a real sequence converging to \(l\in\mathbb{R}\). By (ii),
\begin{equation*}
\begin{split}
\left|\sqrt{|a_{n}|}-\sqrt{|l|}\right| \leq \sqrt{\big|a_n-l\big|}
\end{split}
\end{equation*}
\sphinxAtStartPar
for all \(n\in\mathbb{N}\).

\sphinxAtStartPar
Let \(\varepsilon>0\). Since \(a_n\rightarrow l\), there is \(N\in\mathbb{N}\) such that \(|a_n-l|<\varepsilon^2\) whenever \(n\geq N\). Hence for \(n\geq N\),
\begin{equation*}
\begin{split}
\big|\sqrt{|a_{n}|}-\sqrt{|l|}\big| \leq \sqrt{\varepsilon^2} = \varepsilon.
\end{split}
\end{equation*}
\sphinxAtStartPar
Thus \(\displaystyle\lim_{n\rightarrow\infty}\sqrt{|a_n|} = \sqrt{|l|}\).


\bigskip\hrule\bigskip


\sphinxAtStartPar
{\hyperref[\detokenize{Problems:p5}]{\sphinxcrossref{\DUrole{std,std-ref}{P5.}}}} When it exists, the supremum of a set \(A\subseteq\mathbb{R}\) is defined to be the unique number \(\alpha\in\mathbb{R}\) such that
\begin{itemize}
\item {} 
\sphinxAtStartPar
\(\alpha\) is an upper bound for \(A\), and

\item {} 
\sphinxAtStartPar
if \(M\) is any other upper bound for \(A\), then \(\alpha\leq M\).

\end{itemize}

\sphinxAtStartPar
We write \(\sup A\) for the supremum of \(A\), when it exists.

\sphinxAtStartPar
The definition of the infimum is similar: when it exists, the infimum of \(A\) is the unique number \(\beta\in\mathbb{R}\) such that

\sphinxAtStartPar
(i) \(\beta\) is a lower bound for \(A\), and

\sphinxAtStartPar
(ii) if \(L\) is any other lower bound for \(A\), then \(\alpha\geq L\).

\sphinxAtStartPar
We write \(\inf A\) for the infimum of \(A\), when it exists.

\sphinxAtStartPar
The axiom of completeness for the real numbers says that every non\sphinxhyphen{}empty bounded above subset of \(\mathbb{R}\) has a supremum. Equivalently, it says that every non\sphinxhyphen{}empty bounded below subset of \(\mathbb{R}\) has an infimum.

\sphinxAtStartPar
For more details, see page 32 of your Semester 2 MAS107 notes.


\bigskip\hrule\bigskip


\sphinxAtStartPar
{\hyperref[\detokenize{Problems:p6}]{\sphinxcrossref{\DUrole{std,std-ref}{P6.}}}}(i) \(\displaystyle\sup\left\{\frac{m}{n}:m,n\in\mathbb{N} \text{ s.t } m<n\right\}=1\), \(\displaystyle\inf\left\{\frac{m}{n}:m,n\in\mathbb{N} \text{ s.t } m<n\right\}=0\).

\sphinxAtStartPar
(ii) \(\displaystyle\sup\left\{\frac{(-1)^m}{n}:m,n\in\mathbb{N} \text{ s.t } m<n\right\}=\frac{1}{3}\), \(\displaystyle\inf\left\{\frac{(-1)^m}{n}:m,n\in\mathbb{N} \text{ s.t } m<n\right\}=-\frac{1}{2}\).

\sphinxAtStartPar
(iii) \(\displaystyle\sup\left\{\frac{n}{3n + 1} :n\in\mathbb{N}\right\}=\frac{1}{3}\), \(\displaystyle\inf\left\{\frac{n}{3n + 1} :n\in\mathbb{N}\right\}=\frac{1}{4}\).


\bigskip\hrule\bigskip


\sphinxAtStartPar
{\hyperref[\detokenize{Problems:p7}]{\sphinxcrossref{\DUrole{std,std-ref}{P7.}}}}(i) This is false — for a counter\sphinxhyphen{}example, take any singleton set. Then \(\sup A=\inf A=1\).

\sphinxAtStartPar
A correct version of the statement would be \(\inf A \leq \sup A\).

\sphinxAtStartPar
(ii) True. The analogous statement for \(\sup\) was called the characteristic property of the supremum in your MAS107 lecture notes — see Lemma 2.8 on page 36.

\sphinxAtStartPar
(iii) True. Since \(\sup B\) is an upper bound for \(B\), and \(A\) is a subset of \(B\), \(\sup B\) must be an upper bound for \(A\). But \(\sup A\) is a the least upper bound for \(A\), hence \(\sup A\leq \sup B\). The analogous property for \(\inf\) is that \(\inf A\geq\inf B\) whenever \(A\subseteq B\).

\sphinxAtStartPar
(iv) False. For a counter\sphinxhyphen{}example, take \(B=\left\{\frac{1}{n}:n\in\mathbb{N}\right\}\), for which \(\inf B=0\).

\sphinxAtStartPar
(v) True. Let \(s=\max\{\sup A,\sup B\}\). Then \(s\geq \sup A\) and \(s\geq \sup B\), so \(s\) is an upper bound for both \(A\) and \(B\). Therefore, \(s\) is an upper bound for \(A\cup B\). But \(\sup(A\cup B)\) is the least upper bound of \(A\cup B\), and so \(s\geq\sup(A\cup B)\). Also, since \(A\) and \(B\) are both subsets of \(A\cup B\), we have by statement (iii) that \(s=\max\{\sup A,\sup B)\leq\sup(A\cup B)\). Hence \(\max\{\sup A,\sup B)=\sup(A\cup B)\).


\subsection{Limits of functions}
\label{\detokenize{Solutions-full:limits-of-functions}}
\sphinxAtStartPar
{\hyperref[\detokenize{Problems:id1}]{\sphinxcrossref{\DUrole{std,std-ref}{1.}}}} One has to identify the real numbers where the given formula does not make sense, usually because of a zero in a denominator somewhere, and exclude them.

\sphinxAtStartPar
(i) \(A=\mathbb{R} \setminus \{0, -1\}\).

\sphinxAtStartPar
(ii) \(g_{2}(x) = \displaystyle\frac{(x-1)(x+4)}{(x-1)(x+2)(x+3)}\), so \(A=\mathbb{R} \setminus \{1, -2, -3\}\).

\sphinxAtStartPar
(iii) \(g_{3}(x) = \displaystyle\frac{x+4}{(x+2)(x+3)}\), so \(A=\mathbb{R} \setminus \{-2, -3\}\).

\sphinxAtStartPar
(iv) \(A=\mathbb{R} \setminus \{1\}\).

\sphinxAtStartPar
(v) \(A=\mathbb{R} \setminus \{0\}\).


\bigskip\hrule\bigskip


\sphinxAtStartPar
{\hyperref[\detokenize{Problems:id2}]{\sphinxcrossref{\DUrole{std,std-ref}{2.}}}} (Homework 1 question)

\sphinxAtStartPar
(i) \(L=[0,1]\cup[2,3]\).

\sphinxAtStartPar
(ii) \(L=\emptyset\).  (All convergent sequences in \(\mathbb{Z}\) are eventually constant.)

\sphinxAtStartPar
(iii) \(L=\mathbb{R}\).

\sphinxAtStartPar
(iv) \(L=[0,1]\)

\sphinxAtStartPar
(v) \(L=\{0\}\)


\bigskip\hrule\bigskip


\sphinxAtStartPar
{\hyperref[\detokenize{Problems:id3}]{\sphinxcrossref{\DUrole{std,std-ref}{3.}}}} (i) We are given \(f:\mathbb{R}\to\mathbb{R}\); \(f(x)=4x+7\). Intuition tells us that
\begin{equation*}
\begin{split}
\lim_{x\rightarrow 2}f(x) = 4\cdot 2+7 = 15.
\end{split}
\end{equation*}
\sphinxAtStartPar
To prove this, let \(\varepsilon>0\). We seek \(\delta>0\) such that \(|x-2|<\delta\) implies \(|f(x)-15|<\varepsilon\).

\sphinxAtStartPar
Note that
\begin{equation*}
\begin{split}
|f(x)-15| = |4x+7-15| = |4x-8| = 4|x-2|.
\end{split}
\end{equation*}
\sphinxAtStartPar
Therefore, putting \(\delta=\frac{\varepsilon}{4}\), we get that whenever \(|x-2|<\delta\),
\begin{equation*}
\begin{split}
|f(x)-17| = 4|x-2| < 4\cdot\frac{\varepsilon}{4} = \varepsilon.
\end{split}
\end{equation*}
\sphinxAtStartPar
For the limit as \(x\rightarrow 0\), we claim that \(\lim_{x\rightarrow 0}f(x)=7\). To prove this, note that \(|f(x)-7|=4|x|\). This means that given \(\varepsilon>0\), if \(|x|<\frac{\varepsilon}{4}\) then
\begin{equation*}
\begin{split}
|f(x)-7|<4\cdot\frac{\varepsilon}{4} = \varepsilon.
\end{split}
\end{equation*}
\sphinxAtStartPar
It follows that \(\lim_{x\rightarrow 0}f(x)\) exists and is equal to \(7\).

\sphinxAtStartPar
(ii) We have \(f:\{0\}\cup[1,3]\to\mathbb{R}\); \(f(x)=3x^2-1\), so using intuition only, \(\lim_{x\rightarrow 2} f(x)=3\cdot 4-1=11\).

\sphinxAtStartPar
Let \(\varepsilon>0\). We seek \(\delta>0\) such that for all \(x\in\{0\}\cup[1,3]\),
\begin{equation*}
\begin{split}
|x-2|<\delta \; \Rightarrow \; |f(x)-11|<\varepsilon.
\end{split}
\end{equation*}
\sphinxAtStartPar
Now, for \(x\in\{0\}\cup[1,3]\),
\begin{equation*}
\begin{split}
|f(x)-11| = |3x^2-12| = 3|x^2-4| = 3|x-2||x+2| \leq 15|x-2|.
\end{split}
\end{equation*}
\sphinxAtStartPar
(Here, we have used the fact that \(|x+2| = x+2 \leq 5\), for \(x\in\{0\}\cup[1,3]\).)

\sphinxAtStartPar
Hence we can let \(\delta:=\frac{1}{15}\) and conclude that if \(x\in\{0\}\cup[1,3]\) and \(|x-2|<\frac{1}{15}\), then
\begin{equation*}
\begin{split}
|f(x)-11|<15\cdot\frac{\varepsilon}{15} =\varepsilon.
\end{split}
\end{equation*}
\sphinxAtStartPar
For this \(f\), \(\lim_{x\rightarrow 0}f(x)\) is not defined, since \(0\) is not a limit point of the domain of \(f\).

\sphinxAtStartPar
(iii) Finally, let \(f:(0,\infty)\to\mathbb{R}\); \(f(x)=x+\frac{1}{x}\). The limit of \(f(x)\) as \(x\rightarrow 2\) ought to be \(2+\frac{1}{2}=\frac{5}{2}\).

\sphinxAtStartPar
Let \(\varepsilon>0\), and consider
\begin{equation*}
\begin{split}
\left|f(x)-\frac{5}{2}\right| = \left|x+\frac{1}{x}-\frac{5}{2}\right| = \left|\frac{2x^2+2-5x}{2x}\right| = \left(\frac{(x-2)(2x-1)}{2x}\right) = |x-2|\left|1-\frac{1}{2x}\right|.
\end{split}
\end{equation*}
\sphinxAtStartPar
Now, for \(x>0\), \(0<1-\frac{2x}<1\), and so
\begin{equation*}
\begin{split}
\left|f(x)-\frac{5}{2}\right| = |x-2|\left|1-\frac{1}{2x}\right| < |x-2|.
\end{split}
\end{equation*}
\sphinxAtStartPar
Therefore, we can take \(\delta:=\varepsilon\) in this case to get that
\begin{equation*}
\begin{split}
|x-2|<\delta \; \Rightarrow \; \left|f(x)-\frac{5}{2}\right|<\varepsilon.
\end{split}
\end{equation*}
\sphinxAtStartPar
So \(\lim_{x\rightarrow 2}f(x)=\frac{5}{2}\) in this case.

\sphinxAtStartPar
For the last part, note that \(f(x)=x+\frac{1}{x}\) does not converge to a finite limit as \(x\rightarrow 0\). In fact, \(\lim_{x\rightarrow 0}f(x)=\infty\). One way to prove this is to show that \(f(x)\) surpasses any possible bound as \(x\rightarrow 0\). Note that
\begin{equation*}
\begin{split}
f(x) = x+\frac{1}{x} > \frac{1}{x}.
\end{split}
\end{equation*}
\sphinxAtStartPar
Therefore, given an arbitrary number \(K>0\), we can take \(0<x<\frac{1}{K}\) to ensure that
\begin{equation*}
\begin{split}
f(x) = x+\frac{1}{x} > \frac{1}{x} > K.
\end{split}
\end{equation*}
\sphinxAtStartPar
This proves that \(\lim_{x\rightarrow 0} f(x) = \infty\), for this choice of \(f\).


\bigskip\hrule\bigskip


\sphinxAtStartPar
{\hyperref[\detokenize{Problems:id4}]{\sphinxcrossref{\DUrole{std,std-ref}{4.}}}} We had \(A= \mathbb{R} \setminus \{1, -2, -3\}\), and \(f_2:A\to\mathbb{R}\); \(\displaystyle g_{2}(x)= \frac{(x + 4)}{(x + 2)(x + 3)}\).

\sphinxAtStartPar
It is considerably easier to use the sequential criterion for functional limits (Theorem 2.1) than it is to proceed directly using the Definition 2.1. We include both methods, for completeness.

\sphinxAtStartPar
\sphinxstylestrong{Method 1: Sequences:} 
Let \((x_n)\) be a sequence in \(A\) converging to \(1\). Then, using algebra of limits for real sequences,
\begin{equation*}
\begin{split}
\lim_{n\to\infty}f_2(x_n)= \lim_{n\to\infty} \frac{(x_n + 4)}{(x_n + 2)(x_n + 3)} =  \frac{(1 + 4)}{(1 + 2)(1 + 3)} = \frac{5}{12},
\end{split}
\end{equation*}
\sphinxAtStartPar
where the last equality is using algebra of limits. So \(\displaystyle\lim_{x \rightarrow 1}g_{2}(x) = \frac{5}{12}\).

\sphinxAtStartPar
Let \(x_n=-2 + \frac{1}{n}\), so that \((x_n)\) is a sequence in \(A\) converging to \(-2\). Then we see that \((f_2(x_n))\) diverges to \(+\infty\) and so \(\lim_{x \rightarrow -2}g_{2}(x)\)  does not exist.

\sphinxAtStartPar
Similarly, considering \(x_n=-3 + \frac{1}{n}\),  we see that \(\lim_{x \rightarrow -3}g_{2}(x)\) does not exist.

\sphinxAtStartPar
\sphinxstylestrong{Method 2: \((\varepsilon-\delta)\) criterion}: 
Intuitively speaking, the limit as \(x\rightarrow 1\) “should” be \(\frac{(1 + 4)}{(1 + 2)(1 + 3)}=\frac{5}{12}\).

\sphinxAtStartPar
We prove \(\lim_{x\rightarrow 1}f_2(x) = \frac{5}{12}\).

\sphinxAtStartPar
Let \(\varepsilon>0\). We wish to find \(\delta>0\) so that if \(x\in A\) and \(0<|x-1|<\delta\), then \(\left|f_2(x)-\frac{5}{12}\right| <\varepsilon\).

\sphinxAtStartPar
Now,
\begin{align*}
\left|f_2(x)-\frac{5}{12}\right| &= \left|\frac{(x + 4)}{(x + 2)(x + 3)}-\frac{5}{12}\right| \\
&= \left|\frac{12(x+4)-5(x+2)(x+3)}{(x+2)(x+3)}\right| \\
&= \left|\frac{18-5x^2-13x}{(x+2)(x+3)}\right| \\
&= \left|\frac{(1-x)(5x+18)}{(x+2)(x+3)}\right|  = |x-1|\cdot\frac{|5x+18|}{|x+2||x+3|}.
\end{align*}
\sphinxAtStartPar
We are going to constrain \(x\) to be very close to \(1\). In particular, we can assume \(0<x<2\), so
\begin{equation*}
\begin{split}
\left|f_2(x)-\frac{5}{12}\right| =|x-1|\cdot\frac{|5x+18|}{|x+2||x+3|} < |x-1| \frac{5(2)+18}{(2)(3)} = \frac{14}{3}|x-1|.
\end{split}
\end{equation*}
\sphinxAtStartPar
Finally, we choose \(\delta:= \frac{3}{14}\varepsilon\). Then, for \(|x-1|< \delta\),
\begin{equation*}
\begin{split}
\left|f_2(x)-\frac{5}{12}\right| <  \frac{14}{3}|x-1| <  \frac{14}{3}\cdot \frac{3}{14}\varepsilon = \varepsilon.
\end{split}
\end{equation*}

\bigskip\hrule\bigskip


\sphinxAtStartPar
{\hyperref[\detokenize{Problems:id5}]{\sphinxcrossref{\DUrole{std,std-ref}{5.}}}} The first part follows by using the definition of the limit of a function in terms of limits of sequences, and then applying the result of {\hyperref[\detokenize{Problems:p4}]{\sphinxcrossref{\DUrole{std,std-ref}{P4 (iii)}}}}..

To be precise let \((x_{n})\) be any sequence in \(\mathbb{R} \setminus \{a\}\) that converges to \(a\). Then since we are given that \(\lim_{x \rightarrow a} f(x) = l\), we must have that \(\lim_{n\rightarrow\infty} f(x_{n}) = l\). But then by {\hyperref[\detokenize{Problems:p4}]{\sphinxcrossref{\DUrole{std,std-ref}{P4 (iii)}}}}., we have \(\lim_{n\rightarrow\infty} \sqrt{f(x_{n}}) = \sqrt{l}\). So by definition of the limit of a function, \(\lim_{x \rightarrow a} \sqrt{f(x)} = \sqrt{l}\).

Using algebra of limits,  \(\lim_{x \rightarrow 1}\displaystyle\frac{x+1}{x^{2}}=\frac{1+1}{1^2}=2\) and
then by the first part \(\lim_{x \rightarrow 1}\sqrt{\displaystyle\frac{x+1}{x^{2}}} = \sqrt{2}\).


\bigskip\hrule\bigskip


\sphinxAtStartPar
{\hyperref[\detokenize{Problems:id6}]{\sphinxcrossref{\DUrole{std,std-ref}{6.}}}} Let \(f:A\to \mathbb{R}\) and suppose that \(\lim_{x \rightarrow a} f(x) = l\) and \(\lim_{x \rightarrow a} f(x) = l'\). Then given any sequence \((x_{n})\) in \(A \setminus \{a\}\) that converges to \(a\), we have \(\lim_{n\rightarrow\infty} f(x_{n}) = l\) and also \(\lim_{n\rightarrow\infty} f(x_{n}) = l'\). But then \(l = l'\), by uniqueness of limits.


\bigskip\hrule\bigskip


\sphinxAtStartPar
{\hyperref[\detokenize{Problems:id7}]{\sphinxcrossref{\DUrole{std,std-ref}{7.}}}} (Homework 2 question)

When \(x > 0, \displaystyle\frac{|x|}{x} = \displaystyle\frac{x}{|x|} = \displaystyle\frac{x}{x} = 1 = \text{sgn}(x)\),

when \(x < 0, \displaystyle\frac{|x|}{x} = \displaystyle\frac{x}{|x|} = -\displaystyle\frac{x}{x} = -1 = \text{sgn}(x)\).

The left limit is \(\displaystyle\lim_{x \rightarrow 0^-} \text{sgn}(x) = -1\), since for any sequence \((x_n)\) approaching \(0\) from the left, we have \(\text{sgn}(x_n) = -1\) for all \(n\).

Similarly, the right limit is \(\displaystyle\lim_{x \rightarrow 0^+} \text{sgn}(x) = 1\),  since for any sequence \((x_n)\) approaching \(0\) from the right, we have \(\text{sgn}(x_n) = 1\) for all \(n\).

Since the left and right limits are different, \(\displaystyle\lim_{x \rightarrow 0}\text{sgn}(x)\) does not exist.


\bigskip\hrule\bigskip


\sphinxAtStartPar
{\hyperref[\detokenize{Problems:id8}]{\sphinxcrossref{\DUrole{std,std-ref}{8.}}}} (Homework 2 question)

\sphinxAtStartPar
(i) For \(a\neq 1\), the left and right limits exist and are both equal to \(f(a)\), using the algebra of limits. The only point at which left and right limits disagree is \(a = 1\), with
\$\(
\lim_{x \rightarrow 1-}f(x) = 0, \; \text{ and } \; \lim_{x \rightarrow 1+}f(x) = 1.
\)\$

\sphinxAtStartPar
(ii) In this case, left and right limits disagree at every \(n \in \mathbb{Z}\), with \(\lim_{x \rightarrow a-}[x] = n-1, \lim_{x \rightarrow n+}[x] = n\). At every other real number, the left and right limits exist and are both equal to the value of the function there.

\sphinxAtStartPar
(iii) Here, left and right limits disagree at \(x = 0, 1\) and \(2\). We have
\$\(
\lim_{x \rightarrow 0^-}h(x) = 3, \hspace{1em} \lim_{x \rightarrow 1-}h(x) = -2, \hspace{1em}  \lim_{x \rightarrow 2-}h(x) = 10,
\)\(
\)\(
\lim_{x \rightarrow 0^+}h(x) = -2, \hspace{1em} \lim_{x \rightarrow 1+}h(x) = 10, \hspace{1em}  \lim_{x \rightarrow 2+}h(x) = 3.
\)\$


\bigskip\hrule\bigskip


\sphinxAtStartPar
{\hyperref[\detokenize{Problems:id9}]{\sphinxcrossref{\DUrole{std,std-ref}{9.}}}} The largest subset of \(\mathbb{R}\) for which the formula \(f(x)=\sin\left(\frac{1}{x}\right)\) makes sense is \(A = \mathbb{R} \setminus \{0\}\).

Observe that for any real number \(\theta\in\mathbb{R}\),
\begin{equation*}
\begin{split}
f\left(\frac{1}{\theta+2\pi n}\right) = \sin(\theta+2\pi n) = \sin(\theta).
\end{split}
\end{equation*}
\sphinxAtStartPar
To prove \(f\) has no limit at \(0\), we need only choose two values of \(\theta\) for which sine takes distinct values. We choose \(\theta=\frac{\pi}{2}\) and \(\theta=\frac{3\pi}{2}\).

By choosing the right values for \(\theta\), we can use this to construct two distinct sequences \((x_n)\) and \((y_n)\) that both converge to \(0\), but for which \(\sin(x_n)\) and \(\sin(y_n)\) have different limits.

We use \(\theta=\frac{\pi}{2}\) and \(\theta=\frac{3\pi}{2}\) (other choices possible).

For each \(n\in\mathbb{N}\), let \(x_n=\frac{1}{\frac{\pi}{2} + 2\pi n}\) and \(y_n=\frac{1}{3\frac{\pi}{2} + 2\pi n}\). Then
\begin{equation*}
\begin{split}
\lim_{n\rightarrow\infty} x_{n} =\lim_{n\rightarrow\infty} y_n= 0,
\end{split}
\end{equation*}
\sphinxAtStartPar
while for all \(n\in\mathbb{N}\),
\begin{equation*}
\begin{split}
f(x_n) = \sin\left(\frac{\pi}{2}\right) =1 \hspace{2em} \text{ and } \hspace{2em} f(y_n)=\sin\left(\frac{3\pi}{2}\right)=-1.
\end{split}
\end{equation*}
\sphinxAtStartPar
So \(\lim_{n\rightarrow\infty} f(x_{n}) = 1\), but \(\lim_{n\rightarrow\infty} f(y_n) = -1\). Since these limits are not equal, \(\lim_{x\to 0} f(x)\) does not exist.

Of course, we could have chosen any values of \(\theta\) we liked, and constructed a sequence \((x_n)\) with limit \(0\) for which \(\lim_{n\rightarrow\infty}f(x_n)\) is equal to any real number we liked in the interval \([-1,1]\). The graph of this function below may give some intuition as to how this is possible.

\begin{figure}[htbp]
\centering
\capstart

\noindent\sphinxincludegraphics[width=700\sphinxpxdimen]{{sin(1,x)}.png}
\caption{Graph of the function \(f:\mathbb{R}\to\mathbb{R}\); \(f(x)=\sin\left(\frac{1}{x}\right)\).}\label{\detokenize{Solutions-full:s1x}}\end{figure}


\bigskip\hrule\bigskip


\sphinxAtStartPar
{\hyperref[\detokenize{Problems:id10}]{\sphinxcrossref{\DUrole{std,std-ref}{10.}}}} The largest subset of \(\mathbb{R}\) for which the formula \(f(x)=x\sin\left(\frac{1}{x}\right)\) makes sense is \(A = \mathbb{R} \setminus \{0\}\).

We claim that \(\displaystyle\lim_{x \rightarrow 0} x \sin\left(\frac{1}{x}\right) = 0\).

To see this, let \((x_{n})\) be an arbitrary sequence in \(A\) that converges to \(0\). The sine function takes values only in the interval \([-1,1]\), and so
\begin{equation*}
\begin{split}
\left|x_n\sin\left(\frac{1}{x_n}\right)\right| \leq |x_n|
\end{split}
\end{equation*}
\sphinxAtStartPar
for each \(n\in\mathbb{N}\). Put another way,
\begin{equation*}
\begin{split}
-|x_{n}| \leq x_{n}\sin\left(\frac{1}{x_n}\right) \leq |x_{n}|
\end{split}
\end{equation*}
\sphinxAtStartPar
for all \(n\in\mathbb{N}\). Since \(\displaystyle\lim_{n\to \infty} x_n=0\), we have \( \displaystyle\lim_{n\to \infty} |x_n| =0\). Therefore, but the sandwich rule, \(\displaystyle\lim_{n\rightarrow 0}x_n\sin\left(\frac{1}{x_n}\right) = 0\).

Since \((x_n)\) was arbitrary, \(\displaystyle\lim_{x \rightarrow 0} x \sin\left(\frac{1}{x}\right) = 0\).

You should contrast the picture below with that for the previous question.

\begin{figure}[htbp]
\centering
\capstart

\noindent\sphinxincludegraphics[width=700\sphinxpxdimen]{{xsin(1,x)}.png}
\caption{Graph of the function \(f:\mathbb{R}\to\mathbb{R}\); \(f(x)=x\sin\left(\frac{1}{x}\right)\).}\label{\detokenize{Solutions-full:xs1x}}\end{figure}


\bigskip\hrule\bigskip


\sphinxAtStartPar
{\hyperref[\detokenize{Problems:id11}]{\sphinxcrossref{\DUrole{std,std-ref}{11.}}}} We’ll just do \(\displaystyle\lim_{x \rightarrow \infty}f(x)\) here, as \(\displaystyle\lim_{x \rightarrow -\infty}f(x)\) is so similar.

\sphinxAtStartPar
(i) Let \(X\subset\mathbb{R}\) and \(f:X\to\mathbb{R}\). We say that \(\lim_{x \rightarrow \infty}f(x) = l\) if whenever \((x_{n})\) is a sequence that diverges to infinity, with \(x_{n}\in X\) for all \(n\in\mathbb{N}\), then \(\lim_{x \rightarrow \infty}f(x_{n}) = l\).

\sphinxAtStartPar
(ii) The analogue of the \((\varepsilon- \delta)\) criterion is \((\varepsilon-K)\): given any \(\varepsilon > 0\), there exists \(K > 0\) such that if \(x > K\) then \(|f(x) -l| < \varepsilon\).

For the proof, suppose the \((\varepsilon- K)\) criterion holds and \(\lim_{n\rightarrow\infty} x_{n} = \infty\). Then given any \(L > 0\), there exists \(N\in\mathbb{N}\) such that if \(n\geq N\), we have \(x_{n} > L\). Now take \(L\) to be \(K\) from the criterion and we get that for all \(n\geq N, |f(x_{n}) - l| < \varepsilon\). Hence \(\lim_{x \rightarrow \infty}f(x_{n}) = l\), as required.

For the converse, we again imitate the proof of Theorem 2.1. So suppose the \((\varepsilon-K)\) criterion does not hold. This time choose successively \(K = 1, 2, \ldots\) and construct \(x_{n}\) in the domain \(X\) of \(f\) such that \(x_{n} > n\), and \(|f(x_{n}) - l| \geq \varepsilon\) for each \(n\in\mathbb{N}\). So \(f\) does not converge to \(l\) as \(x\to\infty\).

\sphinxAtStartPar
(iii) Given any \(\varepsilon > 0\), choose \(K = \frac{1}{\varepsilon}\). Then \(x > K \Rightarrow \frac{1}{x} < \varepsilon\), and so \(\lim_{x\to\infty} \frac{1}{x}=0\). The case where \(x\to-\infty\) is similar.


\bigskip\hrule\bigskip


\sphinxAtStartPar
{\hyperref[\detokenize{Problems:id12}]{\sphinxcrossref{\DUrole{std,std-ref}{12.}}}}
(i) Let \(f:X\to\mathbb{R}\) be a function, where \(X\subseteq\mathbb{R}\). We say that \(\lim_{x \rightarrow \infty} f(x) = \infty\) if for any sequence \((x_{n})\) in \(X\) which diverges to infinity, we also have that \((f(x_{n}))\) diverges to infinity.

The analogue of \((\varepsilon - \delta)\) is:
\begin{quote}

\sphinxAtStartPar
Given any \(M > 0\), there exists \(K > 0\) such that \(x > K\) implies \(f(x) > M\).
The other cases are similar.
\end{quote}

\sphinxAtStartPar
(ii) Let \(f:\mathbb{R}\to[0,\infty)\), \(g:\mathbb{R}\to[0,\infty)\), and suppose \(\displaystyle\lim_{x\rightarrow\infty}f(x)=\infty\) and \(\displaystyle\lim_{x\rightarrow\infty}g(x)=l\), where \(l>0\).

Let \(M > 0\). Since \(\lim_{x\rightarrow\infty}f(x)=\infty\), there exists \(K_1 > 0\) such that
\$\(
f(x)>\frac{2M}{l} \hspace{2em} \forall x > K_1.
\)\(
Since \)\textbackslash{}lim\_\{x\textbackslash{}rightarrow\textbackslash{}infty\}g(x)=l\(, for all \)\textbackslash{}varepsilon>0\( we can find \)K\_2>0\( such that \)x>K\_2\( implies \)|g(x)\sphinxhyphen{}l|<\textbackslash{}varepsilon\(. In other words,
\)\(
l-\varepsilon < g(x) < l+\varepsilon, \hspace{2em} \forall x>K_2.
\)\(
This is true for all \)\textbackslash{}varepsilon\(. We apply it specifically to \)\textbackslash{}varepsilon = \textbackslash{}frac\{l\}\{2\}\(. Then, there is \)K\_2>0\( such that
\)\(
\frac{l}{2} < g(x) < \frac{3l}{2}, \hspace{2em} \forall x>K_2.
\)\(
Let \)K=\textbackslash{}max\{K\_1,K\_2\}\(. If \)x>K\(, then \)x>K\_1\( and \)x>K\_2\(, so \)f(x)>\textbackslash{}frac\{2M\}\{l\}\( and \)\textbackslash{}frac\{l\}\{2\} < g(x) < \textbackslash{}frac\{3l\}\{2\}\(.
<br>
Hence for \)x>K\(,
\)\(
f(x)g(x) > \frac{2M}{l}\frac{l}{2} = M.
\)\(
That is, \)\textbackslash{}lim\_\{x\textbackslash{}rightarrow\textbackslash{}infty\}f(x)g(x) = \textbackslash{}infty\$.

\sphinxAtStartPar
(iii) Let \(p:\mathbb{R}\to\mathbb{R}\) be a polynomial of even degree \(m\), with positive leading coefficient. Write \(m = 2n\) and let
\begin{align*}
p(x) &= a_{2n}x^{2n} + a_{2n-1}x^{2n-1} + \cdots + a_{1}x + a_{0}\\
&= x^{2n}\left(a_{2n} + \frac{a_{2n-1}}{x} + \cdots + \frac{a_{1}}{x^{2n-1}} + \frac{a_{0}}{x^{2n}}\right). 
\end{align*}
\sphinxAtStartPar
We use part (ii). Take \(f(x) = x^{2n}\) and \(\displaystyle g(x) = a_{2n} + \frac{a_{2n-1}}{x} + \cdots + \frac{a_{1}}{x^{2n-1}} + \frac{a_{0}}{x^{2n}}\).

Observe that \(\displaystyle\lim_{x \rightarrow \infty}f(x) = \infty\) and \(\displaystyle\lim_{x \rightarrow \infty}g(x) = a_{2n}\). Hence by part (ii),
\begin{equation*}
\begin{split}
\lim_{x \rightarrow \infty}p(x) = \lim_{x \rightarrow \infty}f(x)g(x) = \infty.
\end{split}
\end{equation*}
\sphinxAtStartPar
If \(n\) is odd, \(\lim_{x \rightarrow \infty}p(x) = \infty\), but \(\lim_{x \rightarrow -\infty}p(x) = -\infty\).


\subsection{Continuity}
\label{\detokenize{Solutions-full:continuity}}
\sphinxAtStartPar
{\hyperref[\detokenize{Problems:id13}]{\sphinxcrossref{\DUrole{std,std-ref}{13.}}}} For each of (i) to (v), the function is continuous at each point of its domain. For (i), (ii) and (iii), we have rational functions, which are continuous on all points of \(A\) (which is the subset of \(\mathbb{R}\) where the denominator is non\sphinxhyphen{}zero). For (iv) and (v), we have compositions of rational functions with the exponential and cosine functions, which are continuous on the whole of \(\mathbb{R}\). Again, the functions are continuous on all points of \(A\) (which is the subset of \(\mathbb{R}\) where the denominator of the rational function is non\sphinxhyphen{}zero).


\bigskip\hrule\bigskip


\sphinxAtStartPar
{\hyperref[\detokenize{Problems:id14}]{\sphinxcrossref{\DUrole{std,std-ref}{14.}}}} If \((x_{n})\) is any sequence that converges to \(a\), we know that \(f(x_n)\) converges to \(f(a)\) as \(f\) is continuous at \(a\), and
we need to show that \((|f|(x_n))\) converges to \(|f|(a)\).

By Corollary 1.1, if \((x_{n})\) is any sequence that converges to \(a\),
\begin{align*}
0 \leq ||f|(a)| -|f|(x_{n})| &= ||f(a) - |f(x_{n})|| \\
&\leq |f(a) - f(x_{n})| \rightarrow 0~\mbox{as}~n \rightarrow \infty, 
\end{align*}
\sphinxAtStartPar
as \(f\) is continuous. So by the sandwich rule, \(\lim_{n\rightarrow\infty} |f|(x_{n}) = |f|(a)\), and so \(f\) is continuous at \(a\).


\bigskip\hrule\bigskip


\sphinxAtStartPar
{\hyperref[\detokenize{Problems:id15}]{\sphinxcrossref{\DUrole{std,std-ref}{15.}}}} Let \((x_{n})\) be a sequence in \(A\) that converges to \(a\). Since \(f\) is continuous at \(a\) we have \(\lim_{n\rightarrow\infty} f(x_{n}) = f(a)\). And then, since \(g\) is continuous at \(f(a)\), we have \(\lim_{n\rightarrow\infty} g(f(x_{n})) = g(f(a))\), and the result follows.


\bigskip\hrule\bigskip


\sphinxAtStartPar
{\hyperref[\detokenize{Problems:id16}]{\sphinxcrossref{\DUrole{std,std-ref}{16.}}}} \(f \circ g:\mathbb{R}\to \mathbb{R}\) given by \((f \circ g)(x) = \frac{1}{1  + x^{2}}\). It is continuous on \(\mathbb{R}\) by Theorem 3.2(iv).

\(g \circ f:\mathbb{R} \setminus \{0\} \to \mathbb{R} \setminus \{0\}\) given by \((g \circ f)(x) = {1  + \frac{1}{x^2}}\). It is continuous on \(\mathbb{R} \setminus \{0\}\) by Theorem 3.2(iv).


\bigskip\hrule\bigskip


\sphinxAtStartPar
{\hyperref[\detokenize{Problems:id17}]{\sphinxcrossref{\DUrole{std,std-ref}{17.}}}}
(i) Continuous on \(\mathbb{R} \setminus \{1\}\). Jump discontinuity at \(1\) with \(J_{f}(1) = 1\).
(ii) Continuous on \(\mathbb{R} \setminus \mathbb{Z}\). Jump discontinuity at \(n\) with \(J_{g}(n) = 1\) for all \(n\in\mathbb{Z}_+\).
(iii) Continuous at \(\mathbb{R} \setminus \{0,1,2\}\). Each of \(0, 1, 2\) is a jump discontinuity and we have \(J_{h}(0) =-5, J_{h}(1) = 12, J_{h}(2) = -7.\)


\bigskip\hrule\bigskip


\sphinxAtStartPar
{\hyperref[\detokenize{Problems:id18}]{\sphinxcrossref{\DUrole{std,std-ref}{18.}}}} HW2 question

\sphinxAtStartPar
(i) For \(x \neq 0, f(x) = x+2\). Since \(\lim_{x \rightarrow 0}f(x) = 2\), the required continuous extension is \(\tilde{f}\) where
\begin{equation*}
\begin{split}
\tilde{f}(x) = \left\{\begin{array}{c c} \displaystyle\frac{(1 + x)^{2} - 1}{x} & ~\mbox{if}~x \neq 0\\ 
& \\
2 & ~\mbox{if}~x = 0. \end{array} \right.
\end{split}
\end{equation*}
\sphinxAtStartPar
(ii) The largest domain is \(A=\mathbb{R} \setminus \{-2, 2\}\) (the set where the denominator is non\sphinxhyphen{}zero). For \(x \neq 2, -2\),
\begin{equation*}
\begin{split}
f(x) = \frac{(x-2)(x^{2} + 2x + 4)}{(x-2)(x+2)}=\frac{x^{2} + 2x + 4}{x + 2}.
\end{split}
\end{equation*}
\sphinxAtStartPar
This is a rational function, and hence continuous everywhere the denominator is non\sphinxhyphen{}zero by Theorem 3.2(iv).

We claim that \(\lim_{x \rightarrow -2}f(x)\) does not exist. To see this, consider the sequence \((x_{n})\), whose \(n\)\textbackslash{}textsuperscript\{th\} term is \(-2 + \frac{1}{n}\), and check that
\begin{equation*}
\begin{split}
f(x_{n}) = n\left(-4 -\frac{2}{n} + \frac{1}{n^2}\right) \rightarrow -\infty~\mbox{when}~n \rightarrow \infty.
\end{split}
\end{equation*}
\sphinxAtStartPar
Thus \(f\) has no continuous extension to the point \(x =-2\).

But on the other hand,
\begin{equation*}
\begin{split}
\lim_{x \rightarrow 2}f(x) = \lim_{x\rightarrow 2}  \frac{x^{2} + 2x + 4}{x + 2} = \frac{12}{4} = 3,
\end{split}
\end{equation*}
\sphinxAtStartPar
and so \(f\) has a continuous extension
\begin{equation*}
\begin{split}
\tilde{f}:\mathbb{R} \setminus \{-2\}\to\mathbb{R}; \hspace{1em} \tilde{f}(x) = \frac{x^2+2x+4}{x+2}.
\end{split}
\end{equation*}
\sphinxAtStartPar
We can see this behaviour in the graph of the function:

\begin{figure}[htbp]
\centering
\capstart

\noindent\sphinxincludegraphics[width=500\sphinxpxdimen]{{(x2+2x+4),(x+2)}.png}
\caption{Graph of the function \(\tilde{f}:\mathbb{R}\setminus\{-2\}\to\mathbb{R}\); \(f(x)=\frac{x^2+2x+4}{x+2}\).}\label{\detokenize{Solutions-full:q18}}\end{figure}


\bigskip\hrule\bigskip


\sphinxAtStartPar
{\hyperref[\detokenize{Problems:id19}]{\sphinxcrossref{\DUrole{std,std-ref}{19.}}}} Assume \(g\) is continuous and \(g(a) > 0\). Suppose, for a contradiction, that  there is no  \(\delta > 0\) such that \(g(x) > 0\)  for all \( x \in (a - \delta, a + \delta)\). Then, in particular, for all \(n\in\mathbb{N}\) there exists \(x_{n} \in \left(a - \frac{1}{n}, a + \frac{1}{n}\right)\) such that \(g(x_{n}) \leq 0\). By the sandwich rule, we have \(\lim_{n\rightarrow\infty} x_{n} = a\). So, by continuity of \(g\) at \(a\), we have \(\lim_{n\rightarrow\infty} g(x_{n})\) exists and equals \(g(a)\). But \(g(x_n)\leq 0\) for all \(n\in\mathbb{N}\), and so
\begin{equation*}
\begin{split}
g(a)=\lim_{n\rightarrow\infty} g(x_{n}) \leq 0,
\end{split}
\end{equation*}
\sphinxAtStartPar
which is a contradiction.


\bigskip\hrule\bigskip


\sphinxAtStartPar
{\hyperref[\detokenize{Problems:id20}]{\sphinxcrossref{\DUrole{std,std-ref}{20.}}}}
(i)
\$\(
\max\{a, b\} = \left\{\begin{array}{c c} a & ~\mbox{if}~a \geq b\\
b & ~\mbox{if}~a < b.\\ \end{array} \right.
\)\(
On the other hand,
\)\(
\frac{1}{2}( a+b) + \frac{1}{2}|a-b| = \left\{\begin{array}{c c c} \frac{1}{2}(a+b) + \frac{1}{2}(a- b) & = a & ~\mbox{if}~a \geq b\\[.5em]
\frac{1}{2}(a+b) + \frac{1}{2}(b- a) &=  b & ~\mbox{if}~a < b,\\ \end{array} \right.
\)\$
and the result follows.

\sphinxAtStartPar
Then for all \(x \in A\cap B,\)
\$\(
\max\{f, g\}(x) = \frac{1}{2}(f(x) + g(x)) + \frac{1}{2}|f(x) - g(x)|,
\)\$
and continuity follows by Theorem  3.2(i) and (iii) and {\hyperref[\detokenize{Problems:id14}]{\sphinxcrossref{\DUrole{std,std-ref}{Problem 14}}}}.

\sphinxAtStartPar
(ii) Check that for all \(a, b \in \mathbb{R}\),
\begin{equation*}
\begin{split}
\min\{a, b\} = \frac{1}{2}(a + b) - \frac{1}{2}|a - b|,
\end{split}
\end{equation*}
\sphinxAtStartPar
and then argue as in (i).

\sphinxAtStartPar
Alternatively, one could derive (ii) from (i) by using \(\min\{f,g\} = - \max\{-f, -g\}\).


\bigskip\hrule\bigskip


\sphinxAtStartPar
{\hyperref[\detokenize{Problems:id21}]{\sphinxcrossref{\DUrole{std,std-ref}{21.}}}}
(i) \(f(0) = f(0 + 0) = f(0) + f(0) = 2f(0)\), hence \(f(0) = 0\).

\sphinxAtStartPar
(ii) By (i), \(0 = f(0) = f(x + -x) = f(x) + f(-x)\). So \(f(-x)=-f(x)\).

\sphinxAtStartPar
(iii) If \(a \neq 0\) then any sequence \((x_{n})\) which converges to \(a\) can be written as \(x_{n} = a + y_{n}\) where \((y_{n})\) converges to zero. If \(f\) is continuous at \(0\), then \( \lim_{n\rightarrow\infty} f(y_{n})\) exists and equals \(f(0)\), which is \(0\) by part (i). So
\$\(
\lim_{n\rightarrow\infty} f(x_{n}) = \lim_{n\rightarrow\infty} f(a + y_{n}) = f(a) + \lim_{n\rightarrow\infty} f(y_{n}) = f(a) + f(0) = f(a).
\)\$

\sphinxAtStartPar
(iv) Firstly, assume \(n\in\mathbb{N}\) and use induction. It is true for \(n = 1\). Assume it holds for some \(n\in\mathbb{N}\), then
\begin{equation*}
\begin{split}
f(n+1) = f(n) + f(1) = nk + k = (n+1)k.
\end{split}
\end{equation*}
\sphinxAtStartPar
So by induction, the result holds for all \(n\in\mathbb{N}\).
Combine this with part (ii), to extend to all \(n\in \mathbb{Z}\).

\sphinxAtStartPar
(v) Consider \(\frac{p}{q}\in\mathbb{Q}\), where \(p\in \mathbb{Z}\) and \(q\in \mathbb{N}\). By (iv),
\$\(
pk = f(p) = f\left(q\cdot \frac{p}{q}\right) = qf\left(\frac{p}{q}\right),
\)\$ and the result follows.

\sphinxAtStartPar
(vi) We have already proved this for rational \(x\), so suppose that \(x\) is irrational. Then we can find a sequence \(\left(\frac{p_{n}}{q_{n}}\right)\) of rational numbers that converges to \(x\) (as we did in the solution to Example 3.7). Then using (iii), (v) and algebra of limits, we have
\begin{equation*}
\begin{split}
f(x) = \lim_{n\rightarrow\infty} f\left(\frac{p_n}{q_n}\right) =   \lim_{n\rightarrow\infty} k \frac{p_n}{q_n}=k \lim_{n\rightarrow\infty} \frac{p_n}{q_n} = kx.
\end{split}
\end{equation*}

\bigskip\hrule\bigskip


\sphinxAtStartPar
{\hyperref[\detokenize{Problems:id22}]{\sphinxcrossref{\DUrole{std,std-ref}{22.}}}} Suppose that \(x \in \mathbb{Q}\). Then as in the solution to Example 3.7 we can find a sequence \((x_{n})\) of irrationals that converges to \(x\) and then
\begin{equation*}
\begin{split}
\lim_{n\rightarrow\infty} g(x_{n}) = 0 \neq g(x),
\end{split}
\end{equation*}
\sphinxAtStartPar
and so \(g\) is not continuous at \(x\).


\bigskip\hrule\bigskip


\sphinxAtStartPar
{\hyperref[\detokenize{Problems:id23}]{\sphinxcrossref{\DUrole{std,std-ref}{23.}}}} HW2 question

The function \({\bf 1}_{(a, b)}\) is left continuous at \(a\) (but not right continuous), and right continuous at \(b\) (but not left continuous).
To prove the left continuity at \(a\), let \((x_{n})\) be any sequence in \(\mathbb{R}\) which converges to \(a\) with \(x_{n} < a\) for all \(n\in\mathbb{N}\). Then \(\lim_{n\rightarrow\infty} {\bf 1}_{(a, b)}(x_{n}) = 0 = {\bf 1}_{(a, b)}(a).\) On the other hand to see that it is not right continuous at \(a\), let \((y_{n})\) be any sequence in \(\mathbb{R}\) which converges to \(a\) with \(a < y_{n} < b\) for all \(n\in\mathbb{N}\). Then
\( \lim_{n\rightarrow\infty} {\bf 1}_{(a, b)}(y_{n}) = 1 \neq {\bf 1}_{(a, b)}(a). \) The other assertion is proved similarly.

{[}Contrast this with \({\bf 1}_{[a, b]}\), which was discussed in the notes.{]}


\bigskip\hrule\bigskip


\sphinxAtStartPar
{\hyperref[\detokenize{Problems:id24}]{\sphinxcrossref{\DUrole{std,std-ref}{24.}}}} Since \(f\) is continuous on \([a, b]\), so is \(g\). We have \(g(a) = f(a) - \gamma < 0\) and \(g(b) = f(b) - \gamma > 0\). Hence by the intermediate value theorem (Theorem 3.4), there exists \(c \in (a, b)\) with \(g(c) = 0\), i.e. \(f(c) = \gamma\), as was required.


\bigskip\hrule\bigskip


\sphinxAtStartPar
{\hyperref[\detokenize{Problems:id25}]{\sphinxcrossref{\DUrole{std,std-ref}{25.}}}}
(i) If \(f\) is continuous on \(\mathbb{R}\) it is continuous on \([a, b]\) for each \(a < b\). If \(f\) is not a constant, we must be able to find \(a, b\) such that \(f(a) \neq f(b)\). Now either \(f(a) < f(b)\) or \(f(a) > f(b)\). Assume the former (without loss of generality). Then there exists \(m, n \in \mathbb{Z}\) with \(m < n\) such that \(f(a) = m\) and \(f(b) = n\). Hence by Corollary 3.1, there exists \(c \in (a, b)\) so that \(f(c) = m + \frac{1}{2} \notin\mathbb{Z}\), and that is the desired contradiction.
(ii) Argue as in (i), using the fact that between any two rational numbers, we can find an irrational number.


\bigskip\hrule\bigskip


\sphinxAtStartPar
{\hyperref[\detokenize{Problems:id26}]{\sphinxcrossref{\DUrole{std,std-ref}{26.}}}} (Homework 3 question)

\sphinxstyleemphasis{Something} is \(x\). That is, define \(g:[a,b]\to\mathbb{R}\) given by \(g(x) = f(x) - x\). Then \(g\) is continuous (because \(f\) is and the function \(h(x)=x\) is, and using algebra of limits). Since the range of \(f\) is contained in \((a, b)\), we have \(f(a) > a\) and \(f(b) < b\), and so \(g(a) = f(a) - a > 0\) and \(g(b) = f(b) - b < 0\). So we can apply the intermediate value theorem to the function \(g\) on \([a,b]\). This says that there exists \(c \in (a, b)\) such that \(g(c) = 0\), i.e. \(f(c) = c\).

For the counter–example, consider \(f(x) = x^2\). It is continuous on \((0, 1)\) but there is no \(c \in (0, 1)\) for which \(c^2 = c\).


\bigskip\hrule\bigskip


\sphinxAtStartPar
{\hyperref[\detokenize{Problems:id27}]{\sphinxcrossref{\DUrole{std,std-ref}{27.}}}} Define \(\gamma = \inf_{x \in [a, b]}f(x)\) and assume that it is not attained, so \(\gamma < f(x)\) for all \(x \in [a,b]\). Then consider the function \(h:[a,b]\to \mathbb{R}\) given by \(h(x) = \displaystyle\frac{1}{f(x) - \gamma}\). This is continuous, and hence bounded on \([a, b]\). So there exists \(K \geq 0\) such that \(|h(x)| \leq K\) for all \(x \in [a, b]\). By Problem 16(ii), given any \(\varepsilon > 0\), there exists \(x \in [a, b]\) such that \(f(x) < \gamma + \varepsilon\). Now take \(\varepsilon = \frac{1}{K}\) to deduce that \(h(x) > K\), which yields the required contradiction.


\bigskip\hrule\bigskip


\sphinxAtStartPar
{\hyperref[\detokenize{Problems:id28}]{\sphinxcrossref{\DUrole{std,std-ref}{28.}}}} By algebra of limits, \(\frac{1}{f}\) is continuous on \([0, 1]\) and so is bounded by Theorem 3.5.


\bigskip\hrule\bigskip


\sphinxAtStartPar
{\hyperref[\detokenize{Problems:id29}]{\sphinxcrossref{\DUrole{std,std-ref}{29.}}}} (Homework 3 question)

If \(f\) is continuous on \([0, 1]\), then it is bounded by Theorem 3.5, and so there exists \(L \geq 0\) such that \(|f(x)| \leq L\) for all \(x \in [0, 1]\). Hence the range of \(f\) is a subset of  \([-L, L]\) and cannot be all of \(\mathbb{R}\).


\bigskip\hrule\bigskip


\sphinxAtStartPar
{\hyperref[\detokenize{Problems:id30}]{\sphinxcrossref{\DUrole{std,std-ref}{30.}}}} Since \((x_{n})\) is bounded, by the Bolzano–Weierstrass theorem it has a convergent subsequence \((x_{n_{k}})\). Let \(c=\lim_{k \rightarrow \infty}x_{n_{k}}\) and note that \(c \in [0, 1]\), since \(x_{n_k}\in[0,1]\) for all \(k\).

A straight\sphinxhyphen{}forward induction argument shows that \(f(x_{n_{k}}) \leq r^{n_{k}-1}f(x_{n_{1}})\) for all \(k\in\mathbb{N}\). But then, since the range of \(f\) is contained in \([0,1]\), we have
\begin{equation*}
\begin{split}
0 \leq f(x_{n_{k}}) \leq r^{n_{k}-1}
\end{split}
\end{equation*}
\sphinxAtStartPar
for all \(k\in\mathbb{N}\). Since \(r < 1\), \(\lim_{k\to\infty} r^{n_k-1} =0\). By the sandwich rule, \( \lim_{k \rightarrow \infty}f(x_{n_{k}}) = 0\) and by continuity of \(f\) at \(c\), \(f(c) = \lim_{k \rightarrow \infty}f(x_{n_{k}})\), so \(f(c)=0\).


\bigskip\hrule\bigskip


\sphinxAtStartPar
{\hyperref[\detokenize{Problems:id31}]{\sphinxcrossref{\DUrole{std,std-ref}{31.}}}} For all \(x > y\),
\begin{equation*}
\begin{split}
x^{n} - y^{n} = (x - y)(x^{n-1} + x^{n-2}y + \cdots + xy^{n-1} + y^{n-1}) > 0.
\end{split}
\end{equation*}

\bigskip\hrule\bigskip


\sphinxAtStartPar
{\hyperref[\detokenize{Problems:id32}]{\sphinxcrossref{\DUrole{std,std-ref}{32.}}}} For all \(-\frac{\pi}{2} \leq x < y \leq \frac{\pi}{2}\),
\begin{equation*}
\begin{split}
\sin(y) - \sin(x) = 2\sin\left(\frac{y - x}{2}\right)\cos\left(\frac{y + x}{2}\right) > 0,
\end{split}
\end{equation*}
\sphinxAtStartPar
so the function is strictly monotonic increasing on this interval. It is also continuous (stated in notes), and so by Theorem 3.6, it is has a continuous inverse \(f^{-1}(x) = \arcsin(x)\) (or \(\sin^{-1}(x)\)) defined on \([-1, 1]\). Outside the interval \(\left[-\frac{\pi}{2}, \frac{\pi}{2}\right]\), the function \(f\) might fail to be strictly increasing. In fact it is strictly decreasing on each interval of the form \(\left((4n+1)\frac{\pi}{2}, (4n + 3)\frac{\pi}{2}\right)\), and strictly increasing on each interval of the form \(\left((4n-1)\frac{\pi}{2}, (4n + 1)\frac{\pi}{2}\right)\), where \(n\in\mathbb{Z}_+\).


\bigskip\hrule\bigskip


\sphinxAtStartPar
{\hyperref[\detokenize{Problems:id33}]{\sphinxcrossref{\DUrole{std,std-ref}{33.}}}} We seek a continuous extension of the mapping \(x\mapsto \frac{1-x}{1-x^{\frac{m}{n}}}\) to a domain that includes \(1\). Then, we can use continuity to evaluate the limit as \(x\rightarrow 1\).

Observe that, if we write \(y = x^{\frac{1}{n}}\), then
\begin{equation*}
\begin{split}
\frac{1 - x}{1 - x^{\frac{m}{n}}} = \frac{1 - y^{n}}{1 - y^{m}} = \frac{\;\frac{1}{1-y^m}\;}{\;\frac{1}{1-y^n}\;} = \frac{1+y+y^2+\ldots+y^{m-1}}{1+y+y^2+\ldots+y^{n-1}},
\end{split}
\end{equation*}
\sphinxAtStartPar
by the geometric sum formula. Note that the left hand side is not defined at \(x=1\), but the right hand side is.

Let \(f:[0,\infty)\to[0,\infty)\); \(f(x)=x^{\frac{1}{n}}\), and let
\begin{equation*}
\begin{split}
g:\mathbb{R}\to\mathbb{R};\hspace{1em} g(y)=\frac{1+y+y^2+\ldots+y^m}{1+y+y^2+\ldots+y^m}.
\end{split}
\end{equation*}
\sphinxAtStartPar
Both \(f\) and \(g\) are continuous at every point in their domain (for \(f\), this was proven in Example 3.9). Hence by the composition theorem for continuous functions (Theorem 3.3), the map
\begin{equation*}
\begin{split}
g\circ f:[0,\infty)\to\mathbb{R}; \hspace{1em} g\circ f(x) = \frac{1+x^{\frac{1}{n}}+x^{\frac{2}{n}}+\ldots+x^{\frac{m-1}{n}}}{1+x^{\frac{1}{n}}+x^{\frac{2}{n}}+\ldots+x^{\frac{n-1}{n}}}
\end{split}
\end{equation*}
\sphinxAtStartPar
is continuous. Also, by our initial calculation,
\begin{equation*}
\begin{split}
g\circ f(x) = \frac{1 - x}{1 - x^{\frac{m}{n}}}
\end{split}
\end{equation*}
\sphinxAtStartPar
whenever \(x\geq 0\) and \(x^{\frac{m}{n}}\neq 1\). Therefore,
\begin{equation*}
\begin{split}
\lim_{x\rightarrow 1}\frac{1 - x}{1 - x^{\frac{m}{n}}} = \lim_{x\rightarrow 1}g\circ f(x) = g\circ f(1) = \frac{1+1+\ldots+1}{1+1+\ldots+1} = \frac{m}{n},
\end{split}
\end{equation*}
\sphinxAtStartPar
by continuity of \(g\circ f\).


\bigskip\hrule\bigskip


\sphinxAtStartPar
{\hyperref[\detokenize{Problems:id34}]{\sphinxcrossref{\DUrole{std,std-ref}{34.}}}} We claim that if \(a < x < b\), we have \(f(a) < f(x) < f(b)\). To see this, note that if \(f(a) \geq f(x)\) then either \(f(a) = f(x)\), or \(f(a) > f(x)\).

If \(f(a)=f(x)\), then \(f\) cannot be bijective, which is a contradiction.

Suppose that \(f(a)>f(x)\). Applying Corollary 3.3 to \(f|_{[x,b]}\), the image of \([x,b]\) under \(f\) must an interval, \([m,M]\), say. Note that \(f(a)\in(m,M)\): indeed, \(f(a)>f(x)>M\), and \(f(a)<f(b)<m\). Therefore, by the intermediate value theorem (or Corollary 3.1 more specifically), there exists \(c \in (x, b)\) such that \(f(c) = f(a)\), and this again violates the injectivity of the mapping \(f\). A similar argument can be used to show that we cannot have \(f(b) \leq f(x)\).

Finally, applying our initial result to \(f|_{[a, y]}\), we see that if \(a<x<y<b\), then \(f(a) < f(x) < f(y)\). Hence \(f\) is strictly monotonic increasing.


\bigskip\hrule\bigskip


\sphinxAtStartPar
{\hyperref[\detokenize{Problems:id35}]{\sphinxcrossref{\DUrole{std,std-ref}{35.}}}} First observe that \(f+g\) is monotonic increasing since both \(f\) and \(g\) are. Choose \(a \in \mathbb{R}\). Given \(\varepsilon > 0\), there exists \(\delta > 0\) so that if \(x > a + \delta\) then
\begin{equation*}
\begin{split}
|f(x) + g(x) - f(a) - g(a)|  =  f(x) + g(x) - f(a) - g(a) < \varepsilon,
\end{split}
\end{equation*}
\sphinxAtStartPar
and so
\begin{equation*}
\begin{split}
|f(x) - f(a)| = f(x) - f(a) < \varepsilon +g(a) - g(x) < \varepsilon,
\end{split}
\end{equation*}
\sphinxAtStartPar
as \(g\) is increasing. This proves that \(g\) is right–continuous at \(a\). A similar argument (interchanging the roles of \(a\) and \(x\)) proves that it is left–continuous, and hence continuous at \(a\). Then \(g = (f + g) - f\) is the difference of two continuous functions, and hence is itself continuous.


\bigskip\hrule\bigskip


\sphinxAtStartPar
{\hyperref[\detokenize{Problems:id36}]{\sphinxcrossref{\DUrole{std,std-ref}{36.}}}}
(i) If \(0<a<1\), then \(0<a<\sqrt{a}<1\). So \((x_n)\) is monotonic increasing and bounded above by \(1\), and hence converges to some limit \(0\leq l\leq 1\). By continuity of the square root function, \(\sqrt{l}=\lim_{n\rightarrow\infty}\sqrt{x_n}\). But \(\sqrt{x_n}=x_{n+1}\rightarrow l\) as \(n\rightarrow\infty\). By uniqueness of limits, \(\sqrt{l}=l\), and so \(l=1\).

If instead \(a>1\), then \(a>\sqrt{a}>1\), and so \((x_n)\) is monotonic decreasing and bounded below by \(1\). So the sequences converges to a limit, and by an identical argument to above, this limit has to be \(1\).

\sphinxAtStartPar
(ii) For all \(a>0, n\in\mathbb{N}, f(a) = f\left(a^{\frac{1}{2}}\right) = \cdots = f\left(a^{\frac{1}{2^{n-1}}}\right)\). That is, \(f(a)=f(x_n)\) for all \(n\in\mathbb{N}\). By continuity,
\begin{equation*}
\begin{split}
f(a) = \lim_{n\rightarrow\infty} f(x_n) = f\left(\lim_{n\rightarrow\infty} x_n\right) = f(1).
\end{split}
\end{equation*}
\sphinxAtStartPar
If \(a<0\), then \(a^2>0\) and so \(f(a)=f\left(a^2\right)=f(1)\). Finally, by continuity,
\begin{equation*}
\begin{split}
f(0)=\lim_{x\rightarrow 0}f(x) = \lim_{x\rightarrow 0}f(1) = f(1),
\end{split}
\end{equation*}
\sphinxAtStartPar
and we have shown that \(f(x)=f(1)\) for all \(x\in\mathbb{R}\).


\subsection{Differentiation}
\label{\detokenize{Solutions-full:differentiation}}
\sphinxAtStartPar
{\hyperref[\detokenize{Problems:id37}]{\sphinxcrossref{\DUrole{std,std-ref}{37.}}}} This is Definition 2.2 in the notes:

The function \(f\) has limit \(l\) at the point \(a\) if for every \(\varepsilon>0\) there exists \(\delta>0\) such that for all \(x\in X\),
\begin{equation*}
\begin{split}
0<|x-a|<\delta \; \text{ implies } \; |f(x)-l|<\varepsilon.
\end{split}
\end{equation*}
\sphinxAtStartPar
We write \(\lim_{x\to a}f(x)=l\).


\bigskip\hrule\bigskip


\sphinxAtStartPar
{\hyperref[\detokenize{Problems:id38}]{\sphinxcrossref{\DUrole{std,std-ref}{38.}}}} (Homework 4 question)
(i) This is Definition 4.1 in the notes: we say that \(f\) is \{\textbackslash{}it differentiable\} at \(a \in A\) if \( \lim_{x \rightarrow a}\frac{f(x) - f(a)}{x - a}\) exists. Or, equivalently, \( \lim_{h \rightarrow 0}\frac{f(a +h) - f(a)}{h}\) exists.

That is, we fix \(a\in A\) and we consider the function \(g\) given by \(g(h)=\frac{f(a +h) - f(a)}{h}\). Then \(f\) is differentiable at \(a\) if the limit of \(g\) as \(h\) goes to zero exists.

\sphinxAtStartPar
(ii) Using the sequential criterion (Theorem 2.1 in the notes):

\(f\) is differentiable at \(a\) if there exists some number \(f'(a) \in \mathbb{R}\) for which, given any sequence \((h_{n})\) that converges to \(0\), with \(h_n\neq 0\) for all \(n\), we have
\begin{equation*}
\begin{split}
\frac{f(a + h_{n}) - f(a)}{h_{n}} \rightarrow f'(a) \hspace{1em} \text{ as } \hspace{1em} n\rightarrow\infty.
\end{split}
\end{equation*}
\sphinxAtStartPar
(iii) This is asking for the equivalent description of limit of a function given by the \((\varepsilon - \delta)\) criterion (Theorem 2.1). Again it needs to be applied to \(g\) as above not \(f\) itself:

We say \(f\) is differentiable at \(a\) if there exists a number \(f'(a)\) for which, given \(\varepsilon>0\), there exists \(\delta > 0\) such that if \(0<|h| < \delta\), then 	
\$\(
\left|\displaystyle\frac{f(a+h) - f(a)}{h} - f'(a)\right| < \varepsilon.
\)\$


\bigskip\hrule\bigskip


\sphinxAtStartPar
{\hyperref[\detokenize{Problems:id39}]{\sphinxcrossref{\DUrole{std,std-ref}{39.}}}} Let \(x\in\mathbb{R}\setminus\{0\}\). Then
\begin{align*}
\frac{f(x + h) -f(x)}{h} &= \frac{1}{h}\left(\frac{1}{x+ h} -\frac{1}{x}\right)\\
&= -\frac{1}{x(x+h)} \rightarrow -\frac{1}{x^{2}},~\mbox{as}~h \rightarrow 0. 
\end{align*}
\sphinxAtStartPar
Thus \(f\) is differentiable at all \(x\in\mathbb{R}\setminus\{0\}\) and \(f'(x)=-\frac{1}{x^2}\) for all \(x\in\mathbb{R}\setminus\{0\}\).

The extended function (where we define its value at \(x=0\) to be zero) is not continuous at \(x=0\), since \(\lim_{x \rightarrow 0^+}\frac{1}{x} =\infty\). Therefore it is not differentiable at \(x=0\), by Theorem 4.1.


\bigskip\hrule\bigskip


\sphinxAtStartPar
{\hyperref[\detokenize{Problems:id40}]{\sphinxcrossref{\DUrole{std,std-ref}{40.}}}} As in the hint, let \(g(h) = e^{kh} - 1 - kh \). Then
\begin{align*}
\frac{f(x + h) -f(x)}{h} &= \frac{1}{h}(e^{k(x +h)} - e^{kx})\\
&= e^{kx}\frac{1}{h}(e^{kh} - 1) = e^{kx}\left(k + \frac{g(h)}{h}\right) \rightarrow ke^{kx},~\mbox{as}~h \rightarrow 0,
\end{align*}
\sphinxAtStartPar
using the given fact that  \(\lim_{h \rightarrow 0}\frac{g(h)}{h} = 0\). Thus \(f\) is differentiable at each \(x\in\mathbb{R}\) and \(f'(x)=ke^{kx}\).


\bigskip\hrule\bigskip


\sphinxAtStartPar
{\hyperref[\detokenize{Problems:id41}]{\sphinxcrossref{\DUrole{std,std-ref}{41.}}}}
(i) Using the product and chain rules for differentiation, and that \(\sin\) is differentiable with derivative \(\cos\),
if \(x \neq 0\), \(f\) is differentiable at \(x\) with \(f'(x) = \sin\left(\frac{1}{x}\right) - \frac{1}{x}\cos\left(\frac{1}{x}\right)\).
(ii) But
\(\frac{f(x) - f(0)}{x} = \sin\left(\frac{1}{x}\right)\) has no limit as \(x \rightarrow 0\) (see {\hyperref[\detokenize{Problems:id9}]{\sphinxcrossref{\DUrole{std,std-ref}{Problem 9}}}}), so \(f\) is not differentiable at \(0\).


\bigskip\hrule\bigskip


\sphinxAtStartPar
{\hyperref[\detokenize{Problems:id42}]{\sphinxcrossref{\DUrole{std,std-ref}{42.}}}} Again using the product and chain rules and standard derivatives,
for \(x \neq 0\), \(f\) is differentiable at \(x\) with \(f'(x) = 2x\sin\left(\frac{1}{x}\right) - \cos\left(\frac{1}{x}\right)\). At \(x = 0,\)
\begin{equation*}
\begin{split}
\lim_{x \rightarrow 0}\frac{f(x) - f(0)}{x} = \lim_{x \rightarrow 0}x\sin\left(\frac{1}{x}\right) = 0,
\end{split}
\end{equation*}
\sphinxAtStartPar
by {\hyperref[\detokenize{Problems:id10}]{\sphinxcrossref{\DUrole{std,std-ref}{Problem 10}}}}. So \(f\) is differentiable at \(0\) with \(f'(0) = 0\). For the second derivative, with \(x \neq 0\), we have
\begin{equation*}
\begin{split}
f^{\prime \prime}(x) = 2\sin\left(\frac{1}{x}\right) - \frac{2}{x}\cos\left(\frac{1}{x}\right) - \frac{1}{x^{2}}\sin\left(\frac{1}{x}\right).
\end{split}
\end{equation*}
\sphinxAtStartPar
But \(f^{\prime \prime}\) doesn’t exist at \(x = 0\), as in {\hyperref[\detokenize{Problems:id41}]{\sphinxcrossref{\DUrole{std,std-ref}{Problem 41}}}}.


\bigskip\hrule\bigskip


\sphinxAtStartPar
{\hyperref[\detokenize{Problems:id43}]{\sphinxcrossref{\DUrole{std,std-ref}{43.}}}} We give sketches of simple examples of functions as described. There should be some kind of “corner” or “cusp” at the relevant points, so that there is clearly no well\sphinxhyphen{}defined gradient there. But the functions are required to be continuous, so there should be no jump or other discontinuity.

\sphinxAtStartPar
(i)

\begin{figure}[htbp]
\centering
\capstart

\noindent\sphinxincludegraphics[width=500\sphinxpxdimen]{{nondiff1,2}.png}
\caption{Graph of a function \(f:[0,1]\to\mathbb{R}\) that is not differentiable at \(\frac{1}{2}\), but is everywhere else.}\label{\detokenize{Solutions-full:q43i}}\end{figure}

\sphinxAtStartPar
(ii)

\begin{figure}[htbp]
\centering
\capstart

\noindent\sphinxincludegraphics[width=500\sphinxpxdimen]{{nondiff1,3_2,3}.png}
\caption{Graph of a function \(f:[0,1]\to\mathbb{R}\) that is differentiable at all points in its domain apart from \(\frac{1}{3}\) and \(\frac{2}{3}\).}\label{\detokenize{Solutions-full:q43ii}}\end{figure}


\bigskip\hrule\bigskip


\sphinxAtStartPar
{\hyperref[\detokenize{Problems:id44}]{\sphinxcrossref{\DUrole{std,std-ref}{44.}}}} It helps to sketch the graph. On the interval \([0,1]\), we have \([x]=0\) and so the graph resembles \(y=x\). This pattern then repeats, and we have the following graph for \(f(x)=x-[x]\):

\begin{figure}[htbp]
\centering
\capstart

\noindent\sphinxincludegraphics[width=500\sphinxpxdimen]{{x-[x]}.png}
\caption{Graph of the function \(f:\mathbb{R}\to\mathbb{R}\); \(f(x)=x-[x]\).}\label{\detokenize{Solutions-full:q44}}\end{figure}

\sphinxAtStartPar
The function \(f\) is differentiable for all \(x \in \mathbb{R} \setminus \mathbb{Z}\). For such points, taking \(h>0\) sufficiently small, we have
\([x+h]=[x]\) and so
\begin{equation*}
\begin{split}
\frac{f(x+h)-f(x)}{h} = \frac{(x + h)- [x + h] - x + [x]}{h} = \frac{h}{h}=1.
\end{split}
\end{equation*}
\sphinxAtStartPar
Thus \(f'(x) = 1\)  for all \(x \in \mathbb{R} \setminus \mathbb{Z}\). If \(x \in \mathbb{Z}\), then \(f\) is not continuous at \(x\)  (it has a jump discontinuity of \(-1\) there) and so cannot be differentiable there (by Theorem  4.1).


\bigskip\hrule\bigskip


\sphinxAtStartPar
{\hyperref[\detokenize{Problems:id45}]{\sphinxcrossref{\DUrole{std,std-ref}{45.}}}} First observe that if \(f\) is differentiable at \(a\), then
\begin{equation*}
\begin{split}
f'(a) = \lim_{h \rightarrow 0}\frac{f(a + h) - f(a)}{h} = \lim_{h \rightarrow 0}\frac{f(a - h) - f(a)}{-h} = \lim_{h \rightarrow 0}\frac{f(a) - f(a-h)}{h}.
\end{split}
\end{equation*}
\sphinxAtStartPar
So
\begin{align*}
\frac{f(a+ h) - f(a - h)}{2h} &=  \frac{1}{2}\left(\frac{f(a + h) - f(a)}{h} +  \frac{f(a) - f(a-h)}{h}\right) \\
&\hspace{7em}\rightarrow  \frac{1}{2}2f'(a) = f'(a),~\mbox{as}~h \rightarrow 0. 
\end{align*}
\sphinxAtStartPar
On the other hand, by Example 5.2.5, \(f:\mathbb{R}\to\mathbb{R}\) given by
\(f(x)=|x|\) is not differentiable at \(x=0\). But
\begin{equation*}
\begin{split}
\lim_{h \rightarrow 0^+} \displaystyle\frac{|h|-|{}-h|}{2h} =0.
\end{split}
\end{equation*}

\bigskip\hrule\bigskip


\sphinxAtStartPar
{\hyperref[\detokenize{Problems:id46}]{\sphinxcrossref{\DUrole{std,std-ref}{46.}}}}
(i) True: \(f\) is continuous at zero as \(f(0) = 0= \lim_{x \rightarrow 0^-}f(x) = \lim_{x \rightarrow 0^+}f(x)\).

\sphinxAtStartPar
(ii) True: \(f'(0)\) exists and is zero. To see this compute
\begin{equation*}
\begin{split}
f'_{+}(0) = \lim_{h \rightarrow 0^+}\frac{h^{2}}{h} = \lim_{h \rightarrow 0^+} h= 0.
\end{split}
\end{equation*}
\sphinxAtStartPar
and
\begin{equation*}
\begin{split}
f'_{-}(0)= \lim_{h \rightarrow 0^-}\frac{-h^{2}}{h} = \lim_{h \rightarrow 0^-} -h =0.
\end{split}
\end{equation*}
\sphinxAtStartPar
(iii) True: \(f'\) is continuous at zero since \(f':\mathbb{R}\to\mathbb{R}\) is given by
\begin{equation*}
\begin{split}
f'(x) = \left\{\begin{array}{c c} -2x & ~\mbox{if}~x < 0\\ 0& ~\mbox{if}~x =0 \\ 2x & ~\mbox{if}~x > 0 \end{array} \right.,
\end{split}
\end{equation*}
\sphinxAtStartPar
so \(f'(0) = \lim_{x \rightarrow 0^-} f'(x) = \lim_{x \rightarrow 0^+}f'(x) = 0\).

\sphinxAtStartPar
(iv) False:  \(f^{\prime \prime}_{+}(0) = \lim_{h \rightarrow 0^+}\frac{2h - 0}{h} = 2, f^{\prime \prime}_{-}(0) = \lim_{h \rightarrow 0^-}\frac{-2h - 0}{h} = -2\), and so \(f^{\prime \prime}(0)\) does not exist.


\bigskip\hrule\bigskip


\sphinxAtStartPar
{\hyperref[\detokenize{Problems:id47}]{\sphinxcrossref{\DUrole{std,std-ref}{47.}}}}
(i) Yes: \(f\) is differentiable on \([a, b]\) and hence continuous on \([a, b]\) by Theorem  4.1, so it attains its sup and inf on \([a, b]\) by the extreme value theorem, Theorem 3.5, and these are the maximum and minimum (respectively).

\sphinxAtStartPar
(ii) No: the maximum or minimum could be \(f(a)=f(b)\).
If \(f(a)\) is not the maximum value, then this must occur in \((a, b)\). Similarly for the minimum. If \(f(a) = f(b)\) is both the maximum and minimum value, then \(f\) is constant, and the value occurs in \((a, b)\).


\bigskip\hrule\bigskip


\sphinxAtStartPar
{\hyperref[\detokenize{Problems:id48}]{\sphinxcrossref{\DUrole{std,std-ref}{48.}}}} Following the hint, define \(g:\mathbb{R}\rightarrow \mathbb{R}\) by
\begin{equation*}
\begin{split}
g(x) = a_{0}x + \frac{a_{1}}{2}x^{2} + \frac{a_{2}}{3}x^{3}  + \cdots + \frac{a_{n}}{n+1}x^{n}.
\end{split}
\end{equation*}
\sphinxAtStartPar
Then \(g\) is a polynomial, so it is differentiable on \([0, 1]\) and using standard derivatives, \(g'(x)=f(x)\). Also \(g(0) = g(1) = 0\). So by Rolle’s theorem, there exists \(c \in (0, 1)\) such that \(g'(c) = 0\), i.e.\(f(c) = 0\).


\bigskip\hrule\bigskip


\sphinxAtStartPar
{\hyperref[\detokenize{Problems:id49}]{\sphinxcrossref{\DUrole{std,std-ref}{49.}}}}
(i) Let \(x, y\) be such that \(a \leq x < y \leq b\). We want to show that \(f(x)=f(y)\).

\sphinxAtStartPar
Apply the mean value theorem to the restriction of \(f\) to the interval \([x, y]\), to find there exists \(d \in (x, y)\) for which
\begin{equation*}
\begin{split}
f(y) - f(x) = f'(d)(y-x).
\end{split}
\end{equation*}
\sphinxAtStartPar
By assumption, \(f'(c)=0\) for all \(c\in(a,b)\), so \(f'(d)=0\).

\sphinxAtStartPar
Thus \(f(x)=f(y)\). Since this holds for all \(x,y\in [a,b]\), \(f\) is constant on \([a,b]\).

\sphinxAtStartPar
(ii) Define \(f:\mathbb{R}\to\mathbb{R}\) by \(f = h-g\). Then the function \(f\) is continuous on \([a,b]\) and differentiable on \((a,b)\), because \(g\) and \(h\) are, and \(f'(x)=h'(x)-g'(x)=0\) for all \(x\in (a,b)\). Thus \(f\) satisfies all the conditions of (i). By part (i), \(f\) is constant on \([a,b]\). That is, there exists some \(k\in \mathbb{R}\), such that \(f(x)=k\) for all \(x\in [a,b]\). Thus \(h(x)=g(x)+k\) for all \(x\in [a,b]\).


\bigskip\hrule\bigskip


\sphinxAtStartPar
{\hyperref[\detokenize{Problems:id50}]{\sphinxcrossref{\DUrole{std,std-ref}{50.}}}} By the mean value theorem, \(f(b) = f(a) + f'(c)(b - a)\) for some \(c\in (a,b)\). So, since \(m \leq f'(c) \leq M\), for all \(c \in (a, b)\),
\begin{equation*}
\begin{split}
f(a) + m(b - a) \leq f(b) \leq f(a) + M(b - a).
\end{split}
\end{equation*}

\bigskip\hrule\bigskip


\sphinxAtStartPar
{\hyperref[\detokenize{Problems:id51}]{\sphinxcrossref{\DUrole{std,std-ref}{51.}}}} The polynomial \(p\) is of odd degree so it has at least one real root by Corollary 3.2. Also \(p\) is differentiable with \(p'(x) = 3x^{2} + r > 0\), for all \(x \in \mathbb{R}\), so \(p\) is strictly monotonic increasing on any closed interval \([a, b]\), and hence on the whole of \(\mathbb{R}\), by Corollary 4.1. Then by the inverse function theorem (Theorem 3.6), \(p\) is invertible and hence injective, and so there is exactly one zero.


\bigskip\hrule\bigskip


\sphinxAtStartPar
{\hyperref[\detokenize{Problems:id52}]{\sphinxcrossref{\DUrole{std,std-ref}{52.}}}} Let \(r>1\) and fix \(y\in (0,1)\). Apply the mean value theorem to the function \(f:[y,1]\to\mathbb{R}\) given by \(f(x) = x^r\). Then there exists \(c \in (y, 1)\) such that
\begin{equation*}
\begin{split}
\frac{f(1)-f(y)}{1-y}=f'(c)=rc^{r-1}.
\end{split}
\end{equation*}
\sphinxAtStartPar
Now, \(0<c<1\), and \(r>1\), so \(c^{r-1}<1\). Thus
\begin{equation*}
\begin{split}
1 - y^r = rc^{r-1}( 1- y) < r(1 - y).
\end{split}
\end{equation*}

\bigskip\hrule\bigskip


\sphinxAtStartPar
{\hyperref[\detokenize{Problems:id53}]{\sphinxcrossref{\DUrole{std,std-ref}{53.}}}} Consider the first case. Here we have
\begin{equation*}
\begin{split}
f''(a) = \lim_{h \rightarrow 0}\frac{f'(a + h) - f'(a)}{h} =  \lim_{h \rightarrow 0}\frac{f'(a + h)}{h} < 0.
\end{split}
\end{equation*}
\sphinxAtStartPar
In particular, \(f^{\prime \prime}_{+}(a) = \lim_{h \rightarrow 0^+}\frac{f'(a + h)}{h} < 0\), and so \(f'(a + h) < 0\) for sufficiently small positive \(h\), (say \(0 < h < \delta\)), in which case, by Corollary 4.1, \(f\) is strictly decreasing on \([a,  a + \delta]\). We also have \(f^{\prime \prime}_{-}(a) = \lim_{h \rightarrow 0^-}\frac{f'(a + h)}{h} < 0\), and so \(f'(a + h) > 0\) for sufficiently small negative \(h\), (say \(-\delta_{1} < h < 0\)), in which case, by Corollary 4.1, \(f\) is strictly increasing on \([a- \delta_{1}, a]\). Then it follows that \(f\) has a local minimum at \(a\). The other case is similar.


\subsection{Sequences and series of functions}
\label{\detokenize{Solutions-full:sequences-and-series-of-functions}}
\sphinxAtStartPar
{\hyperref[\detokenize{Problems:id54}]{\sphinxcrossref{\DUrole{std,std-ref}{54.}}}} For \(x\in [0,\pi ]\) we have \(0\leq \sin x<1\) except when \(x=\frac{\pi}{2}\), where \(\sin x=1\). Thus \(f_n(x)\) converges pointwise to the function \(f\colon [0, \pi ]\rightarrow \mathbb{R}\) defined by
\begin{equation*}
\begin{split}
f(x) = \left\{ \begin{array}{ll}
0 & x\neq \frac{\pi}{2}, \\
1 & x=\frac{\pi}{2}. \\
\end{array} \right.
\end{split}
\end{equation*}
\sphinxAtStartPar
Each function \(f_n\) is continuous. The function \(f\) is not continuous. So the sequence \((f_n)\) does not converge uniformly (otherwise we would have a contradiction to Theorem 5.1 — the uniform limit theorem).


\bigskip\hrule\bigskip


\sphinxAtStartPar
{\hyperref[\detokenize{Problems:id55}]{\sphinxcrossref{\DUrole{std,std-ref}{55.}}}}
(i) Observe that \((f_n)\) converges pointwise to the function
\begin{equation*}
\begin{split}
f(x) = \left\{ \begin{array}{ll}
0 & x=0, \\
1 & x>0. \\
\end{array} \right.
\end{split}
\end{equation*}
\sphinxAtStartPar
Each function \((f_n)\) is continuous, but the function \(f\) is not continuous at \(0\), so the convergence cannot be uniform.

\sphinxAtStartPar
(ii) Let \(x\in \mathbb{R}\). Then for any \(n\geq x\), we have \(f_n (x) =0\), so the sequence \((f_n)\) converges pointwise to the zero function.

On the other hand
\begin{equation*}
\begin{split}
M_n = \sup \{ |f_n(t) -0| : t\in \mathbb{R} \} = \infty,
\end{split}
\end{equation*}
\sphinxAtStartPar
so certainly \((M_n)\) does not converge to zero, and the sequence \((f_n)\) therefore does not converge uniformly to the zero function.

\sphinxAtStartPar
(iii) For \(x>1\), \(\frac{1}{x^n} \rightarrow 0\) as \(n\rightarrow \infty\). Hence for each \(x\in (1,\infty )\), \(\frac{e^x}{x^n}\rightarrow 0\) as \(n\rightarrow \infty\), and the sequence \((f_n)\) converges pointwise to the zero function.

But \(\frac{e^x}{x^n}\rightarrow \infty\) as \(x\rightarrow \infty\), so
\begin{equation*}
\begin{split}
M_n = \sup \{ |f_n(t) -0| : t\in (1,\infty ) \} = \infty.
\end{split}
\end{equation*}
\sphinxAtStartPar
So certainly \((M_n)\) does not converge to zero, and the sequence \((f_n)\) therefore does not converge uniformly to the zero function.

\sphinxAtStartPar
(iv) We know that \(e^{-k}\rightarrow 0\) as \(k\rightarrow \infty\), so \(f_n(x)\rightarrow 0\) as \(n\rightarrow \infty\) if \(x\neq 0\). If \(x=0\), then \(f_n(x)=1\) for all \(n\). So \((f_n)\) converges pointwise to the function
\begin{equation*}
\begin{split}
f(x) = \left\{ \begin{array}{ll}
1 & x=0, \\
0 & x\neq 0. \\
\end{array} \right.
\end{split}
\end{equation*}
\sphinxAtStartPar
Each function \((f_n)\) is continuous, but the function \(f\) is not continuous at \(0\), so the convergence cannot be uniform.

\sphinxAtStartPar
(v) The sequence \((f_n)\) clearly does not converge pointwise.


\bigskip\hrule\bigskip


\sphinxAtStartPar
{\hyperref[\detokenize{Problems:id56}]{\sphinxcrossref{\DUrole{std,std-ref}{56.}}}} We consider \(g_n:D\to\mathbb{R}\), and consider the cases \(D=[0,1]\) and \(D=[0,\infty)\), with \(g_n(x)\) given by different expressions in each part below.

By Proposion 5.2, \((g_n)\) will converge uniformly to a function \(g\) if and only if
\begin{equation*}
\begin{split}
M_n := \sup \{ |g_n(t) -g(t)| : t\in D \} \rightarrow 0 \text{ as } n\rightarrow \infty.
\end{split}
\end{equation*}
\sphinxAtStartPar
(i) If \(\displaystyle g_n(x)=\frac{x}{n}\), then for each \(x\), \(g_n(x)\rightarrow 0\) as \(n\rightarrow \infty\), so the sequence \((g_n)\) converges pointwise to the zero function.

On the interval \([0,1]\),
\begin{equation*}
\begin{split}
M_n = \sup \{ |g_n(t) -0 | : t\in [0,1] \}  = \frac{1}{n}
\end{split}
\end{equation*}
\sphinxAtStartPar
Note that \(M_n\rightarrow 0\) as \(n\rightarrow \infty\), so on the interval \([0,1]\), the sequence \((g_n)\) converges uniformly to the zero function.

On the other hand,  on \([0,\infty )\),
\begin{equation*}
\begin{split}
M_n = \sup \{ |g_n(t) -0 | : t\in [0,\infty ) \}  = \infty,
\end{split}
\end{equation*}
\sphinxAtStartPar
so the sequence \((g_n)\) does not converge uniformly on \([0,\infty )\).

\sphinxAtStartPar
(ii) Now consider \(\displaystyle g_n(x) := \frac{x^n}{1+x^n}\). Observe that
\begin{equation*}
\begin{split}
g_n (x) = \frac{1}{1+\frac{1}{x^n}} \rightarrow \left\{ \begin{array}{ll}
0 & x<1, \\
\frac{1}{2} & x=1, \\
1 & x>1. \\
\end{array} \right.
\end{split}
\end{equation*}
\sphinxAtStartPar
Thus the pointwise limit function is not continuous on \([0,1]\) or on \([0,\infty )\), so convergence is not uniform on either interval.

\sphinxAtStartPar
(iii) Let \(\displaystyle g_n(x) =  \frac{x^n}{n+x^n}\).

\sphinxAtStartPar
If \(0\leq x\leq 1\), then \(\frac{n}{x^n} \rightarrow \infty\) as \(n\rightarrow \infty\), so \(g_n(x)\rightarrow 0\) as \(n\rightarrow \infty\). If \(x>1\), then \(\frac{n}{x^n} \rightarrow 0\) as \(n\rightarrow \infty\), so \(g_n(x)\rightarrow 1\) as \(n\rightarrow \infty\). So \((g_n)\) converges pointwise to
\begin{equation*}
\begin{split}
g (x) =  \left\{ \begin{array}{ll}
0 & x\leq 1, \\
1 & x>1. \\
\end{array} \right.
\end{split}
\end{equation*}
\sphinxAtStartPar
On the interval \([0,1]\),
\begin{equation*}
\begin{split}
M_n = \sup \{ |g_n(t) -0 | : t\in [0,1] \}  = \frac{1}{1+n} \rightarrow 0
\end{split}
\end{equation*}
\sphinxAtStartPar
as \(n\rightarrow \infty\). So the sequence \((g_n)\) converges uniformly to the zero function on \([0,1]\).

On the other hand the above pointwise limit \(g\) is not continuous on \([0,\infty )\), whereas each function \(g_n\) is continuous on \([0,\infty )\), so convergence is not uniform on \([0,\infty )\).


\bigskip\hrule\bigskip


\sphinxAtStartPar
{\hyperref[\detokenize{Problems:id57}]{\sphinxcrossref{\DUrole{std,std-ref}{57.}}}}
(i) Observe that, for all \(x\in[0,1]\), \(h_n(x)\rightarrow 1\) as \(n\rightarrow \infty\).
So \((h_n)\) converges pointwise to the constant function \(h:[0,1]\to\mathbb{R}\), \(h(x)=1\) for all \(x\).

\sphinxAtStartPar
Let
\$\(
M_n = \sup \{ |h_n(x)-1| : x\in [0,1] \} = 1 - \left(1-\frac{1}{n}\right)^2 =\frac{1}{n^2}-\frac{2}{n}.
\)\$

\sphinxAtStartPar
So \(M_n\rightarrow 0\) as \(n\rightarrow \infty\). So \((h_n)\) converges uniformly to the constant function \(h\).

\sphinxAtStartPar
(ii) We know that \(x^n \rightarrow 0\) as \(n\rightarrow \infty\) if \(0\leq x<1\), and \(x^n\rightarrow 1\) as \(n\rightarrow \infty\) if \(x=1\). Hence \((h_n)\) converges pointwise to the function \(h\), where
\begin{equation*}
\begin{split}
h (x) =\left\{ \begin{array}{ll}
x & 0\leq x<1, \\
0 & x=1. \\
\end{array} \right.
\end{split}
\end{equation*}
\sphinxAtStartPar
Each function \(h_n\) is continuous, and this limit is not continuous, so convergence is not uniform.

\sphinxAtStartPar
(iii) The sequence \((h_n(x))\) has pointwise limit \(\frac{1}{1-x}\) if \(0\leq x<1\) (geometric series), but it does not converge if \(x=1\). Because it does not converge when \(x=1\), the sequence of functions \((h_n)\) does not converge pointwise  on \([0,1]\) (and so it certainly does not converge uniformly on \([0,1]\)).


\bigskip\hrule\bigskip


\sphinxAtStartPar
{\hyperref[\detokenize{Problems:id58}]{\sphinxcrossref{\DUrole{std,std-ref}{58.}}}}
(i) For each \(t\), we have
\begin{equation*}
\begin{split}
f_n(t) = \frac{n+\cos x}{2n+\sin^2 x} = \frac{1+\frac{\cos x}{n}}{2+\frac{\sin^2 x}{n}} \rightarrow \frac{1}{2}
\end{split}
\end{equation*}
\sphinxAtStartPar
as \(n\rightarrow \infty\). So the sequence \((f_n)\) converges pointwise to the constant function \(f:\mathbb{R}\to\mathbb{R}\), with \(f(x)=\frac{1}{2}\) for all \(x\in\mathbb{R}\).

\sphinxAtStartPar
(ii) Note that, for all \(x\in\mathbb{R}\),
\begin{align*}
\left|f_n(x) - \frac{1}{2}\right| =  \left| \frac{2n+2\cos x -2n -\sin^2 x}{2(2n+\sin^2 x)} \right| 
&= \left| \frac{2\cos x -\sin^2 x}{2(2n+\sin^2 x)} \right| \\
&\leq \frac{2+1}{2(2n)}=\frac{3}{4n} \rightarrow 0 
\end{align*}
\sphinxAtStartPar
as \(n\rightarrow \infty\).

It follows that \((f_n)\) converges to \(f\) uniformly.

\sphinxAtStartPar
(iii) Using the quotient rule,
\begin{align*}
f_n'(x) &= \frac{-\sin(x)(2n+\sin^2(x))-(n+\cos(x))\cdot 2\sin(x)\cos(x)}{\left(2n+\sin^2(x)\right)^2} \\
&= \frac{-\sin(x)\left(2n+\sin^2(x)+2n\cos(x)+2\cos^2(x)\right)}{\left(2n+\sin^2(x)\right)^2} \\
&= \frac{-\sin(x)\left(2n+1+2n\cos(x)+\cos^2(x)\right)}{\left(2n+\sin^2(x)\right)^2}.
\end{align*}
\sphinxAtStartPar
Since \(f\) is a constant function, \(f'(x)=0\) for all \(x\in\mathbb{R}\). We show that \(f_n'(x)\rightarrow 0\) as \(n\rightarrow\infty\), for each \(x\in\mathbb{R}\).

We have:
\begin{align*}
|f_n'(x)| &= \frac{|\sin(x)|\left|2n+1+2n\cos(x)+\cos^2(x)\right|}{\left(2n+\sin^2(x)\right)^2} \\
&\leq \frac{2n+1+2n|\cos(x)|+|\cos(x)|^2}{\left(2n\right)^2},
\end{align*}
\sphinxAtStartPar
(using the triangle inequality and the fact that \(\sin^2(x)\geq 0\)), so
\begin{align*}
|f_n'(x)| &\leq \frac{4n+2}{4n^2} = \frac{2n+1}{2n^2} \rightarrow 0 \text{ as } n\rightarrow \infty.
\end{align*}
\sphinxAtStartPar
Therefore \(f_n'(x)\rightarrow 0\) as \(n\rightarrow \infty\), for all \(x\in\mathbb{R}\).

The conditions of Theorem 5.2 will be satisfied if we can prove that \(f_n'\rightarrow 0\) uniformly.

We have already shown that \(|f_n'(x)|\leq \frac{2n+1}{2n^2}\) for all \(x\in\mathbb{R}\) and \(n\in\mathbb{N}\), so most of the work is already done here. We have
\begin{equation*}
\begin{split}
M_n := \sup\{|f_n(x)|:x\in\mathbb{R}\} \leq \frac{2n+1}{2n^2}\rightarrow 0 \text{ as } n\rightarrow\infty,
\end{split}
\end{equation*}
\sphinxAtStartPar
so by Proposition 5.2, \(f_n'\rightarrow 0\) uniformly. So \((f_n)\) satisfies the conditions of Theorem 5.2.


\bigskip\hrule\bigskip


\sphinxAtStartPar
{\hyperref[\detokenize{Problems:id59}]{\sphinxcrossref{\DUrole{std,std-ref}{59.}}}}
(i) As indicated in the question, we need to check continuity at \(x=0\) and \(x=1\).

Let \(0<x<1\), so
\begin{equation*}
\begin{split}
f_n(x) = \frac{x^{\frac{1}{n}}-1}{\ln x}.
\end{split}
\end{equation*}
\sphinxAtStartPar
We have \(x^{\frac{1}{n}} -1 \rightarrow -1 \neq 0\) as \(x\rightarrow 0\), and \(\ln x\rightarrow -\infty\) as \(x\rightarrow 0\). So clearly
\begin{equation*}
\begin{split}
\frac{x^{\frac{1}{n}}-1}{\ln x} \rightarrow 0 =f_n(0)
\end{split}
\end{equation*}
\sphinxAtStartPar
as \(x\rightarrow 0\), and \(f_n\) is continuous at \(0\).

On the other hand,
\begin{equation*}
\begin{split}
x^{\frac{1}{n}} -1 \rightarrow 1-1=0 \qquad \text{and}\qquad \ln x\rightarrow 0
\end{split}
\end{equation*}
\sphinxAtStartPar
as \(x\rightarrow 1\), so by L’H\textasciicircum{}opital’s rule,
\begin{equation*}
\begin{split}
\lim_{x\rightarrow 1} \frac{x^{\frac{1}{n}} -1}{\ln x} = \lim_{x\rightarrow 1} \frac{\left(\frac{1}{n}\right)x^{\frac{1}{n}-1}}{\frac{1}{x}} = \frac{1}{n} = f_n(1),
\end{split}
\end{equation*}
\sphinxAtStartPar
so  \(f_n\) is continuous at \(1\).

\sphinxAtStartPar
(ii) Since we can assume the function \(f_n\) is increasing, we have
\begin{equation*}
\begin{split}
|f_n(x) |\leq \frac{1}{n}
\end{split}
\end{equation*}
\sphinxAtStartPar
for all \(x\in [0,1]\). But \(\frac{1}{n}\rightarrow 0\) as \(n\rightarrow \infty\), so it follows that \((f_n)\) converges uniformly to the zero function.


\bigskip\hrule\bigskip


\sphinxAtStartPar
{\hyperref[\detokenize{Problems:id60}]{\sphinxcrossref{\DUrole{std,std-ref}{60.}}}} Let \(s_n(x)=\sum_{i=1}^n f_i(x)\). Then \((s_n)\) converges uniformly to \(f\), and by Proposition 5.2, this is equivalent to \(\sup\{|s_n(x)-f(x)|\}\to 0\) as \(n\to\infty\). So
\begin{align*}
\sup\{|f_n(x)|\}&=\sup\{|s_n(x)-s_{n-1}(x)|\}\\
&\leq \sup \{|s_n(x)-f(x)|+|f(x)-s_{n-1}(x)|\}\\
&\leq \sup \{|s_n(x)-f(x)|\}+\sup\{|f(x)-s_{n-1}(x)|\}\\
&\qquad\to 0, \quad\text{as $n\to\infty$.}
\end{align*}
\sphinxAtStartPar
So by Proposition 5.2, \((f_n)\) converges uniformly to \(0\).


\bigskip\hrule\bigskip


\sphinxAtStartPar
{\hyperref[\detokenize{Problems:id61}]{\sphinxcrossref{\DUrole{std,std-ref}{61.}}}} For each \(n\in\mathbb{N}\), define \(f_n,g_n:\mathbb{R}\to\mathbb{R}\) by \(f_n(x)=\frac{(-1)^nx^{2n+1}}{(2n+1)!}\) and \(g_n(x)=\frac{(-1)^nx^{2n}}{(2n)!}\).

\sphinxAtStartPar
(i) To establish absolute pointwise summability of \((f_n)\), observe that for each \sphinxstylestrong{fixed} \(x\in\mathbb{R}\),
\begin{equation*}
\begin{split}
\frac{|f_{n+1}(x)|}{|f_n(x)|} = \frac{x^{2n+3}(2n+1)!}{(2n+3)!x^{2n+1}} = \frac{x^2}{(2n+3)(2n+1)} \rightarrow 0,
\end{split}
\end{equation*}
\sphinxAtStartPar
as \(n\rightarrow\infty\). Therefore by the ratio test, \(s(x) = \sum_{n=0}^\infty f_n(x)\) is absolutely convergent for each \(x\in\mathbb{R}\).

\sphinxAtStartPar
A similar argument applied instead to the seqence \((g_n)\) will show absolute pointwise convergence of the series associate with \(c\) (make sure you can write out the details yourself, though!).

\sphinxAtStartPar
(ii) Let \(R>0\), and restrict each \(f_n\) defined above to the interval \([-R,R]\). Let \(M_n=\frac{R^{2n+1}}{(2n+1)!}\) for each \(n\in\mathbb{N}\). Then
\begin{equation*}
\begin{split}
|f_n(x)| = \frac{x^{2n+1}}{(2n+1)!} \leq M_n
\end{split}
\end{equation*}
\sphinxAtStartPar
for all \(x\in[-R,R]\) and \(n\in\mathbb{N}\). By the same ratio test argument as above, \((M_n)\) is a summable sequence of real numbers, and so by the Weierstrass \(M\)\sphinxhyphen{}test the series \(s:=\sum_{n=0}^\infty f_n\) converges uniformly on \([-R,R]\).

Applying the same argument to \((g_n)\) establishes the result for \(c\).

\sphinxAtStartPar
(iii) Observe that \(f_n\) and \(g_n\) are differentiable everywhere for all \(n\in\mathbb{N}\), and for each \(x\in\mathbb{R}\),
\begin{equation*}
\begin{split}
f_n'(x) = \frac{(-1)^n(2n+1)x^{2n}}{(2n+1)!}=\frac{(-1)^nx^{2n}}{(2n)!}=g_n(x) \hspace{2em} \text{ for } n=0,1,2,\ldots
\end{split}
\end{equation*}
\sphinxAtStartPar
and
\begin{equation*}
\begin{split}
g_n'(x) = \frac{(-1)^n(2n)x^{2n-1}}{(2n)!} = \frac{(-1)^nx^{2n-1}}{(2n-1)!} = -f_{n-1}(x) \hspace{2em} \text{ for } n=1,2,\ldots.
\end{split}
\end{equation*}
\sphinxAtStartPar
Also, \(g_0(x)=1\), so \(g_0'(x)=0\) for all \(x\).

We can now apply Theorem 5.4 to the sequences \((f_n)\) and \((g_n)\) to obtained the desired results for \(s\) and \(c\). The uniform summability of \((f_n')\) and \((g_n')\) follows from that of \((g_n)\) and \((f_n)\), respectively. Clearly each term \(f_n'\) and \(g_n'\) are also continuous, and so by Theorem 5.4, \(s=\sum_{n=0}^\infty f_n\) and \(c=\sum_{n=0}^\infty g_n\) are both differentiable on \((-R,R)\), and can be differentiated term by term. But \(R\) was arbitrary, and so in fact we have shown that \(s\) and \(c\) are differentiable everywhere, with derivatives given by
\begin{equation*}
\begin{split}
s'(x) = \sum_{n=0}^\infty g_n(x) = c(x)
\end{split}
\end{equation*}
\sphinxAtStartPar
and
\begin{equation*}
\begin{split}
c'(x) = \sum_{n=1}^\infty (-f_{n-1}(x)) = -s(x),
\end{split}
\end{equation*}
\sphinxAtStartPar
for all \(x\in\mathbb{R}\).

\sphinxAtStartPar
(iv) The real crux of this part of this question is that the series’ for \(\exp\), \(s(x)\) and \(c(x)\) are absolutely convergent for each \(x\in R\). This allows us to rearrange their terms without disturbing any convergence properties — this was the content of Theorem 4.18 in MAS107.

\sphinxAtStartPar
Using rearrangement, we have for each \(x\in\mathbb{R}\),
\begin{align*}
\exp(ix) = \sum_{n=0}^\infty\frac{(ix)^n}{n!} &= \sum_{n \text{ even}}^\infty\frac{(ix)^n}{n!}  + \sum_{n \text{ odd}}^\infty\frac{(ix)^n}{n!} \\
&= \sum_{k=0}^\infty\frac{(ix)^{2k}}{(2k)!} + \sum_{k=0}^\infty\frac{(ix)^{2k+1}}{(2k+1)!} \\
&= \sum_{k=0}^\infty\frac{(-1)^kx^{2k}}{(2k)!} + \sum_{k=0}^\infty\frac{i(-1)^kx^{2k+1}}{(2k+1)!} \\
&= c(x) + is(x).
\end{align*}

\bigskip\hrule\bigskip


\sphinxAtStartPar
{\hyperref[\detokenize{Problems:id62}]{\sphinxcrossref{\DUrole{std,std-ref}{62.}}}}
(i) Let \(M_n =\frac{1}{n}^2\). Note that
\$\(
\frac{1}{n^2+x^2} \leq \frac{1}{n^2}
\)\$

\sphinxAtStartPar
for all \(x\in \mathbb{R}\), and the series \(\sum_{n=1}^\infty M_n\) converges. So by the Weierstrass \(M\)\sphinxhyphen{}test, the series \( \sum_{n=1}^\infty \frac{1}{n^2 +x^2}\) converges uniformly on \(\mathbb{R}\).

\sphinxAtStartPar
It also converges uniformly on \([0,1]\) by the same argument.

\sphinxAtStartPar
(ii) Observe that, for each \(n\in \mathbb{N}\),
\begin{equation*}
\begin{split}
\sup \left\{ \left| \frac{(-1)^nx^{2n+1}}{(2n+1)!} \right| : x\in \mathbb{R} \right\} = \infty
\end{split}
\end{equation*}
\sphinxAtStartPar
so the series cannot converge uniformly on \(\mathbb{R}\). (The series converging uniformly means the sequence
of partial sums \(s_n\) converging uniformly. But then you would have \(s_{n+1}-s_n\) converging uniformly
to the zero function, which is not the case.)

\sphinxAtStartPar
But let
\begin{equation*}
\begin{split}
M_n = \sup \left\{ \left| \frac{(-1)^nx^{2n+1}}{(2n+1)!} \right|  : x\in [0,1] \right\} = \frac{1}{(2n+1)!}.
\end{split}
\end{equation*}
\sphinxAtStartPar
Then the terms of the series are bounded above by \(M_n\), and \(\sum_{n=0}^\infty M_n\) converges. So the series converges uniformly on \([0,1]\).

\sphinxAtStartPar
(iii) \(\sin (nx)\) does not converge to \(0\) as \(n\rightarrow \infty\) at certain values of \(x\), (for example \(x=\frac{\pi}{4}\in [0,1]\)),  so the series does not converge, let alone uniformly, either on \([0,1]\) or \(\mathbb{R}\).

\sphinxAtStartPar
(iv)* The series \(\sum_{n=1}^\infty  \frac{\sin (nx)}{n}\) does not converge uniformly on \([0,1]\). So it does not converge uniformly on \(\mathbb{R}\) either. To see this, suppose it does converge uniformly. Then the limit \(f(x)\) would be a continuous function on \([0,1]\). Since \(f\) is continuous at \(0\), we have
\begin{equation*}
\begin{split}
0=f(0)=\lim_{N\to\infty} f(\frac{\pi}{N})=\lim_{N\to\infty} \sum_{n=1}^N \frac{\sin\left(\frac{n\pi}{N}\right)}{n}.
\end{split}
\end{equation*}
\sphinxAtStartPar
On the other hand,
\begin{align*}
\sum_{n=1}^N \frac{\sin\left(\frac{n\pi}{N}\right)}{n}&\geq \frac{1}{N}\left( \sum_{n=1}^N \sin\left(\frac{n\pi}{N}\right)\right)\\
&\quad =\frac{1}{N}\left( \frac{\sin(\frac{N}{2}\frac{\pi}{N})\sin(\frac{N+1}{2}\frac{\pi}{N})}{\sin(\frac{\pi}{2N})}\right).
\end{align*}
\sphinxAtStartPar
As \( N\to \infty\), this has limit
\$\(
\lim_{N\to\infty} \frac{1}{N}\left(\frac{\sin(\frac{\left(1+\frac{1}{N}\right)\pi}{2})}{\sin(\frac{\pi}{2N})}\right)
=\lim_{x\to 0}\frac{2}{\pi}\frac{x}{\sin(x)}=\frac{2}{\pi},
\)\$
giving a contradiction.


\bigskip\hrule\bigskip


\sphinxAtStartPar
{\hyperref[\detokenize{Problems:id63}]{\sphinxcrossref{\DUrole{std,std-ref}{63.}}}}
(i) Let \(M_n =a^n\). Then for \(x\in [0,a]\), we have \(|x^n|\leq M_n\), and since \(|a|<1\), the series \(\sum_{n=1}^\infty M_n\) converges.

\sphinxAtStartPar
Hence, by the Weierstrass \(M\)\sphinxhyphen{}test, the series converges uniformly for \(x\in [0,a]\).

\sphinxAtStartPar
(ii) Let
\$\(
S_n(x) = 1+x+x^2+\cdots + x^n.
\)\$

\sphinxAtStartPar
Let
\$\(
A_n = \sup \{ S_n (x) : x\in [0,1 ) \} = S_n(1) =1+1 + \cdots + 1 = n.
\)\$

\sphinxAtStartPar
The sequence  \((A_n)\) does not converge. It follows that the sequence of partial sums \((S_n(x))\) does not converge uniformly on \([0,1)\). Hence the series does not converge uniformly on \([0,1)\).


\bigskip\hrule\bigskip


\sphinxAtStartPar
{\hyperref[\detokenize{Problems:id64}]{\sphinxcrossref{\DUrole{std,std-ref}{64.}}}}
Let \(f_n:[0,2\pi]\to\mathbb{R}\) be given by
\begin{equation*}
\begin{split}
f_n (x) = \frac{\sin (nx )}{n^2}.
\end{split}
\end{equation*}
\sphinxAtStartPar
Then \(f_n\) is continuous. Note that
\begin{equation*}
\begin{split}
|f_n (x )| \leq  \frac{1}{n^2}
\end{split}
\end{equation*}
\sphinxAtStartPar
for all \(x \in [0,2\pi]\), and
\begin{equation*}
\begin{split}
\sum_{n=1}^\infty \frac{1}{n^2}
\end{split}
\end{equation*}
\sphinxAtStartPar
converges.

\sphinxAtStartPar
Hence, by the Weierstrass \(M\)\sphinxhyphen{}test, the series \(\sum_{n=1}^\infty f_n (x )\) converges uniformly for \(x \in [0,2\pi]\). The uniform limit of a series of continuous functions is continuous, so \(f\) is therefore continuous.

\phantomsection\label{\detokenize{Solutions-full:id1}}
\sphinxAtStartPar
{\hyperref[\detokenize{Problems:id65}]{\sphinxcrossref{\DUrole{std,std-ref}{65.}}}}* Note this question is starred, meaning it is intended more for interest and deviates more from the standard benchmark of question for this module

\sphinxAtStartPar
(i) Entering the series partial sums \(\frac{4}{\pi}\sum_{k=0}^N\frac{(-1)^k}{2k+1}\cos\left(\frac{(2k+1)\pi x}{2}\right)\) into \sphinxhref{https://www.desmos.com/calculator/bd3xikfhb0}{Desmos} for a large value of \(N\) yields the following:

\begin{figure}[htbp]
\centering
\capstart

\noindent\sphinxincludegraphics[width=400\sphinxpxdimen]{{fourier1}.png}
\caption{Desmos graph for \(\frac{4}{\pi}\sum_{k=0}^100\frac{(-1)^k}{2k+1}\cos\left(\frac{(2k+1)\pi x}{2}\right)\)}\label{\detokenize{Solutions-full:fourier1}}\end{figure}

\sphinxAtStartPar
This graph was created with \(N=100\), but any \(N>50\) looks similar.

\sphinxAtStartPar
The graph suggests that the infinite series \(\frac{4}{\pi}\sum_{k=0}^\infty\frac{(-1)^k}{2k+1}\cos\left(\frac{(2k+1)\pi x}{2}\right)\) is the Fourier series of the following square wave:

\begin{figure}[htbp]
\centering
\capstart

\noindent\sphinxincludegraphics[width=600\sphinxpxdimen]{{fourier2}.png}
\caption{Square wave \(f(x)=\frac{4}{\pi}\sum_{k=0}^\infty\frac{(-1)^k}{2k+1}\cos\left(\frac{(2k+1)\pi x}{2}\right)\).}\label{\detokenize{Solutions-full:fourier2}}\end{figure}

\sphinxAtStartPar
In \hyperref[\detokenize{Solutions-full:fourier2}]{Fig.\@ \ref{\detokenize{Solutions-full:fourier2}}}, we have been deliberately vague about the value of \(f(x)\) when \(x\) is an odd integer. In fact, direct calculation using the series shows that \(f(x)=0\) for all odd integer values of \(x\).

\begin{figure}[htbp]
\centering
\capstart

\noindent\sphinxincludegraphics[width=600\sphinxpxdimen]{{fourier3}.png}
\caption{Improved graph of the square wave \(f(x)=\frac{4}{\pi}\sum_{k=0}^\infty\frac{(-1)^k}{2k+1}\cos\left(\frac{(2k+1)\pi x}{2}\right)\).}\label{\detokenize{Solutions-full:fourier3}}\end{figure}

\sphinxAtStartPar
In particular,
\begin{equation*}
\begin{split}
f(x) = \left\{\begin{array}{cl} 0 & \text{ for } x \text{ an odd integer},\\
1 & \text{ for } 4k-1<x<4k+1, \; k\in\mathbb{Z}, \\
-1 & \text{ for } 4k+1<x<4k+3 \; k\in\mathbb{Z}  \end{array}\right.
\end{split}
\end{equation*}
\sphinxAtStartPar
Of course, \(f\) is a periodic function with period \(2\).

\sphinxAtStartPar
(ii) We already showed directly in (i) that \(f(x)=0\) whenever \(x\) is an odd integer, but this might not have been obvious from the \sphinxhref{https://www.desmos.com/calculator/bd3xikfhb0}{Desmos graph} alone.

\sphinxAtStartPar
(iii) \(f\) is differentiable at all points in \(\mathbb{R}\) except for the odd integers. So \(\text{Dom}(f') = \mathbb{R}\setminus\{2k+1:k\in\mathbb{Z}\}\). For \(x\in\text{Dom}(f')\), \(f'(x)=0\).

\sphinxAtStartPar
(iv) Differentiating the formal series
\begin{equation*}
\begin{split}
\frac{4}{\pi}\left[\cos\left(\frac{\pi x}{2}\right)-\frac{1}{3}\cos\left(\frac{3\pi x}{2}\right)+\frac{1}{5}\cos\left(\frac{5\pi x}{2}\right)-\frac{1}{7}\cos\left(\frac{7\pi x}{2}\right)+\ldots\right]
\end{split}
\end{equation*}
\sphinxAtStartPar
term\sphinxhyphen{}by\sphinxhyphen{}term yields
\begin{equation*}
\begin{split}
-2\left[\sin\left(\frac{\pi x}{2}\right)-\sin\left(\frac{3\pi x}{2}\right)+\sin\left(\frac{5\pi x}{2}\right)-\sin\left(\frac{7\pi x}{2}\right)+\ldots\right].
\end{split}
\end{equation*}
\sphinxAtStartPar
When \(x\) is an even integer, this series converges and equals \(0\). A quick plug into \sphinxhref{https://www.desmos.com/calculator/req6vnpabz}{Desmos} suggests the series diverges for all other values of \(x\).

\sphinxAtStartPar
\sphinxstylestrong{Extension:} Can you prove this?

\begin{sphinxadmonition}{note}{Historical note}

\sphinxAtStartPar
In {\hyperref[\detokenize{Solutions-full:id1}]{\sphinxcrossref{Problem 65}}}, we found that at all points aside from the odd integers, the function
\begin{equation*}
\begin{split}
f(x)=
\frac{4}{\pi}\left[\cos\left(\frac{\pi x}{2}\right)-\frac{1}{3}\cos\left(\frac{3\pi x}{2}\right)+\frac{1}{5}\cos\left(\frac{5\pi x}{2}\right)-\frac{1}{7}\cos\left(\frac{7\pi x}{2}\right)+\ldots\right]
\end{split}
\end{equation*}
\sphinxAtStartPar
is differentiable with derivative equal to \(0\). However, differentiating term\sphinxhyphen{}by\sphinxhyphen{}term yields a series that diverges for all values of \(x\) except the even integers.

\sphinxAtStartPar
It was apparent contradictions such as this that created such a stir among the contemporaries of Fourier, and meant his 1807 paper was never accepted for publication. As modern mathematicians, we have access to subtle ideas like uniform and absolute convergence (not to mention a functioning definition of differentiability) that allow us to make sense of the strange behaviour exhibited by this series. Mathematicians of the early 19th century could not reconcile the obvious usefulness of the method Fourier had showcased with everything they knew or believed to be true about functions, series and derivatives.

\sphinxAtStartPar
Following these events was a huge collective effort of the whole mathematical community to properly understand and develop rigorous foundations for calculus — or as it came to be known, real analysis.

\sphinxAtStartPar
\sphinxstylestrong{Disclaimer:} I am not a historian. For more on this fascinating subject, I recommend reading Chapter 1 of \sphinxhref{https://find.shef.ac.uk/permalink/f/1lephdb/44SFD\_ALMA\_DS21193257230001441}{A Radical Approach to Real Analysis by Bressoud}.
\end{sphinxadmonition}


\subsection{Integration}
\label{\detokenize{Solutions-full:integration}}
\sphinxAtStartPar
{\hyperref[\detokenize{Problems:id66}]{\sphinxcrossref{\DUrole{std,std-ref}{66.}}}} Let \(f:[1,4]\to\mathbb{R}\); \(f(x)=\frac{1}{x}\), and let \(P=\{1,\frac{3}{2},2,4\}\).

\sphinxAtStartPar
(i) Here is a sketch of the graph of \(f\), the partition \(P\), and the associated upper and lower Riemann sums.

\begin{figure}[htbp]
\centering
\capstart

\noindent\sphinxincludegraphics[width=600\sphinxpxdimen]{{int1i}.png}
\caption{\(y=\frac{1}{x}\), with upper and lower sums.}\label{\detokenize{Solutions-full:q64i}}\end{figure}

\sphinxAtStartPar
We have \(U(f,P) = \sum_{k=1}^4M_k(x_k-x_{k-1})\), where \(x_0=1\), \(x_1=\frac{3}{2}\), \(x_2=2\), \(x_3=4\), and
\begin{equation*}
\begin{split}
M_1=\sup\left\{\frac{1}{x}:1\leq x\leq\frac{3}{2}\right\}=1, \hspace{1em} M_2=\sup\left\{\frac{1}{x}:\frac{3}{2}\leq x\leq 2\right\}=\frac{2}{3},
\end{split}
\end{equation*}\begin{equation*}
\begin{split}
M_3=\sup\left\{\frac{1}{x}:2\leq x\leq 4\right\}=\frac{1}{2}.
\end{split}
\end{equation*}
\sphinxAtStartPar
Therefore,
\begin{align*}
U(f,P) &= 1\cdot\left(\frac{3}{2}-1\right) + \frac{2}{3}\cdot\left(2-\frac{3}{2}\right) + \frac{1}{2}\cdot(4-2) \\
&= \frac{1}{2} + \frac{1}{3} + 1 = \frac{11}{6}.
\end{align*}
\sphinxAtStartPar
Similarly, \(L(f,P) = \sum_{k=1}^4m_k(x_k-x_{k-1})\), where
\begin{equation*}
\begin{split}
m_1=\inf\left\{\frac{1}{x}:1\leq x\leq\frac{3}{2}\right\}=\frac{2}{3}, \hspace{1em} m_2=\inf\left\{\frac{1}{x}:\frac{3}{2}\leq x\leq 2\right\}=\frac{1}{2},
\end{split}
\end{equation*}
\sphinxAtStartPar
and
\begin{equation*}
\begin{split}
m_3=\inf\left\{\frac{1}{x}:2\leq x\leq 4\right\}=\frac{1}{4}.
\end{split}
\end{equation*}
\sphinxAtStartPar
Hence
\begin{align*}
L(f,P) = \frac{2}{3}\cdot\left(\frac{3}{2}-1\right) + \frac{1}{2}\left(2-\frac{3}{2}\right) + \frac{1}{4}(4-2) \\
&= \frac{1}{3} + \frac{1}{4} + \frac{1}{2} = \frac{13}{12},
\end{align*}
\sphinxAtStartPar
and so
\begin{equation*}
\begin{split}
U(f,P)-L(f,P) = \frac{11}{6}-\frac{13}{12} = \frac{9}{12} = \frac{3}{4}.
\end{split}
\end{equation*}
\sphinxAtStartPar
(ii) Adding the point \(P\) has the following effect:

\begin{figure}[htbp]
\centering
\capstart

\noindent\sphinxincludegraphics[width=600\sphinxpxdimen]{{int1ii}.png}
\caption{Upper and lower sums with additional point \(3\).}\label{\detokenize{Solutions-full:q64ii}}\end{figure}

\sphinxAtStartPar
We now have \(x_0=1\), \(x_1=\frac{3}{2}\), \(x_2=2\), \(x_3=3\) and \(x_4=4\). \(M_1\), \(M_2\), \(m_1\) and \(m_2\) are not affected by the additional point, while
\begin{equation*}
\begin{split}
M_3 = \sup\left\{\frac{1}{x}:2\leq x\leq 3\right\}=\frac{1}{2},\hspace{1em} M_4 =\sup\left\{\frac{1}{x}:3\leq x\leq 4\right\}=\frac{1}{3},
\end{split}
\end{equation*}\begin{equation*}
\begin{split}
m_3 = \inf\left\{\frac{1}{x}:2\leq x\leq 3\right\}=\frac{1}{3}, \hspace{1em} \text{and} \hspace{1em} m_4 =\inf\left\{\frac{1}{x}:3\leq x\leq 4\right\}=\frac{1}{4}.
\end{split}
\end{equation*}
\sphinxAtStartPar
Therefore
\begin{equation*}
\begin{split}
U(f,P) = 1\left(\frac{1}{2}\right) + \frac{2}{3}\left(\frac{1}{2}\right) + \frac{1}{2}(1) + \frac{1}{3}(1) = \frac{1}{2}+\frac{1}{3}+\frac{1}{2}+\frac{1}{3} = \frac{5}{3},
\end{split}
\end{equation*}\begin{equation*}
\begin{split}
L(f,P) = \frac{2}{3}\left(\frac{1}{2}\right) + \frac{1}{2}\left(\frac{1}{2}\right) + \frac{1}{3}(1) + \frac{1}{4}(1) = \frac{1}{3}+\frac{1}{4}+\frac{1}{3}+\frac{1}{4} = \frac{7}{6},
\end{split}
\end{equation*}
\sphinxAtStartPar
and
\begin{equation*}
\begin{split}
U(f,P) - L(f,P) = \frac{5}{3}-\frac{7}{6} = \frac{1}{2}.
\end{split}
\end{equation*}
\sphinxAtStartPar
So the size of \(U(f,P)-L(f,P)\) has decreased after adding the point \(3\) to \(P\).

\sphinxAtStartPar
(iii) Mimicking the proof of Theorem 5.2.5, let \(x_k=1+\frac{3}{n}k\) for \(k=0,\ldots,n\). Then, \(P=\{x_0,\ldots,x_n\}\) partitions \([1,3]\) uniformly into \(n\) intervals, each of width \(\frac{3}{n}\). Since \(\frac{1}{x}\) is decreasing,
\begin{equation*}
\begin{split}
M_k = \sup\left\{\frac{1}{x}:x\in[x_{k-1},x_k]\right\} =\frac{1}{x_{k-1}}
\end{split}
\end{equation*}
\sphinxAtStartPar
and
\begin{equation*}
\begin{split}
m_k = \inf\left\{\frac{1}{x}:x\in[x_{k-1},x_k]\right\} =\frac{1}{x_k}
\end{split}
\end{equation*}
\sphinxAtStartPar
for each \(k\). So,
\begin{align*}
U(f,P) - L(f,P) &= \sum_{k=1}^n\left(\frac{1}{x_k} - \frac{1}{x_{k-1}}\right)\frac{3}{n} \\
&= \frac{3}{n}\left(\frac{1}{x_n}-\frac{1}{x_0}\right) = \frac{3}{n}\left(1-\frac{1}{4}\right) = \frac{9}{4n}.
\end{align*}
\sphinxAtStartPar
We have been asked to find \(P\) so that \(U(f,P)-L(f,P)<\frac{2}{5}\), so we choose \(n\) so that \(\frac{9}{4n}<\frac{2}{5}\). That is, \(n>\frac{45}{8}\). In fact, \(n=6\) will do.

\sphinxAtStartPar
So \(x_k=1+\frac{1}{2}k\) for \(k=0,\ldots,6\), and \(P=\left\{1,\frac{3}{2},2,\frac{5}{2},3,\frac{7}{2},4\right\}\).

\begin{figure}[htbp]
\centering
\capstart

\noindent\sphinxincludegraphics[width=600\sphinxpxdimen]{{int1iii}.png}
\caption{A partition for which \(U(f,P)-L(f,P)<\frac{2}{5}\).}\label{\detokenize{Solutions-full:q64iii}}\end{figure}


\bigskip\hrule\bigskip


\sphinxAtStartPar
{\hyperref[\detokenize{Problems:id67}]{\sphinxcrossref{\DUrole{std,std-ref}{67.}}}} Let \(g:[0,1]\to\mathbb{R}\); \(\displaystyle g(x)=\left\{\begin{array}{cc} 1 & \text{for } 0\leq x<1 \\ 2 &\text{for } x=1 \end{array}\right.\).

\sphinxAtStartPar
(i) Let \(P=\{x_0,x_1,\ldots x_n\}\) be any partition \([0,1]\), with \(0=x_0<x_1<\ldots<x_n=1\). Then for \(k=1,\ldots n\),
\begin{equation*}
\begin{split}
m_k=\inf\{g(x) : x\in[x_{k-1},x_k\} = 1.
\end{split}
\end{equation*}
\sphinxAtStartPar
Therefore
\begin{equation*}
\begin{split}
L(g,P) = \sum_{k=1}^n(x_k-x_{k-1}) = x_n-x_0 = 1.
\end{split}
\end{equation*}
\sphinxAtStartPar
(ii) For \(k\neq n\), \(1\notin[x_{k-1},x_k]\) and so
\begin{equation*}
\begin{split}
M_k=\sup\{g(x) : x\in[x_{k-1},x_k\} = 1.
\end{split}
\end{equation*}
\sphinxAtStartPar
On the other hand,
\begin{equation*}
\begin{split}
M_n=\sup\{g(x) : x\in[x_{n-1},1\} = 2.
\end{split}
\end{equation*}
\sphinxAtStartPar
Therefore
\begin{equation*}
\begin{split}
U(g,P) = (x_1-x_0) + (x_2-x_1) + \ldots (x_{n-1},x_{n-2}) + 2(x_n-x_{n-1}) = 2x_n-x_{n-1}-x_0 = 2-x_{n-1}.
\end{split}
\end{equation*}
\sphinxAtStartPar
Let \(x_k=\frac{k}{n}\) for \(k=0,\ldots,n\), where \(n\in\mathbb{N}\) is yet to be chosen. Then
\begin{equation*}
\begin{split}
U(g,P) = 2-\frac{n-1}{n} = 1+\frac{1}{n}.
\end{split}
\end{equation*}
\sphinxAtStartPar
Taking \(n=10\) will give a partition \(P\) so that \(U(g,P)=1+\frac{1}{10}\).

\sphinxAtStartPar
(iii) Now let \(\varepsilon>0\), and let \(n\) be any integer strictly greater than \(\frac{1}{\varepsilon}\). The partition described in (ii) then satisfies
\begin{equation*}
\begin{split}
U(g,P) = 1+\frac{1}{n} < 1+\varepsilon.
\end{split}
\end{equation*}

\bigskip\hrule\bigskip


\sphinxAtStartPar
{\hyperref[\detokenize{Problems:id68}]{\sphinxcrossref{\DUrole{std,std-ref}{68.}}}}
(i) By Proposition 5.3.1(ii) ,
\$\(
\int_0^3 r(t)\, dt = \int_0^11dt+\int_1^2edt+\int_2^3e^4dt = 1+e+e^4,
\)\(
and similarly
\)\(
\int_0^3 s(t)\, dt = \int_0^1edt+\int_1^2e^4dt+\int_2^3e^9dt = e+e^4+e^9.
\)\$

\sphinxAtStartPar
(ii)

\sphinxAtStartPar
(iii) Note that \(r(t)\leq f(t)\leq s(t)\) for all \(t\), and so by Proposition 5.3.1(i),
\begin{equation*}
\begin{split}
1+e+e^4  = \int_0^3 r(t)\, dt \leq \int_0^3 f(t)\, dt \leq \int_0^3 s(t)\, dt = e+e^4+e^9.
\end{split}
\end{equation*}

\bigskip\hrule\bigskip


\sphinxAtStartPar
{\hyperref[\detokenize{Problems:id69}]{\sphinxcrossref{\DUrole{std,std-ref}{69.}}}} Let \(f,g:[a,b]\to\mathbb{R}\) be integrable functions.

\sphinxAtStartPar
(i) For any subset \(A\subset[a,b]\),
\begin{equation*}
\begin{split}
\sup\{f(x)+g(x):x\in A\} \leq \sup\{f(x):x\in A\} + \sup\{g(x):x\in A\},
\end{split}
\end{equation*}
\sphinxAtStartPar
and similarly
\begin{equation*}
\begin{split}
\inf\{f(x)+g(x):x\in A\} \geq \inf\{f(x):x\in A\} + \inf\{g(x):x\in A\}.
\end{split}
\end{equation*}
\sphinxAtStartPar
It is then immediate from the definition of the upper and lower sums that
\begin{equation*}
\begin{split}
U(f+g,P) \leq U(f,P)+U(g,P)
\end{split}
\end{equation*}
\sphinxAtStartPar
and
\begin{equation*}
\begin{split}
L(f+g,P) \geq L(f,P)+L(g,P).
\end{split}
\end{equation*}
\sphinxAtStartPar
For the example where this is strict, a good strategy is to look for functions with features that ``cancel out’’ in some sense. For example, you could take
\begin{equation*}
\begin{split}
f(x) = \left\{\begin{array}{cc} 1 & \text{ if } 0\leq x\leq 1 \\  -1 & \text{ if } 1\leq x\leq 2\end{array}\right.
\end{split}
\end{equation*}
\sphinxAtStartPar
and
\begin{equation*}
\begin{split}
g(x) = \left\{\begin{array}{cc} -1 & \text{ if } 0\leq x\leq 1 \\  1 & \text{ if } 1\leq x\leq 2.\end{array}\right.
\end{split}
\end{equation*}
\sphinxAtStartPar
Then \(f+g\) is the zero function, so \(L(f+g,P)=U(f+g,P)=0\). On the other hand
\begin{equation*}
\begin{split}
U(f,P)+U(g,P) = 1+1 = 2,
\end{split}
\end{equation*}
\sphinxAtStartPar
and
\begin{equation*}
\begin{split}
L(f,P)+L(g,P) = -1 -1 =-2.
\end{split}
\end{equation*}
\sphinxAtStartPar
(ii) Let \(P_1\) and \(P_2\) be partitions of \([a,b]\). Then
\begin{align*}
U(f+g) &\leq U(f+g,P_1\cup P_2) \\
&\leq U(f,P_1\cup P_2) + U(g,P_1\cup P_2) \\
&\leq U(f,P_1) + U(g,P_2),
\end{align*}
\sphinxAtStartPar
since \(P_1\cup P_2\) is a refinement of both partitions \(P_1\) and \(P_2\).

\sphinxAtStartPar
The proof of the lower sum statement is similar.

\sphinxAtStartPar
(iii) Taking \(\inf\) over \(P_1\),
\begin{equation*}
\begin{split}
U(f+g) \leq U(f) + U(g,P_2),
\end{split}
\end{equation*}
\sphinxAtStartPar
and then taking \(\inf\) over \(P_2\),
\begin{equation*}
\begin{split}
U(f+g) \leq U(f)+U(g).
\end{split}
\end{equation*}
\sphinxAtStartPar
Since \(f\) and \(g\) are integrable, this shows that
\begin{equation*}
\begin{split}
U(f+g) \leq \int_a^bf(x)dx + \int_a^bg(x)dx.
\end{split}
\end{equation*}
\sphinxAtStartPar
A similar argument shows that
\begin{equation*}
\begin{split}
L(f+g) \geq \int_a^bf(x)dx + \int_a^bg(x)dx.
\end{split}
\end{equation*}
\sphinxAtStartPar
Hence
\begin{equation*}
\begin{split}
\int_a^bf(x)dx + \int_a^bg(x)dx \leq L(f+g) \leq U(f+g) \leq \int_a^bf(x)dx + \int_a^bg(x)dx.
\end{split}
\end{equation*}
\sphinxAtStartPar
The only option is
\begin{equation*}
\begin{split}
L(f+g)=U(f+g)=\int_a^bf(x)dx + \int_a^bg(x)dx.
\end{split}
\end{equation*}
\sphinxAtStartPar
That is, \(f+g\) is integrable and
\begin{equation*}
\begin{split}
\int_a^b(f(t)+g(t))dt = \int_a^bf(x)dx + \int_a^bg(x)dx.
\end{split}
\end{equation*}

\bigskip\hrule\bigskip


\sphinxAtStartPar
{\hyperref[\detokenize{Problems:id70}]{\sphinxcrossref{\DUrole{std,std-ref}{70.}}}} Let \(P\) be the partition \(a=x_0<x_1<\ldots<x_n=b\). If \(k\geq 0\), then
\begin{equation*}
\begin{split}
\sup_{x\in[x_{k-1},x_k]}kf(x) = k\sup_{x\in[x_{k-1},x_k]}f(x)
\end{split}
\end{equation*}
\sphinxAtStartPar
and
\begin{equation*}
\begin{split}
\inf_{x\in[x_{k-1},x_k]}kf(x) = k\inf_{x\in[x_{k-1},x_k]}f(x).
\end{split}
\end{equation*}
\sphinxAtStartPar
Therefore \(U(kf,P)=kU(f,P)\) and \(L(kf,P)=kL(f,P)\), and taking infima/suprema, \(U(kf)=kU(f)\) and \(L(kf)=kL(f)\).

\sphinxAtStartPar
Since \(f\) is integrable, \(U(f)=L(f)=\int_a^bf(x)dx\), and hence
\begin{equation*}
\begin{split}
U(kf)=kU(f)=kL(f)=L(kf)=k\int_a^bf(x)dx.
\end{split}
\end{equation*}
\sphinxAtStartPar
It follows that \(kf\) is integrable, with \(\displaystyle\int_a^bkf(x)dx=k\int_a^bf(x)dx\).

\sphinxAtStartPar
To extend this result to include negative \(k\), it is sufficient to prove the \(k=-1\) case. Note that
\begin{equation*}
\begin{split}
\sup_{x\in[x_{k-1},x_k]}\big(-f(x)\big) = -\inf_{x\in[x_{k-1},x_k]}f(x),
\end{split}
\end{equation*}
\sphinxAtStartPar
and
\begin{equation*}
\begin{split}
\inf_{x\in[x_{k-1},x_k]}\big(-f(x)\big) = -\sup_{x\in[x_{k-1},x_k]}f(x).
\end{split}
\end{equation*}
\sphinxAtStartPar
Therefore, \(U(-f,P)=-L(f,P)\) and \(L(-f,P)=-U(f,P)\), and so
\begin{align*}
U(-f) &= \inf\{U(-f,P):\text{ partitions } P \text{ of } [a,b]\} \\
&= \inf\{-L(f,P):\text{ partitions } P \text{ of } [a,b]\} \\
&= -\sup\{L(f,P):\text{ partitions } P \text{ of } [a,b]\} = -L(f).
\end{align*}
\sphinxAtStartPar
Similarly,
\begin{align*}
L(-f) &= \sup\{L(-f,P):\text{ partitions } P \text{ of } [a,b]\} \\
&= \sup\{-U(f,P):\text{ partitions } P \text{ of } [a,b]\} \\
&= -\inf\{U(f,P):\text{ partitions } P \text{ of } [a,b]\} = -U(f).
\end{align*}
\sphinxAtStartPar
Therefore \(U(-f) = -L(f) = -U(f) = L(-f)\), so \(-f\) is integrable, and
\begin{equation*}
\begin{split}
\int_a^b(-f(x))dx = -\int_a^b f(x)dx.
\end{split}
\end{equation*}

\bigskip\hrule\bigskip


\sphinxAtStartPar
{\hyperref[\detokenize{Problems:id71}]{\sphinxcrossref{\DUrole{std,std-ref}{71.}}}} We have \(M=\sup\{f(x):x\in A\}, \; m=\inf\{f(x):x\in A\}\), and \(M'=\sup\{|f(x)|:x\in A\}, \; \text{ and } \; m'=\inf\{|f(x)|:x\in A\}\).

\sphinxAtStartPar
(i) If \(m\geq 0\), then \(f\) is a non\sphinxhyphen{}negative function, so \(M=M'\), \(m=m'\), and \(M-m=M'-m'\).

\sphinxAtStartPar
If \(m\leq M\leq 0\) then \(M'=-m\), \(m'=-M\), and so again \(M'-m'=M-m\).

\sphinxAtStartPar
If \(m<0<M\), then \(M'=\max\{M,-m\}\leq M+(-m)\), since both \(M\) and \(-m\) are positive numbers. Then,
\$\(
M'-m' \leq M' \leq M-m,
\)\(
since \)m’\textbackslash{}geq 0\$.

\sphinxAtStartPar
(ii) Let \(f:[a,b]\to\mathbb{R}\) be integrable, and let \(P\) be the partition \(a=x_0<x_1<\ldots<x_n=b\). for \(k=1,\ldots,n\). let
\begin{equation*}
\begin{split}
M_k=\sup_{x\in[x_{k-1},x_k]}f(x), \; m_k=\inf{x\in[x_{k-1},x_k]}f(x),
\end{split}
\end{equation*}\begin{equation*}
\begin{split}
M_k'=\sup_{x\in[x_{k-1},x_k]}|f(x)|, \; \text{ and } \; m_k'=\inf{x\in[x_{k-1},x_k]}|f(x)|.
\end{split}
\end{equation*}
\sphinxAtStartPar
Then by part (i), \(M_k-m_k\geq M_k'-m_k'\) for each \(k\).

\sphinxAtStartPar
Therefore,
\begin{equation*}
\begin{split}
U(|f|,P) - L(|f|,P) = \sum_{k=1}^n(M_k'-m_k')(x_k-x_{k-1}) \leq \sum_{k=1}^n(M_k-m_k)(x_k-x_{k-1}) = U(f,P)-L(f,P).
\end{split}
\end{equation*}
\sphinxAtStartPar
(iii) Since \(f\) is integrable, for all \(\varepsilon>0\), we can choose the a partition \(P\) so that \(U(f,P)-L(f,P)<\varepsilon\). Using the result of part (ii). it follows that \(U(|f|,P)-L(|f|,P)<\varepsilon\) for this partition, and so \(|f|\) is integrable.

\sphinxAtStartPar
Since \(-|f(x)|\leq f(x)\leq |f(x)|\) for all \(x\in[a,b]\), Proposition 6.3(iii) imply that
\begin{equation*}
\begin{split}
-\int_a^b|f(x)|dx \leq \int_a^b f(x)dx \leq \int_a^b|f(x)|dx.
\end{split}
\end{equation*}
\sphinxAtStartPar
That is,
\begin{equation*}
\begin{split}
\left|\int_a^b f(x)dx\right| \leq \int_a^b|f(x)|dx.
\end{split}
\end{equation*}

\bigskip\hrule\bigskip


\sphinxAtStartPar
{\hyperref[\detokenize{Problems:id72}]{\sphinxcrossref{\DUrole{std,std-ref}{72.}}}} We have
\begin{equation*}
\begin{split}
\int_{a(x)}^{b(x)} f(t)dt =  \int_0^{b(x)} f(t)dt - \int_0^{a(x)} f(t)\ dt.
\end{split}
\end{equation*}
\sphinxAtStartPar
Write \(F(x)=\int_0^x f(t)dt.\) Then by the fundamental theorem of calculus, \( F'(t)=f(t)\). Then
\begin{equation*}
\begin{split}
\int_0^{b(x)} f(t)dt=F(b(x)),
\end{split}
\end{equation*}
\sphinxAtStartPar
and using the chain rule
\begin{equation*}
\begin{split}
\frac{d}{dx}\int_0^{b(x)} f(t)dt =F'(b(t))b'(t)=f(b(t))b'(t).
\end{split}
\end{equation*}
\sphinxAtStartPar
Similarly,
\begin{equation*}
\begin{split}
\frac{d}{dx}\int_0^{a(x)} f(y)dt =f(a(x))a'(x).
\end{split}
\end{equation*}
\sphinxAtStartPar
Subtracting these
\begin{equation*}
\begin{split}
\frac{d}{dx}\int_{a(x)}^{b(x)} f(t)dt =f(b(x))b'(x) - f(a(x))a'(x),
\end{split}
\end{equation*}
\sphinxAtStartPar
as required.


\bigskip\hrule\bigskip


\sphinxAtStartPar
{\hyperref[\detokenize{Problems:id73}]{\sphinxcrossref{\DUrole{std,std-ref}{73.}}}}
(i) It follows immediately from the fundamental theorem of calculus that \(l\) is differentiable, with \(l'(x)=\frac{1}{x}\).

\sphinxAtStartPar
(ii) By Proposition 6.2 in the lecture notes,
\begin{equation*}
\begin{split}
l(xy) =\int_1^{xy}\frac{1}{t}\, dt= \int_1^x \frac{1}{t}\, dt + \int_x^{xy} \frac{1}{t}\, dt = l(y) + \int_x^{xy}\frac{1}{t}dt.
\end{split}
\end{equation*}
\sphinxAtStartPar
In the second integral, set \(a(x)=xy\), \(b(x)=x\) and use {\hyperref[\detokenize{Problems:id71}]{\sphinxcrossref{\DUrole{std,std-ref}{Problem 71}}}}. We have \(a'(x)=y\) and \(b'(x)=1\), so
\begin{equation*}
\begin{split}
\frac{d}{dx}\int_x^{xy} \frac{1}{t}dt = \frac{1}{b(x)}b'(x) - \frac{1}{a(x)}a'(x) = \frac{1}{x} - \frac{y}{xy} =0.
\end{split}
\end{equation*}
\sphinxAtStartPar
Therefore \(\int_x^{xy}\frac{1}{t}dt\) is constant as a function of \(x\). In particular, it coincides with its value when \(x=1\):
\begin{equation*}
\begin{split}
\int_x^{xy}\frac{1}{t}dt = \int_1^{y}\frac{1}{t}dt = l(y),
\end{split}
\end{equation*}
\sphinxAtStartPar
and \(l(xy)=l(x)+l(y)\) follows.


\bigskip\hrule\bigskip


\sphinxAtStartPar
{\hyperref[\detokenize{Problems:id74}]{\sphinxcrossref{\DUrole{std,std-ref}{74.}}}} Define \(G\colon [a,b]\rightarrow \mathbb{R}\) by
\begin{equation*}
\begin{split}
G(x) = \int_a^x f(t)\, dt
\end{split}
\end{equation*}
\sphinxAtStartPar
Then \(G\) is a continuous function. This was mentioned in the lectures. To see why, let \(h>0\). Since the function \(f\) is Riemann integrable, it is bounded, so we have \(M\in \mathbb{R}\) such that \(|f(t)|\leq M\) for all \(t\in [a,b]\). Then
\begin{equation*}
\begin{split}
|G(x+h) - G(x) | =\left| \int_x^{x+h} f(t)\, dt \right| \leq \int_x^{x+h} |f(t)|\, dt \leq Mh.
\end{split}
\end{equation*}
\sphinxAtStartPar
Similarly \(|G(x-h) - G(x) | \leq Mh\).

\sphinxAtStartPar
So \(G(x+h)\rightarrow G(x)\) as \(h\rightarrow 0\), and \(G\) is continuous.

\sphinxAtStartPar
Thus we can define a continuous function \(F\colon [a,b]\rightarrow \mathbb{R}\) by
\begin{equation*}
\begin{split}
F(x) = \int_a^x f(t)\, dt - \int_x^b f(t)\, dt.
\end{split}
\end{equation*}
\sphinxAtStartPar
Observe that
\begin{equation*}
\begin{split}
F(a) = -\int_a^b f(t)\, dt,  \qquad F(b) = \int_a^b f(t)\, dt=-F(a).
\end{split}
\end{equation*}
\sphinxAtStartPar
If \(\int_a^b f(t)\, dt =0\), then we can take \(x=a\) and we have the required result. Otherwise, either \(F(a)<0\) and \(F(b)>0\) or \(F(a)>0\) and \(F(b)<0\).

\sphinxAtStartPar
Either way, by the intermediate value theorem, we have \(x\in [a,b]\) such that \(F(x)=0\), that is to say
\begin{equation*}
\begin{split}
\int_a^x f(t)\, dt = \int_x^b f(t)\, dt.
\end{split}
\end{equation*}

\bigskip\hrule\bigskip


\sphinxAtStartPar
{\hyperref[\detokenize{Problems:id75}]{\sphinxcrossref{\DUrole{std,std-ref}{75.}}}}
(i) Consider the step functions \(m\mathbf{1}_{[a,b]}\) and \(M\mathbf{1}_{[a,b]}\) with \(m\mathbf{1}_{[a,b]}(x)\leq f(x)\leq M\mathbf{1}_{[a,b]}(x)\) for all \(x\in[a,b]\). Thus we have
\begin{equation*}
\begin{split}
m(b-a)=\int_a^b m\mathbf{1}_{[a,b]}(x)\, dx \leq \int_a^b f(x)\, dx \leq \int_a^b M\mathbf{1}_{[a,b]}(x)\, dx= M(b-a).
\end{split}
\end{equation*}
\sphinxAtStartPar
Let \(\mu = \frac{1}{b-a} \int_a^b f(x)\, dx\). Then \(\int_a^b f(x)\, dx=(b-a)\mu\) and \(m\leq \mu\leq M\), so \(\mu \in [m,M]\).

\sphinxAtStartPar
(ii) Let \(m = \inf \{ f(x) : x\in [a,b] \}\) and \(M = \sup \{ f(x) : x\in [a,b] \}\). Then by part (i) we have \(\mu \in [m,M]\) where
\begin{equation*}
\begin{split}
\int_a^b f(x)\, dx =(b-a)\mu.
\end{split}
\end{equation*}
\sphinxAtStartPar
But as \(f\) is a continuous function on a closed bounded interval, \(m\) is the minimum value of \(f\), and \(M\) is the maximum value. As \(\mu\in [m,M]\) and \(f\) is continuous, by the intermediate value theorem there is some \(\xi \in [a,b]\) such that \(f(\xi ) = \mu\). Hence
\begin{equation*}
\begin{split}
\int_a^b f(x)\, dx =(b-a)f(\xi ).
\end{split}
\end{equation*}

\bigskip\hrule\bigskip


\sphinxAtStartPar
{\hyperref[\detokenize{Problems:id76}]{\sphinxcrossref{\DUrole{std,std-ref}{76.}}}} Let  \(m = \inf \{ f(x) : x\in [a,b] \}\) and \(M = \sup \{ f(x) : x\in [a,b] \}\). Then \(m\leq f(x)\leq M\) for all \(x\in [a,b]\). As \(g(x)\geq 0\), we have
\begin{equation*}
\begin{split}
mg(x)\leq f(x)g(x)\leq Mg(x)
\end{split}
\end{equation*}
\sphinxAtStartPar
for all \(x\in [a,b]\), and so
\begin{equation*}
\begin{split}
m \int_a^b g(x)\, dx \leq \int_a^b f(x)g(x)\, dx \leq M\int_a^b g(x)\, dx.
\end{split}
\end{equation*}
\sphinxAtStartPar
Let
\begin{equation*}
\begin{split}
I= \int_a^b g(x)\, dx.
\end{split}
\end{equation*}
\sphinxAtStartPar
If \(I=0\), then the above tells us that
\begin{equation*}
\begin{split}
\int_a^b f(x)g(x)\, dx =0
\end{split}
\end{equation*}
\sphinxAtStartPar
and any \(\xi \in [a,b]\) satisfies
\begin{equation*}
\begin{split}
\int_a^b f(x)g(x)\, dx = f(\xi ) \int_a^b g(x)\, dx .
\end{split}
\end{equation*}
\sphinxAtStartPar
As \(g(x)\geq 0\), we need to prove the result when \(I>0\). Then
\begin{equation*}
\begin{split}
m \leq \frac{1}{I} \int_a^b f(x)g(x)\, dx \leq M.
\end{split}
\end{equation*}
\sphinxAtStartPar
By the intermediate value theorem, as in the solution to Q121(ii), we have \(\xi \in [m,M]\) such that
\begin{equation*}
\begin{split}
f(\xi ) = \frac{1}{I}  \int_a^b f(x)g(x)\, dx
\end{split}
\end{equation*}
\sphinxAtStartPar
or in other words
\begin{equation*}
\begin{split}
\int_a^b f(x)g(x)\, dx = f(\xi ) \int_a^b g(x)\, dx .
\end{split}
\end{equation*}
\sphinxAtStartPar
We do need the assumption \(g(x)\geq 0\). To see this, consider the function \(g(x) = x\) for \(x\in [-1,1]\). Then
\begin{equation*}
\begin{split}
\int_{-1}^1 g(x)\, dx =0,
\end{split}
\end{equation*}
\sphinxAtStartPar
so the result would say that for any continuous function \(f\colon [-1,1]\rightarrow \mathbb{R}\) we have
\begin{equation*}
\begin{split}
\int_{-1}^1 xf(x)\, dx =0.
\end{split}
\end{equation*}
\sphinxAtStartPar
But this is clearly not the case, taking for example \(f(x)=x\).







\renewcommand{\indexname}{Index}
\printindex
\end{document}