%% Generated by Sphinx.
\def\sphinxdocclass{jupyterBook}
\documentclass[letterpaper,10pt,english]{jupyterBook}
\ifdefined\pdfpxdimen
   \let\sphinxpxdimen\pdfpxdimen\else\newdimen\sphinxpxdimen
\fi \sphinxpxdimen=.75bp\relax
\ifdefined\pdfimageresolution
    \pdfimageresolution= \numexpr \dimexpr1in\relax/\sphinxpxdimen\relax
\fi
%% let collapsible pdf bookmarks panel have high depth per default
\PassOptionsToPackage{bookmarksdepth=5}{hyperref}
%% turn off hyperref patch of \index as sphinx.xdy xindy module takes care of
%% suitable \hyperpage mark-up, working around hyperref-xindy incompatibility
\PassOptionsToPackage{hyperindex=false}{hyperref}
%% memoir class requires extra handling
\makeatletter\@ifclassloaded{memoir}
{\ifdefined\memhyperindexfalse\memhyperindexfalse\fi}{}\makeatother

\PassOptionsToPackage{booktabs}{sphinx}
\PassOptionsToPackage{colorrows}{sphinx}

\PassOptionsToPackage{warn}{textcomp}

\catcode`^^^^00a0\active\protected\def^^^^00a0{\leavevmode\nobreak\ }
\usepackage{cmap}
\usepackage{fontspec}
\defaultfontfeatures[\rmfamily,\sffamily,\ttfamily]{}
\usepackage{amsmath,amssymb,amstext}
\usepackage{polyglossia}
\setmainlanguage{english}



\setmainfont{FreeSerif}[
  Extension      = .otf,
  UprightFont    = *,
  ItalicFont     = *Italic,
  BoldFont       = *Bold,
  BoldItalicFont = *BoldItalic
]
\setsansfont{FreeSans}[
  Extension      = .otf,
  UprightFont    = *,
  ItalicFont     = *Oblique,
  BoldFont       = *Bold,
  BoldItalicFont = *BoldOblique,
]
\setmonofont{FreeMono}[
  Extension      = .otf,
  UprightFont    = *,
  ItalicFont     = *Oblique,
  BoldFont       = *Bold,
  BoldItalicFont = *BoldOblique,
]



\usepackage[Bjarne]{fncychap}
\usepackage[,numfigreset=2,mathnumfig]{sphinx}

\fvset{fontsize=\small}
\usepackage{geometry}


% Include hyperref last.
\usepackage{hyperref}
% Fix anchor placement for figures with captions.
\usepackage{hypcap}% it must be loaded after hyperref.
% Set up styles of URL: it should be placed after hyperref.
\urlstyle{same}


\usepackage{sphinxmessages}



        % Start of preamble defined in sphinx-jupyterbook-latex %
         \usepackage[Latin,Greek]{ucharclasses}
        \usepackage{unicode-math}
        % fixing title of the toc
        \addto\captionsenglish{\renewcommand{\contentsname}{Contents}}
        \hypersetup{
            pdfencoding=auto,
            psdextra
        }
        % End of preamble defined in sphinx-jupyterbook-latex %
        

\title{MAS2004/9 Semester 2 Problems}
\date{Mar 24, 2025}
\release{}
\author{Rosie Shewell Brockway}
\newcommand{\sphinxlogo}{\vbox{}}
\renewcommand{\releasename}{}
\makeindex
\begin{document}

\pagestyle{empty}
\sphinxmaketitle
\pagestyle{plain}
\sphinxtableofcontents
\pagestyle{normal}
\phantomsection\label{\detokenize{intro::doc}}


\sphinxAtStartPar
Maths is best learned by doing many problems.

\sphinxAtStartPar
This problem booklet serves 3 purposes:

\phantomsection\label{\detokenize{intro:pow}}\begin{enumerate}
\sphinxsetlistlabels{\arabic}{enumi}{enumii}{}{.}%
\item {} 
\sphinxAtStartPar
\sphinxstylestrong{Problems of the Week.} 
Each week, I will circulate a selection of problems from this booklet that relate most closely to the material covered in lectures that week. I recommend working on these steadily throughout the semester, and keeping a record of any you would like \sphinxhref{https://rosiesb.github.io/Analysis-Notes/0Intro.html\#where-to-get-help}{help} with.

\item {} 
\sphinxAtStartPar
\sphinxstylestrong{Tutorials.} 
The problem set for each tutorial will come from this booklet.

\item {} 
\sphinxAtStartPar
\sphinxstylestrong{Homework.} 
Written homework will be a selection of problems from the problem booklet. 

\end{enumerate}

\sphinxAtStartPar
See Blackboard%
\begin{footnote}[1]\sphinxAtStartFootnote
Links: \sphinxhref{https://vle.shef.ac.uk/webapps/blackboard/content/listContentEditable.jsp?content\_id=\_7918618\_1\&amp;course\_id=\_119813\_1\&amp;mode=reset}{MAS2004}; \sphinxhref{https://vle.shef.ac.uk/ultra/courses/\_119818\_1/cl/outline}{MAS2009}
%
\end{footnote} for more information.



\begin{DUlineblock}{0em}
\item[] \sphinxstylestrong{\Large Table of Contents}
\end{DUlineblock}

\sphinxAtStartPar
{\hyperref[\detokenize{Problems:prob}]{\sphinxcrossref{\DUrole{std,std-ref}{1 Problems}}}} 
  {\hyperref[\detokenize{Problems:ch1prob}]{\sphinxcrossref{\DUrole{std,std-ref}{1.1 Preliminary problems}}}} 
  {\hyperref[\detokenize{Problems:ch2prob}]{\sphinxcrossref{\DUrole{std,std-ref}{1.2. Limits of functions}}}} 
  {\hyperref[\detokenize{Problems:ch3prob}]{\sphinxcrossref{\DUrole{std,std-ref}{1.3. Continuity}}}} 
  {\hyperref[\detokenize{Problems:ch4prob}]{\sphinxcrossref{\DUrole{std,std-ref}{1.4. Differentiation}}}} 
  {\hyperref[\detokenize{Problems:ch5prob}]{\sphinxcrossref{\DUrole{std,std-ref}{1.5. Sequences and series of functions}}}} 
  {\hyperref[\detokenize{Problems:ch6prob}]{\sphinxcrossref{\DUrole{std,std-ref}{1.6. Integration}}}}

\sphinxAtStartPar
{\hyperref[\detokenize{Solutions-upto26:sol}]{\sphinxcrossref{\DUrole{std,std-ref}{2 Solutions up to 26}}}} (more to appear in due course) 
  {\hyperref[\detokenize{Solutions-upto26:ch1sol}]{\sphinxcrossref{\DUrole{std,std-ref}{2.1 Preliminary problems}}}} 
  {\hyperref[\detokenize{Solutions-upto26:ch2sol}]{\sphinxcrossref{\DUrole{std,std-ref}{2.2. Limits of functions}}}} 
  {\hyperref[\detokenize{Solutions-upto26:ch3sol}]{\sphinxcrossref{\DUrole{std,std-ref}{2.3. Continuity}}}} 




\bigskip\hrule\bigskip


\sphinxstepscope


\section{Problems}
\label{\detokenize{Problems:problems}}\label{\detokenize{Problems:prob}}\label{\detokenize{Problems::doc}}

\subsection{Preliminary problems}
\label{\detokenize{Problems:preliminary-problems}}\label{\detokenize{Problems:ch1prob}}\phantomsection\label{\detokenize{Problems:p1}}
\sphinxAtStartPar
P1.
Which of the following statements is the correct definition of convergence of a real sequence \((x_n)\) to a limit \(l\in\mathbb{R}\)? How would you correct the incorrect statements?

{[}Note: There is more than one correct answer.{]}

\sphinxAtStartPar
(i) \(\varepsilon>0\) \(N\in\mathbb{N}\) then \(|x_n-l|<\varepsilon\).

\sphinxAtStartPar
(ii) \(\forall\varepsilon>0\) \(\exists N\in\mathbb{N}\) s.t.  \(|x_n-l|<\varepsilon\) \(\forall n\geq N\).

\sphinxAtStartPar
(iii) For all \(\varepsilon>0\) and \(N\in\mathbb{N}\), we have \(|x_n-l|<\varepsilon\) whenever \(n\geq N\).

\sphinxAtStartPar
(iv) \(\exists\varepsilon>0\) s.t. \(\forall N\in\mathbb{N}\), \(|x_n-l|<\varepsilon\) \(\forall n\geq N\).

\sphinxAtStartPar
(v) Given any \(\varepsilon>0\), there is \(N\in\mathbb{N}\) such that \(|x_n-l|<\varepsilon\) whenever \(n\geq N\).


\phantomsection\label{\detokenize{Problems:p2}}
\sphinxAtStartPar
P2. \sphinxstyleemphasis{(Homework 1 question).} Prove using the \((\varepsilon-N)\)\sphinxhyphen{}definition of convergence that the sequence \((x_n)\) given by \(x_n = \frac{2n}{3n-1}\) converges to \(\frac{2}{3}\) as \(n\rightarrow\infty\).


\phantomsection\label{\detokenize{Problems:p3}}
\sphinxAtStartPar
P3. \sphinxstyleemphasis{(Homework 1 question).} Using your MAS107 notes or another resource, look up and carefully state the Bolzano–Weierstrass theorem.

Its proof combines two other important results about sequences from MAS107. What do these results say?

{[}Note: We will use all three of these theorems at various points this semester. Check you are clear on what each of them is saying, and remember your lecturers and tutors are here to help.{]}


\phantomsection\label{\detokenize{Problems:p4}}
\sphinxAtStartPar
P4.\\
(i) If \(a, b \geq 0\), show that \(\sqrt{a + b} \leq \sqrt{a} + \sqrt{b}\).

{[}Hint: Consider the square of both sides. You may use that the square root function is increasing.{]}

\sphinxAtStartPar
(ii) If \(a,b \in \mathbb{R}\), deduce that \(\left|\sqrt{|a|} - \sqrt{|b|}\right| \leq \sqrt{|a - b|}\).

{[}Hint: Immitate the proof of \sphinxhref{https://rosiesb.github.io/Analysis-Notes/1Rev.html\#cor:tri}{Corollary 1.1} in the lecture notes.{]}

\sphinxAtStartPar
(iii) Hence prove that if the sequence \((a_{n})\) converges to \(l\), then \(\left(\sqrt{|a_{n}|}\right)\) converges to \(\sqrt{|l|}\).


\phantomsection\label{\detokenize{Problems:p5}}
\sphinxAtStartPar
P5. Look up and carefully define the \sphinxstyleemphasis{supremum} and \sphinxstyleemphasis{infimum} of a set \(A\subseteq\mathbb{R}\). What conditions must be met for each of these to exist?


\phantomsection\label{\detokenize{Problems:p6}}
\sphinxAtStartPar
P6. Compute, without proof, the suprema and infima (if they exist) of the following sets:

\sphinxAtStartPar
(i) \(\left\{\frac{m}{n}:m,n\in\mathbb{N} \text{ s.t } m<n\right\}\).

\sphinxAtStartPar
(ii) \(\left\{\frac{(-1)^m}{n}:m,n\in\mathbb{N} \text{ s.t } m<n\right\}\).

\sphinxAtStartPar
(iii) \(\left\{\frac{n}{3n + 1} :n\in\mathbb{N}\right\}\).


\phantomsection\label{\detokenize{Problems:p7}}
\sphinxAtStartPar
P7. Let \(A\) and \(B\) be bounded and non\sphinxhyphen{}empty subsets of \(\mathbb{R}\). Which of the following statements are true, and which are false? Prove or supply counter\sphinxhyphen{}examples, as appropriate.

\sphinxAtStartPar
(i) \(\inf A < \sup A\).

\sphinxAtStartPar
(ii) For all \(\varepsilon>0\), there is \(x\in A\) such that \(x-\inf A < \varepsilon\).

\sphinxAtStartPar
(iii) \(A\subseteq B\) implies \(\sup A \leq \sup B\).

\sphinxAtStartPar
(iv) If \(x>0\) for all \(x\in B\), then \(\inf B >0\).

\sphinxAtStartPar
(v) \(\sup(A\cup B) = \max\{\sup A,\sup B\}\).




\subsection{Limits of functions}
\label{\detokenize{Problems:limits-of-functions}}\label{\detokenize{Problems:ch2prob}}\phantomsection\label{\detokenize{Problems:id1}}\begin{enumerate}
\sphinxsetlistlabels{\arabic}{enumi}{enumii}{}{.}%
\item {} 
\sphinxAtStartPar
For each of the following formulas, what is the largest subset \(X\) of \(\mathbb{R}\)  which may be taken as the domain of a function with that formula?

\sphinxAtStartPar
(i) \(g_{1}(x) = \displaystyle\frac{x^{2} + 2x + 7}{x(x+1)}\),

\sphinxAtStartPar
(ii) \(g_{2}(x) = \displaystyle\frac{(x-1)(x+4)}{x^{3} + 4x^{2} + x - 6}\),

\sphinxAtStartPar
(iii) \(g_{3}(x) = \displaystyle\frac{x+4}{x^{2}+5x + 6}\),

\sphinxAtStartPar
(iv) \(g_{4}(x) = \exp{\left(-\displaystyle\frac{1}{x-1}\right)}\),

\sphinxAtStartPar
(v) \(g_{5}(x) = \cos\left(\displaystyle\frac{1}{\pi x}\right)\).

\end{enumerate}
\phantomsection\label{\detokenize{Problems:id2}}\begin{enumerate}
\sphinxsetlistlabels{\arabic}{enumi}{enumii}{}{.}%
\setcounter{enumi}{1}
\item {} 
\sphinxAtStartPar
\sphinxstyleemphasis{(Homework 1 question).} For each of the sets \(X\subseteq\mathbb{R}\) below, calculate the associated set \(L\) of its limit points.

\sphinxAtStartPar
(i) \(X=(0,1)\cup[2,3)\cup\{4,5\}\),

\sphinxAtStartPar
(ii) \(X=\mathbb{Z}\),

\sphinxAtStartPar
(iii) \(X=\mathbb{R}\setminus\mathbb{Z}\),

\sphinxAtStartPar
(iv) \(X=\{x\in\mathbb{Q}:0<x<1\}\),

\sphinxAtStartPar
(v) \(X=\displaystyle\left\{\frac{1}{n}:n\in\mathbb{N}\right\}\).

\end{enumerate}
\phantomsection\label{\detokenize{Problems:id3}}\begin{enumerate}
\sphinxsetlistlabels{\arabic}{enumi}{enumii}{}{.}%
\setcounter{enumi}{2}
\item {} 
\sphinxAtStartPar
\sphinxstyleemphasis{(Homework 1 question).} For each of the following functions \(f\), calculate \(\lim_{x\rightarrow 2}f(x)\), and prove your calculation is correct using the \((\varepsilon-\delta)\) criterion (\sphinxhref{https://rosiesb.github.io/Analysis-Notes/2LoF.html\#functionlimit}{Definition 2.2}). Does \(f\) converge to a limit as \(x\rightarrow 0\)?

\sphinxAtStartPar
(i) \(f:\mathbb{R}\to\mathbb{R}\); \(f(x)=4x+7\).

\sphinxAtStartPar
(ii) \(f:\{0\}\cup[1,3]\to\mathbb{R}\); \(f(x)=3x^2-1\).

\sphinxAtStartPar
(iii) \(f:(0,\infty)\to\mathbb{R}\); \(f(x)=x+\frac{1}{x}\).

\end{enumerate}
\phantomsection\label{\detokenize{Problems:id4}}\begin{enumerate}
\sphinxsetlistlabels{\arabic}{enumi}{enumii}{}{.}%
\setcounter{enumi}{3}
\item {} 
\sphinxAtStartPar
For \(g_2:X\to \mathbb{R}\) as in part (ii) of the {\hyperref[\detokenize{Problems:id1}]{\sphinxcrossref{Problem 1}}}, investigate each of

\end{enumerate}
\begin{equation*}
\begin{split}
\lim_{x \rightarrow 1}g_{2}(x), \hspace{3em} \lim_{x \rightarrow -2}g_{2}(x) \hspace{2em} \text{ and } \hspace{2em} \lim_{x \rightarrow -3}g_{2}(x).
\end{split}
\end{equation*}
\sphinxAtStartPar
Use the sequential criterion (\sphinxhref{https://rosiesb.github.io/Analysis-Notes/2LoF.html\#ed}{Theorem 2.1}) to prove you are right.

\phantomsection\label{\detokenize{Problems:id5}}\begin{enumerate}
\sphinxsetlistlabels{\arabic}{enumi}{enumii}{}{.}%
\setcounter{enumi}{4}
\item {} 
\sphinxAtStartPar
If \(f: \mathbb{R} \rightarrow [0, \infty)\) satisfies \(\displaystyle\lim_{x \rightarrow a} f(x) = l\), where \(l > 0\), show that

\end{enumerate}
\begin{equation*}
\begin{split}
\lim_{x \rightarrow a} \sqrt{f(x)} = \sqrt{l}.
\end{split}
\end{equation*}
\sphinxAtStartPar
Hence calculate \(\displaystyle\lim_{x \rightarrow 1}\sqrt{\displaystyle\frac{x+1}{x^{2}}}\).

{[}Hint: For the first part, use the result of {\hyperref[\detokenize{Problems:p4}]{\sphinxcrossref{\DUrole{std,std-ref}{P4 (iii)}}}} from the preliminary exercises.{]}

\phantomsection\label{\detokenize{Problems:id6}}\begin{enumerate}
\sphinxsetlistlabels{\arabic}{enumi}{enumii}{}{.}%
\setcounter{enumi}{5}
\item {} 
\sphinxAtStartPar
Why are limits of functions unique?

\end{enumerate}
\phantomsection\label{\detokenize{Problems:id7}}\begin{enumerate}
\sphinxsetlistlabels{\arabic}{enumi}{enumii}{}{.}%
\setcounter{enumi}{6}
\item {} 
\sphinxAtStartPar
Verify that \(\text{sgn}(x) = \displaystyle\frac{|x|}{x} = \displaystyle\frac{x}{|x|}\), for \(x \neq 0\) (see \sphinxhref{https://rosiesb.github.io/Analysis-Notes/2LoF.html\#sgn}{Example 2.3}  in the notes for the definition).

Show that \(\displaystyle\lim_{x \rightarrow 0}\text{sgn}(x)\) does not exist. {[}Hint: Use a well\sphinxhyphen{}chosen sequence and \sphinxhref{https://rosiesb.github.io/Analysis-Notes/2LoF.html\#ed}{Theorem 2.1}.{]}

Show that both the left and right limits exist at \(x=0\), and find their values.

\end{enumerate}
\phantomsection\label{\detokenize{Problems:id8}}\begin{enumerate}
\sphinxsetlistlabels{\arabic}{enumi}{enumii}{}{.}%
\setcounter{enumi}{7}
\item {} 
\sphinxAtStartPar
\sphinxstyleemphasis{(Homework 2 question).} For the following functions, each of which is defined on the whole of \(\mathbb{R}\), find every point at which both the left and right limits exist, but are different from each other, and find the values of these limits. 
Prove you are right using \sphinxhref{https://rosiesb.github.io/Analysis-Notes/2LoF.html\#marg}{Definition 2.4}, or the equivalent definitions involving sequences.

\sphinxAtStartPar
(i) \(f(x) = \begin{cases} 1 -x & \text{if }x < 1\\ x^{2}& \text{if }x \geq 1. \end{cases}\)

\sphinxAtStartPar
(ii) \(g(x) = [x]\), where
\begin{equation}\label{equation:Problems:eq:[x]}
\begin{split}[x] = \left\{\begin{array}{cl} \lfloor x\rfloor & \text{ if } x\geq 0 \\ \lceil x \rceil & \text{ if } x<0 \end{array}\right.\end{split}
\end{equation}
\sphinxAtStartPar
denotes the integer part of \(x\).

\sphinxAtStartPar
(iii) \(h(x) =3 - 5\mathbb{1}_{(0, 1]}(x) + 7\mathbb{1}_{(1, 2]}(x)\). {[}Here we have used the notation for indicator functions — see \sphinxhref{https://rosiesb.github.io/Analysis-Notes/2LoF.html\#indicatorfn}{Example 2.4} in the notes.{]}

\end{enumerate}
\phantomsection\label{\detokenize{Problems:id9}}\begin{enumerate}
\sphinxsetlistlabels{\arabic}{enumi}{enumii}{}{.}%
\setcounter{enumi}{8}
\item {} 
\sphinxAtStartPar
\sphinxstyleemphasis{(Homework 2 question).} What is the largest subset \(A\) of \(\mathbb{R}\) for which we can specify a function by \(f(x) = \sin\left(\frac{1}{x}\right)\)? Does \(f:A\to\mathbb{R}\) have a limit at \(x = 0\)? {[}Hint: Consider sequences whose \(n\)th term is \(\frac{1}{\theta + 2n\pi}\), and think about good choices for \(\theta\).{]}

\end{enumerate}
\phantomsection\label{\detokenize{Problems:id10}}\begin{enumerate}
\sphinxsetlistlabels{\arabic}{enumi}{enumii}{}{.}%
\setcounter{enumi}{9}
\item {} 
\sphinxAtStartPar
What is the largest subset \(A\) of \(\mathbb{R}\) for which we can specify a function by \(f(x) = x\sin\left(\frac{1}{x}\right)\)? Does \(f:A\to\mathbb{R}\)  have a limit at \(x = 0\)?

\end{enumerate}
\phantomsection\label{\detokenize{Problems:id11}}\begin{enumerate}
\sphinxsetlistlabels{\arabic}{enumi}{enumii}{}{.}%
\setcounter{enumi}{10}
\item {} 
\sphinxAtStartPar
In the notes, we gave meaning to \(\displaystyle\lim_{x \rightarrow a} f(x)\) where \(f:\mathbb{R} \rightarrow \mathbb{R}\) is a function and \(a \in \mathbb{R}\). In this question, you can investigate what happens when \(x\) tends to \(\infty\) or \(-\infty\).

\sphinxAtStartPar
(i) Formulate a rigorous definition, similar to Definition 2.2, of what it means for \(\displaystyle\lim_{x \rightarrow \infty}f(x)\) and \(\displaystyle\lim_{x \rightarrow -\infty}f(x)\) to exist.

{[}Hint: For the analogue of \((\varepsilon-\delta)\), think very carefully about how you are going to control the behaviour of \(x\). Remember that you cannot treat \(\infty\) as if it were a number!{]}

\sphinxAtStartPar
(ii) Find an analogue of the sequential criterion for this case, and prove the analogous result to \sphinxhref{https://rosiesb.github.io/Analysis-Notes/2LoF.html\#ed}{Theorem 2.1}.

\sphinxAtStartPar
(iii) Check that you can prove that \(\displaystyle\lim_{x \rightarrow \infty}\frac{1}{x} = \lim_{x \rightarrow -\infty}\frac{1}{x} = 0\), using either your definition or sequential criterion.

\end{enumerate}
\phantomsection\label{\detokenize{Problems:id12}}\begin{enumerate}
\sphinxsetlistlabels{\arabic}{enumi}{enumii}{}{.}%
\setcounter{enumi}{11}
\item {} 
\sphinxAtStartPar
(i) Formulate a rigorous \((\varepsilon-\delta)\) definition for \(\displaystyle\lim_{x \rightarrow \infty}f(x) = \infty\) and write down an analogue of the  sequential criterion.

\sphinxAtStartPar
{[}The cases \(\displaystyle\lim_{x \rightarrow \infty}f(x) = -\infty\) and \(\displaystyle\lim_{x \rightarrow -\infty}f(x) = \pm \infty\) can be treated similarly.{]}

\sphinxAtStartPar
(ii) Let \(f:\mathbb{R} \rightarrow [0, \infty)\) and \(g:\mathbb{R} \rightarrow [0, \infty)\), and suppose that \(\displaystyle\lim_{x \rightarrow \infty}f(x) = \infty\) and \(\displaystyle\lim_{x \rightarrow \infty}g(x) = l\), where \(l > 0\). Define \(h(x) = f(x)g(x)\) for all \(x \in \mathbb{R}\). Show that \(\displaystyle\lim_{x \rightarrow \infty}h(x) = \infty\).

\sphinxAtStartPar
(iii) Let \(p: \mathbb{R} \rightarrow \mathbb{R}\) be an even polynomial of degree \(m\), where the leading coefficient (i.e. the coefficient of \(x^{m}\)) is positive. Show that \(\displaystyle\lim_{x \rightarrow \infty}p(x) = \lim_{x \rightarrow -\infty}p(x) = \infty\). What happens when \(m\) is odd?

\end{enumerate}


\subsection{Continuity}
\label{\detokenize{Problems:continuity}}\label{\detokenize{Problems:ch3prob}}\phantomsection\label{\detokenize{Problems:id13}}\begin{enumerate}
\sphinxsetlistlabels{\arabic}{enumi}{enumii}{}{.}%
\setcounter{enumi}{12}
\item {} 
\sphinxAtStartPar
Return to {\hyperref[\detokenize{Problems:id1}]{\sphinxcrossref{Problem 1}}}. Consider each function there, defined on the largest subset \(A\) of \(\mathbb{R}\)
which can be taken as its domain, as found in that problem. For each function, what is the maximum subset of \(A\) on which it is continuous?

\end{enumerate}
\phantomsection\label{\detokenize{Problems:id14}}\begin{enumerate}
\sphinxsetlistlabels{\arabic}{enumi}{enumii}{}{.}%
\setcounter{enumi}{13}
\item {} 
\sphinxAtStartPar
Let \(f: \mathbb{R} \rightarrow \mathbb{R}\) be continuous at a point \(a\).
Prove that \(|f|\) is continuous at \(a\), where \(|f|(x) = |f(x)|\), for all \(x \in \mathbb{R}\).

{[}Hint: Use the formulation of continuity with sequences (\sphinxhref{https://rosiesb.github.io/Analysis-Notes/3Cty.html\#cont1}{Theorem 3.1(ii)} and \sphinxhref{https://rosiesb.github.io/Analysis-Notes/1Rev.html\#tri}{Corollary 1.1} in the notes).{]}

\end{enumerate}
\phantomsection\label{\detokenize{Problems:id15}}\begin{enumerate}
\sphinxsetlistlabels{\arabic}{enumi}{enumii}{}{.}%
\setcounter{enumi}{14}
\item {} 
\sphinxAtStartPar
Prove \sphinxhref{https://rosiesb.github.io/Analysis-Notes/3Cty.html\#fof}{Theorem 3.3} in the notes (i.e. that composition of continuous functions is continuous). {[}Hint: Use sequences.{]}

\end{enumerate}
\phantomsection\label{\detokenize{Problems:id16}}\begin{enumerate}
\sphinxsetlistlabels{\arabic}{enumi}{enumii}{}{.}%
\setcounter{enumi}{15}
\item {} 
\sphinxAtStartPar
Define \(f:\mathbb{R} \setminus \{0\}\to \mathbb{R}\) and \(g: \mathbb{R} \rightarrow \mathbb{R}\setminus \{0\}\) by \(f(x) = \frac{1}{x}\), and \(g(x) = 1 + x^{2}\). Write down the functions \(f \circ g\), and \(g \circ f\), giving their domains explicitly. What can you say about continuity of these functions?

\end{enumerate}
\phantomsection\label{\detokenize{Problems:id17}}\begin{enumerate}
\sphinxsetlistlabels{\arabic}{enumi}{enumii}{}{.}%
\setcounter{enumi}{16}
\item {} 
\sphinxAtStartPar
For each of the functions in {\hyperref[\detokenize{Problems:id8}]{\sphinxcrossref{Problem 8}}}:

\sphinxAtStartPar
(i) Find the maximum subset of \(\mathbb{R}\) on which it is continuous.

\sphinxAtStartPar
(ii) Identify those discontinuities which are jumps, and calculate the size of each jump.

\end{enumerate}
\phantomsection\label{\detokenize{Problems:id18}}\begin{enumerate}
\sphinxsetlistlabels{\arabic}{enumi}{enumii}{}{.}%
\setcounter{enumi}{17}
\item {} 
\sphinxAtStartPar
\sphinxstyleemphasis{(Homework 3 question).}

\sphinxAtStartPar
(i) Define \(f:\mathbb{R} \setminus \{0\} \rightarrow \mathbb{R}\) by
\begin{equation*}
\begin{split}
    f(x) = \displaystyle\frac{(1 + x)^{2} - 1}{x}.
    \end{split}
\end{equation*}
\sphinxAtStartPar
Prove that \(f\) has a continuous extension to the whole of \(\mathbb{R}\), and give its formula.

\sphinxAtStartPar
(ii) Write down the largest subset of \(\mathbb{R}\) which can be taken as the domain \(A\) of the function \(f\) given by
\begin{equation*}
\begin{split}
    f(x) = \displaystyle\frac{x^{3}-8}{x^{2} - 4}
    \end{split}
\end{equation*}
\sphinxAtStartPar
and explain why \(f\) is continuous at every point of \(A\) (you may find it helpful to reference theorems/examples from the lecture notes).

\sphinxAtStartPar
The complement \(A^{c}\) of \(A\) in \(\mathbb{R}\) comprises two points. Show that \(f\) may be extended to be continuous at only one of these points, and write down this continuous extension.

\end{enumerate}
\phantomsection\label{\detokenize{Problems:id19}}\begin{enumerate}
\sphinxsetlistlabels{\arabic}{enumi}{enumii}{}{.}%
\setcounter{enumi}{18}
\item {} 
\sphinxAtStartPar
\sphinxstyleemphasis{(Homework 2 question).} Suppose that the function \(g: \mathbb{R} \rightarrow \mathbb{R}\) is continuous at \(a\) with \(g(a) > 0\). Show that there exists \(\delta > 0\) such that \(g(x) > 0\)  for all \( x \in (a - \delta, a + \delta)\). {[}Hint: Use the \(\varepsilon-\delta\) criterion for continuity.{]}

\end{enumerate}
\phantomsection\label{\detokenize{Problems:id20}}\begin{enumerate}
\sphinxsetlistlabels{\arabic}{enumi}{enumii}{}{.}%
\setcounter{enumi}{19}
\item {} 
\sphinxAtStartPar
(i) For any \(x, y \in \mathbb{R}\) show that
\begin{equation*}
\begin{split}
    \max\{x, y\} = \frac{1}{2}(x + y) + \frac{1}{2}|x - y|.
    \end{split}
\end{equation*}
\sphinxAtStartPar
Hence show that if both \(f:A\to \mathbb{R}\) and \(g: B \rightarrow \mathbb{R}\) are continuous at \(a\), then so is \(\max\{f, g\}\), where for all \(x \in A\cap B\),
\begin{equation*}
\begin{split}
    \max\{f, g\}(x)  = \max\{f(x), g(x)\}.
    \end{split}
\end{equation*}
\sphinxAtStartPar
(ii) Find a similar expression for \(\min\{x, y\}\), and hence prove continuity of \(\min\{f, g\}\) at \(a\).

\end{enumerate}
\phantomsection\label{\detokenize{Problems:id21}}\begin{enumerate}
\sphinxsetlistlabels{\arabic}{enumi}{enumii}{}{.}%
\setcounter{enumi}{20}
\item {} 
\sphinxAtStartPar
The aim of this question is to prove the following: the only functions \(f: \mathbb{R} \rightarrow \mathbb{R}\) which are continuous at zero and satisfy \(f(x+y) = f(x) + f(y)\) for all \(x,y \in \mathbb{R}\) are the linear mappings \(f(x) = kx\) for all \(x \in \mathbb{R}\), where \(k \in \mathbb{R}\) is fixed.

Begin by considering \(f: \mathbb{R} \rightarrow \mathbb{R}\), such that \(f(x+y) = f(x) + f(y)\) for all \(x,y \in \mathbb{R}\).

\sphinxAtStartPar
(i) Prove that \(f(0) = 0\).

\sphinxAtStartPar
(ii) Show that \(f(-x) = -f(x)\) for all \(x \in \mathbb{R}\).

\sphinxAtStartPar
(iii) If \(f\) is continuous at zero, prove that it is continuous at every \(x \in \mathbb{R}\).

\sphinxAtStartPar
(iv) If \(f(1) = k\), prove that \(f(n) = kn\) for all \(n\in \mathbb{Z}\).

\sphinxAtStartPar
(v) If \(f(1) = k\), prove that \(f\left(\frac{p}{q}\right) = k\frac{p}{q}\) for all \(\frac{p}{q} \in \mathbb{Q}\).

\sphinxAtStartPar
(vi) If \(f(1) = k\) and \(f\) is continuous at zero, prove that \(f(x) =kx\) for all \(x \in \mathbb{R}\).

\end{enumerate}
\phantomsection\label{\detokenize{Problems:id22}}\begin{enumerate}
\sphinxsetlistlabels{\arabic}{enumi}{enumii}{}{.}%
\setcounter{enumi}{21}
\item {} 
\sphinxAtStartPar
\sphinxstyleemphasis{(Homework 2 question).} Show that Dirichlet’s “other” function, as discussed in \sphinxhref{https://rosiesb.github.io/Analysis-Notes/3Cty.html\#eg:dirichlet2}{Example 3.10} in the notes, is discontinuous at every rational point in its domain.

{[}Hint: Let \(a\in[0,1)\cap\mathbb{Q}\), and use the fact that for all \(n\in\mathbb{N}\), there is an irrational number \(x_n\) lying strictly between \(a\) and \(a+\frac{1}{n}\). (See also Cor 2.14 on page 42 of the \sphinxhref{https://drive.google.com/file/d/1r9b3XqA1u-dzkbnGjPPBxIyj6YF1TE\_C/view?usp=sharing}{MAS107/117 Sem 2 lecture notes}.) {]}

\end{enumerate}
\phantomsection\label{\detokenize{Problems:id23}}\begin{enumerate}
\sphinxsetlistlabels{\arabic}{enumi}{enumii}{}{.}%
\setcounter{enumi}{22}
\item {} 
\sphinxAtStartPar
What can you say about left/right continuity of the function  \(\mathbb{1}_{(a, b)}:\mathbb{R}\to\mathbb{R}\) at the points \(a\) and \(b\)?

\end{enumerate}
\phantomsection\label{\detokenize{Problems:id24}}\begin{enumerate}
\sphinxsetlistlabels{\arabic}{enumi}{enumii}{}{.}%
\setcounter{enumi}{23}
\item {} 
\sphinxAtStartPar
Finish the proof of the intermediate value theorem (\sphinxhref{https://rosiesb.github.io/Analysis-Notes/3Cty.html\#ivt}{Theorem 3.4}). {[}Hint: Apply the special case \sphinxhref{https://rosiesb.github.io/Analysis-Notes/3Cty.html\#ivt-sc}{Proposition 3.1} to the function \(g(x) = f(x) - \gamma\), for \(x \in [a, b]\).{]}

\end{enumerate}
\phantomsection\label{\detokenize{Problems:id25}}\begin{enumerate}
\sphinxsetlistlabels{\arabic}{enumi}{enumii}{}{.}%
\setcounter{enumi}{24}
\item {} 
\sphinxAtStartPar
\sphinxstyleemphasis{(Homework 3 question).} Prove the following \sphinxstyleemphasis{fixed point theorem}: if \(f:[a,b] \rightarrow (a,b)\) is continuous, then there exists \(c \in (a,b)\) such that \(f(c) = c\).

{[}Hint: This is a similar proof to that of {\hyperref[\detokenize{Problems:id24}]{\sphinxcrossref{Problem 24}}}. This time you need to consider a function of the form \(g(x) = f(x) - \) (\sphinxstyleemphasis{something}). What is \sphinxstyleemphasis{something}?{]} Give a counter\sphinxhyphen{}example to demonstrate that the claim is false if the domain of \(f\) is restricted to \((0, 1)\).

\end{enumerate}
\phantomsection\label{\detokenize{Problems:id26}}\begin{enumerate}
\sphinxsetlistlabels{\arabic}{enumi}{enumii}{}{.}%
\setcounter{enumi}{25}
\item {} 
\sphinxAtStartPar
Complete the proof of the extreme value theorem (\sphinxhref{https://rosiesb.github.io/Analysis-Notes/3Cty.html\#thm:evt}{Theorem 3.5}), i.e. prove that a continuous function \(f:[a,b]\to\mathbb{R}\) attains its infimum.

\end{enumerate}
\phantomsection\label{\detokenize{Problems:id27}}\begin{enumerate}
\sphinxsetlistlabels{\arabic}{enumi}{enumii}{}{.}%
\setcounter{enumi}{26}
\item {} 
\sphinxAtStartPar
Use \sphinxhref{https://rosiesb.github.io/Analysis-Notes/3Cty.html\#interval}{Corollary 3.1} to prove each of the following statements.

\sphinxAtStartPar
(i) Every continuous function from \(\mathbb{R}\) to \(\mathbb{Z}\) is constant.

\sphinxAtStartPar
(ii) Every continuous function from \(\mathbb{R}\) to \(\mathbb{Q}\) is constant.

\end{enumerate}
\phantomsection\label{\detokenize{Problems:id28}}\begin{enumerate}
\sphinxsetlistlabels{\arabic}{enumi}{enumii}{}{.}%
\setcounter{enumi}{27}
\item {} 
\sphinxAtStartPar
Show that if \(f:[0,1] \rightarrow \mathbb{R}\) is continuous and \(0\) is not in the range of  \(f\), then the function \(\frac{1}{f}:[0,1]\to \mathbb{R}\) is bounded, where for each \(x \in [0,1], \left(\frac{1}{f}\right)(x) = \frac{1}{f(x)}\).

\end{enumerate}
\phantomsection\label{\detokenize{Problems:id29}}\begin{enumerate}
\sphinxsetlistlabels{\arabic}{enumi}{enumii}{}{.}%
\setcounter{enumi}{28}
\item {} 
\sphinxAtStartPar
Explain why there are no continuous functions having domain \([0, 1]\) and range \(\mathbb{R}\). {[}Hint: Use \sphinxhref{https://rosiesb.github.io/Analysis-Notes/3Cty.html\#thm:evt}{Theorem 3.5}.{]}

\end{enumerate}
\phantomsection\label{\detokenize{Problems:id30}}\begin{enumerate}
\sphinxsetlistlabels{\arabic}{enumi}{enumii}{}{.}%
\setcounter{enumi}{29}
\item {} 
\sphinxAtStartPar
Suppose that the function \(f:[0,1] \rightarrow [0,1]\) is continuous, and fix \(0 < r < 1\). Suppose that we are given a sequence \((x_{n})\) in \([0,1]\) such that \(f(x_{n+1}) \leq rf(x_{n})\) for all \(n\in\mathbb{N}\). Show that there exists \(c \in [0, 1]\) for which \(f(c) = 0\). {[}Hint: Use the Bolzano\sphinxhyphen{}Weierstrass theorem.{]}

\end{enumerate}
\phantomsection\label{\detokenize{Problems:id31}}\begin{enumerate}
\sphinxsetlistlabels{\arabic}{enumi}{enumii}{}{.}%
\setcounter{enumi}{30}
\item {} 
\sphinxAtStartPar
Show that for each \(n\in\mathbb{N}\), the function \(f: [0, \infty)\to\mathbb{R}\) given by \(f(x) =x^{n}\) is strictly monotonic increasing.

\end{enumerate}
\phantomsection\label{\detokenize{Problems:id32}}\begin{enumerate}
\sphinxsetlistlabels{\arabic}{enumi}{enumii}{}{.}%
\setcounter{enumi}{31}
\item {} 
\sphinxAtStartPar
Show that the function \(f(x) = \sin(x)\) has a continuous inverse when restricted to the interval \(\left[-\frac{\pi}{2}, \frac{\pi}{2}\right]\). What goes wrong outside this interval? {[}Hint: Use an appropriate trigonometric identity.{]}

\end{enumerate}
\phantomsection\label{\detokenize{Problems:id33}}\begin{enumerate}
\sphinxsetlistlabels{\arabic}{enumi}{enumii}{}{.}%
\setcounter{enumi}{32}
\item {} 
\sphinxAtStartPar
Find \(\displaystyle\lim_{x \rightarrow 1}\frac{1 - x}{1 - x^{\frac{m}{n}}}\), where \(m, n \in \mathbb{N}\).

\end{enumerate}
\phantomsection\label{\detokenize{Problems:id34}}\begin{enumerate}
\sphinxsetlistlabels{\arabic}{enumi}{enumii}{}{.}%
\setcounter{enumi}{33}
\item {} 
\sphinxAtStartPar
If \(f:[a, b] \rightarrow \mathbb{R}\) is continuous and bijective, and \(f(a) < f(b)\), show that \(f\) is strictly monotonic increasing.

\end{enumerate}
\phantomsection\label{\detokenize{Problems:id35}}
\sphinxAtStartPar
35.* Show that if \(f, g: \mathbb{R} \rightarrow \mathbb{R}\)  are monotonic increasing, with \(f+g\) continuous, then both \(f\) and \(g\) are continuous.

\phantomsection\label{\detokenize{Problems:id36}}
\sphinxAtStartPar
36.*
(i) Let \((x_{n})\) be a sequence in \(\mathbb{R}\) for which \(x_{1} = a > 0\) and \(x_{n+1} = \sqrt{x_{n}}\), for all \(n\in\mathbb{N}\). Show that \(\displaystyle\lim_{n \rightarrow \infty} x_{n} = 1\).

\sphinxAtStartPar
(ii) Let \(f: \mathbb{R} \rightarrow \mathbb{R}\) be a continuous function for which \(f(x) = f(x^{2})\) for all \(x \in \mathbb{R}\). Use the result of (i) to show that \(f\) is constant.


\subsection{Differentiation}
\label{\detokenize{Problems:differentiation}}\label{\detokenize{Problems:ch4prob}}\phantomsection\label{\detokenize{Problems:id37}}\begin{enumerate}
\sphinxsetlistlabels{\arabic}{enumi}{enumii}{}{.}%
\setcounter{enumi}{36}
\item {} 
\sphinxAtStartPar
Let \(f:A\to\mathbb{R}\) and let \(a\in\mathbb{R}\) be such that there is some sequence \((x_n)\) in \(A\) with
\(x_n\neq a\) for all \(n\) and \(\displaystyle\lim_{n\to\infty} x_n=a\). What does it mean to say that
the function \(f\) has limit \(l\) at the point \(a\)? What notation do we use to write this?

{[}This question is asking you to recall (or revise if you can’t recall) \sphinxhref{https://rosiesb.github.io/Analysis-Notes/2LoF.html\#functionlimit}{Definition 2.2} from the lecture notes.That’s a key definition and we build on it when we define what it means to be differentiable — see the next question.{]}

\end{enumerate}
\phantomsection\label{\detokenize{Problems:id38}}\begin{enumerate}
\sphinxsetlistlabels{\arabic}{enumi}{enumii}{}{.}%
\setcounter{enumi}{37}
\item {} 
\sphinxAtStartPar
(i) Give the definition of what it means for a function \(f:A\to\mathbb{R}\) to be differentiable at \(a \in A\).

\sphinxAtStartPar
(ii) State carefully what this means in terms of the \(\varepsilon-\delta\) criterion.

\sphinxAtStartPar
(iii) State carefully what this means in terms of limits of sequences.
{[}The first part is just asking you to give \sphinxhref{https://rosiesb.github.io/Analysis-Notes/4Diff.html\#def:diff}{Definition 4.1} and the other parts are checking that you know what this means.

The key points from Chapter 2 are \sphinxhref{https://rosiesb.github.io/Analysis-Notes/2LoF.html\#functionlimit}{Definition 2.2} of a limit of a function and \sphinxhref{https://rosiesb.github.io/Analysis-Notes/2LoF.html\#ed}{Theorem 2.1} giving the sequential criterion. But you need to see how to apply these, not just copy them out. Here they have to be applied to the relevant function for the definition of differentiability, not \(f\) itself. {]}

\end{enumerate}
\phantomsection\label{\detokenize{Problems:id39}}\begin{enumerate}
\sphinxsetlistlabels{\arabic}{enumi}{enumii}{}{.}%
\setcounter{enumi}{38}
\item {} 
\sphinxAtStartPar
Give a rigorous proof that the function \(f:\mathbb{R}\setminus\{0\}\to \mathbb{R}\) given by
\(f(x) = \frac{1}{x}\) is differentiable for all \(x \in \mathbb{R} \setminus \{0\}\) from the definition of the derivative and find \(f'(x)\) explicitly. Can we extend the function so that it is differentiable on the whole of \(\mathbb{R}\) by defining its value at zero to be zero?

\end{enumerate}
\phantomsection\label{\detokenize{Problems:id40}}\begin{enumerate}
\sphinxsetlistlabels{\arabic}{enumi}{enumii}{}{.}%
\setcounter{enumi}{39}
\item {} 
\sphinxAtStartPar
Let \(k\in \mathbb{R}\setminus\{0\}\). Give a rigorous proof from the definition of the derivative that  \(f:\mathbb{R}\to \mathbb{R}\) given by  \(f(x) = e^{kx}\) is differentiable for all \(x \in \mathbb{R}\), and find \(f'(x)\) explicitly.

{[}Hint: For fixed \(k\in \mathbb{R}\), let \(g:\mathbb{R}\to\mathbb{R}\) be given by \(g(h)=e^{kh} - 1 - kh \).
You may use the fact that  \(\displaystyle\lim_{h \rightarrow 0}\frac{g(h)}{h} = 0\). (This follows from the series expansion of the exponential function, which you already know, and we will study more later.) {]}

\end{enumerate}
\phantomsection\label{\detokenize{Problems:id41}}\begin{enumerate}
\sphinxsetlistlabels{\arabic}{enumi}{enumii}{}{.}%
\setcounter{enumi}{40}
\item {} 
\sphinxAtStartPar
Consider the function  \(f:\mathbb{R}\to \mathbb{R}\) given by \(f(x) = \begin{cases} x\sin\left(\frac{1}{x}\right) & \text{if }x \neq 0,\\ 0 & \text{if }x = 0.\end{cases}\)

\sphinxAtStartPar
(i) Show that \(f\) is differentiable at every \(x \neq 0\). (You can use standard derivatives and facts about derivatives, such as the product and chain rules.)

\sphinxAtStartPar
(ii) Show that \(f\) is not differentiable at \(x = 0\). (Use the definition of the derivative as a limit. You may assume that \(\sin\left(\frac{1}{x}\right)\) has no limit as \(x\) tends to \(0\). This was {\hyperref[\detokenize{Problems:id8}]{\sphinxcrossref{Problem 8}}}.)

\end{enumerate}
\phantomsection\label{\detokenize{Problems:id42}}\begin{enumerate}
\sphinxsetlistlabels{\arabic}{enumi}{enumii}{}{.}%
\setcounter{enumi}{41}
\item {} 
\sphinxAtStartPar
Show that the function  \(f:\mathbb{R}\to \mathbb{R}\) given by

\end{enumerate}
\begin{equation*}
\begin{split}
f(x) = \begin{cases} x^2\sin\left(\frac{1}{x}\right) & \text{if }x \neq 0,\\ 0 & \text{if }x = 0.\end{cases}
\end{split}
\end{equation*}
\sphinxAtStartPar
is differentiable at every \(x \in \mathbb{R}\). What can you say about its second derivative?

{[}Hint: You will need to consider the case \(x=0\) separately from \(x\in\mathbb{R}\setminus\{0\}\). You may assume that the limit as \(x\) tends to \(0\) of \(x\sin\left(\frac{1}{x}\right)\) exists and equals \(0\). This was studied in {\hyperref[\detokenize{Problems:id10}]{\sphinxcrossref{\DUrole{std,std-ref}{Problem 10}}}}.{]}

\phantomsection\label{\detokenize{Problems:id43}}\begin{enumerate}
\sphinxsetlistlabels{\arabic}{enumi}{enumii}{}{.}%
\setcounter{enumi}{42}
\item {} 
\sphinxAtStartPar
(i) Sketch the graph of any continuous function \(f:[0,1]\to \mathbb{R}\) which is not differentiable at \(x=\frac{1}{2}\), but is differentiable at all other points. {[}You do \sphinxstyleemphasis{not} need to give a formula.{]}

\sphinxAtStartPar
(ii) Sketch the graph of any continuous function \(f:[0,1]\to \mathbb{R}\) which is not differentiable at \(x=\frac{1}{3}\) and is not differentiable at \(x=\frac{2}{3}\), but is differentiable at all other points. {[}You do \sphinxstyleemphasis{not} need to give a formula.{]}

\end{enumerate}
\phantomsection\label{\detokenize{Problems:id44}}\begin{enumerate}
\sphinxsetlistlabels{\arabic}{enumi}{enumii}{}{.}%
\setcounter{enumi}{43}
\item {} 
\sphinxAtStartPar
Let \(f:\mathbb{R}\to \mathbb{R}\) be given by \(f(x) = x - [x]\) for all \(x \in \mathbb{R}\), where \([x]\) denotes the integer part of \(f\) (see {\hyperref[\detokenize{Problems:equation-eq-x}]{\sphinxcrossref{(1.1)}}}).

Explain carefully at which points \(f\)  is differentiable, and find the value of its derivative there. {[}It will probably help to sketch the graph!{]}

\end{enumerate}
\phantomsection\label{\detokenize{Problems:id45}}\begin{enumerate}
\sphinxsetlistlabels{\arabic}{enumi}{enumii}{}{.}%
\setcounter{enumi}{44}
\item {} 
\sphinxAtStartPar
Show that if \(f:\mathbb{R} \rightarrow \mathbb{R}\) is differentiable at \(a\) then

\end{enumerate}
\begin{equation*}
\begin{split}
\lim_{h \downarrow 0} \displaystyle\frac{f(a + h) - f(a - h)}{2h} = f'(a).
\end{split}
\end{equation*}
\sphinxAtStartPar
By considering \(f:\mathbb{R}\to\mathbb{R}\) given by \(f(x) = |x|\), show that the limit on the left\sphinxhyphen{}hand side may exist, even when \(f\) is not differentiable at \(a\).

\phantomsection\label{\detokenize{Problems:id46}}\begin{enumerate}
\sphinxsetlistlabels{\arabic}{enumi}{enumii}{}{.}%
\setcounter{enumi}{45}
\item {} 
\sphinxAtStartPar
A function \(f:\mathbb{R} \rightarrow \mathbb{R}\) is defined by \(f(x) = \begin{cases} -x^{2} & \text{if }x < 0,\\ x^{2} & \text{if }x \geq 0. \end{cases}\)

Determine whether each of the following is true or false. Justify your answers.

\sphinxAtStartPar
(i) \(f\) is continuous at \(0\).

\sphinxAtStartPar
(ii) \(f'(0)\) exists.

\sphinxAtStartPar
(iii) \(f':\mathbb{R}\to\mathbb{R}\) is continuous at \(0\).

\sphinxAtStartPar
(iv) \(f^{\prime \prime}(0)\) exists.

\end{enumerate}
\phantomsection\label{\detokenize{Problems:id47}}\begin{enumerate}
\sphinxsetlistlabels{\arabic}{enumi}{enumii}{}{.}%
\setcounter{enumi}{46}
\item {} 
\sphinxAtStartPar
(i) Must any differentiable function \(f:[a, b] \rightarrow \mathbb{R}\) have a maximum and minimum value? Why?

\sphinxAtStartPar
(ii) If \(f:[a, b] \rightarrow \mathbb{R}\) is differentiable  and \(f(a) = f(b)\), must \(f\) have a maximum and minimum value in \((a, b)\)?

\end{enumerate}
\phantomsection\label{\detokenize{Problems:id48}}\begin{enumerate}
\sphinxsetlistlabels{\arabic}{enumi}{enumii}{}{.}%
\setcounter{enumi}{47}
\item {} 
\sphinxAtStartPar
Let \(a_{0}, a_{1}, \ldots, a_{n}\) be real numbers such that

\end{enumerate}
\begin{equation*}
\begin{split}
a_{0} + \frac{a_{1}}{2} + \frac{a_{2}}{3} + \cdots + \frac{a_{n}}{n+1} = 0.
\end{split}
\end{equation*}
\sphinxAtStartPar
Consider \(f:\mathbb{R}\to\mathbb{R}\) given by \(f(x) = a_{0}+ a_{1}x + a_{2}x^{2} + \cdots + a_{n}x^{n}\). Show that there is some \(c\in (0,1)\) such that \(f(c)=0\). {[}Hint: Integrate the function \(f\) term–by–term, and think about how to use Rolle’s theorem.{]}

\phantomsection\label{\detokenize{Problems:id49}}\begin{enumerate}
\sphinxsetlistlabels{\arabic}{enumi}{enumii}{}{.}%
\setcounter{enumi}{48}
\item {} 
\sphinxAtStartPar
(i) Use the mean value theorem to show the following. If \(f:\mathbb{R} \rightarrow \mathbb{R}\) is continuous on \([a, b]\) and differentiable on \((a, b)\) with \(f'(c)= 0\) for all \(c \in (a, b)\), then \(f\) is constant on \([a, b]\).

\sphinxAtStartPar
(ii) Use part (i) to show that if \(g, h:\mathbb{R}\to\mathbb{R}\) are both continuous on \([a, b]\) and differentiable on \((a, b)\) with \(h'(x) = g'(x)\) for all \(x \in (a, b)\), then there exists \(k \in \mathbb{R}\) such that \(h(x) = g(x) + k\), for all \(x \in [a, b]\).

\end{enumerate}
\phantomsection\label{\detokenize{Problems:id50}}\begin{enumerate}
\sphinxsetlistlabels{\arabic}{enumi}{enumii}{}{.}%
\setcounter{enumi}{49}
\item {} 
\sphinxAtStartPar
If \(f:\mathbb{R} \rightarrow \mathbb{R}\) is continuous on \([a, b]\) and differentiable on \((a, b)\) and there exist \(m, M \in \mathbb{R}\) such that \(m \leq f'(c) \leq M\), for all \(c \in (a, b)\), show that

\end{enumerate}
\begin{equation*}
\begin{split}
f(a) + m(b - a) \leq f(b) \leq f(a) + M(b-a).
\end{split}
\end{equation*}\phantomsection\label{\detokenize{Problems:id51}}\begin{enumerate}
\sphinxsetlistlabels{\arabic}{enumi}{enumii}{}{.}%
\setcounter{enumi}{50}
\item {} 
\sphinxAtStartPar
If \(r > 0\) and \(q \in \mathbb{R}\) show that the polynomial \(p(x) = x^{3} + rx + q\) has exactly one real zero.
{[} You may assume the result of \sphinxhref{https://rosiesb.github.io/Analysis-Notes/3Cty.html\#pol}{Corollary 3.2}: \(p\) has at least one real root. So you need to show that there can’t be more than one. {]}

\end{enumerate}
\phantomsection\label{\detokenize{Problems:id52}}\begin{enumerate}
\sphinxsetlistlabels{\arabic}{enumi}{enumii}{}{.}%
\setcounter{enumi}{51}
\item {} 
\sphinxAtStartPar
Let \(r>1\) and fix \(y\in (0,1)\). By applying the mean value theorem to the function \(f(x)=x^r\) on \([y,1]\), show that

\end{enumerate}
\begin{equation*}
\begin{split}
1 - y^r < r(1 - y).
\end{split}
\end{equation*}\phantomsection\label{\detokenize{Problems:id53}}\begin{enumerate}
\sphinxsetlistlabels{\arabic}{enumi}{enumii}{}{.}%
\setcounter{enumi}{52}
\item {} 
\sphinxAtStartPar
Let \(f:\mathbb{R} \rightarrow \mathbb{R}\) be twice differentiable at \(a\) with \(f'(a) = 0\). If \(f^{\prime \prime}(a) < 0\), show that \(f\) has a local maximum at \(a\), while if \(f^{\prime \prime}(a) > 0\), show that \(f\) has a local minimum at \(a\).

\end{enumerate}


\subsection{Sequences and series of functions}
\label{\detokenize{Problems:sequences-and-series-of-functions}}\label{\detokenize{Problems:ch5prob}}\phantomsection\label{\detokenize{Problems:id54}}\begin{enumerate}
\sphinxsetlistlabels{\arabic}{enumi}{enumii}{}{.}%
\setcounter{enumi}{53}
\item {} 
\sphinxAtStartPar
Consider the sequence of functions  \((f_n)\), where \(f_n:[0,\pi ]\to \mathbb{R}\) is defined by \(f_n(x) = \sin^n (x)\) for each \(n\in\mathbb{N}\). Show that the sequence \((f_n)\) converges pointwise. Does the sequence \((f_n)\)  converge uniformly? Justify your answer.

\end{enumerate}
\phantomsection\label{\detokenize{Problems:id55}}\begin{enumerate}
\sphinxsetlistlabels{\arabic}{enumi}{enumii}{}{.}%
\setcounter{enumi}{54}
\item {} 
\sphinxAtStartPar
For each of the following sequences of functions \((f_n)\) determine the pointwise limit (if it exists), and decide whether \((f_n)\) converges uniformly to this limit.

\sphinxAtStartPar
(i) \(f_n:[0,1]\to\mathbb{R}\), \(f_n (x) = x^{\frac{1}{n}}\).

\sphinxAtStartPar
(ii) \(f_n:\mathbb{R}\to\mathbb{R}\), where \(\displaystyle f_n (x)  = \left\{ \begin{array}{ll} 0 & x\leq n, \\ x-n & x\geq n \\ \end{array} \right.\).

\sphinxAtStartPar
(iii) \(f_n:(1,\infty)\to\mathbb{R}\), \(f_n(x) = \frac{e^x}{x^n}\).

\sphinxAtStartPar
(iv) \(f_n:[-1,1]\to\mathbb{R}\), \(f_n(x) = e^{-nx^2}\).

\sphinxAtStartPar
(v) \(f_n:\mathbb{R}\to\mathbb{R}\), \(f_n(x) = e^{-x^2}{n}\).

\end{enumerate}
\phantomsection\label{\detokenize{Problems:id56}}\begin{enumerate}
\sphinxsetlistlabels{\arabic}{enumi}{enumii}{}{.}%
\setcounter{enumi}{55}
\item {} 
\sphinxAtStartPar
For each of the following sequences of functions \((g_n)\) find the pointwise limit, and determine whether the sequence converges uniformly on \([0,1]\), and on \([0,\infty)\).

\sphinxAtStartPar
(i) \(\displaystyle g_n(x) = \frac{x}{n}\).

\sphinxAtStartPar
(ii) \(\displaystyle g_n(x) = \frac{x^n}{1+x^n}\).

\sphinxAtStartPar
(iii) \(\displaystyle g_n (x) = \frac{x^n}{n+x^n}\).

\end{enumerate}
\phantomsection\label{\detokenize{Problems:id57}}\begin{enumerate}
\sphinxsetlistlabels{\arabic}{enumi}{enumii}{}{.}%
\setcounter{enumi}{56}
\item {} 
\sphinxAtStartPar
For each of the following sequences of functions \((h_n)\), where \(h_n\colon [0,1]\rightarrow \mathbb{R}\), find the pointwise limit, if it exists, and in that case determine whether the sequence converges uniformly.

\sphinxAtStartPar
(i) \(h_n(x) = \left(1-\frac{x}{n}\right)^2\).

\sphinxAtStartPar
(ii) \(h_n(x) = x-x^n\).

\sphinxAtStartPar
(iii) \(h_n (x) = \sum_{k=0}^n x^k\).

\end{enumerate}
\phantomsection\label{\detokenize{Problems:id58}}\begin{enumerate}
\sphinxsetlistlabels{\arabic}{enumi}{enumii}{}{.}%
\setcounter{enumi}{57}
\item {} 
\sphinxAtStartPar
Define \(f_n\colon \mathbb{R}\rightarrow \mathbb{R}\) by \(f_n (x) = \frac{n+\cos x}{2n+\sin^2 x}\).

\sphinxAtStartPar
(i) Find the pointwise limit of the sequence of functions \((f_n)\).

\sphinxAtStartPar
(ii) Show that the sequence \((f_n)\) converges uniformly.

\sphinxAtStartPar
(iii) Calculate \(f_n'\) and show that \(f'(x) = \lim_{n\rightarrow \infty} f_n'(x)\). Does \((f_n)\) satisfy the conditions of \sphinxhref{https://rosiesb.github.io/Analysis-Notes/5SSF.html\#udiff}{Theorem 5.2}?

\end{enumerate}
\phantomsection\label{\detokenize{Problems:id59}}\begin{enumerate}
\sphinxsetlistlabels{\arabic}{enumi}{enumii}{}{.}%
\setcounter{enumi}{58}
\item {} 
\sphinxAtStartPar
(i) Let \(n\in \mathbb{N}\). Show that we can define a continuous function \(f_n\colon [0,1]\rightarrow \mathbb{R}\) by
\begin{equation*}
\begin{split}
    f_n(x) = \left\{ \begin{array}{ll}
    \displaystyle 0 & x=0, \\
    \displaystyle\frac{x^{\frac{1}{n}}-1}{\ln x} & 0<x<1, \\
    \displaystyle\frac{1}{n} & x=1. \\
    \end{array} \right.
    \end{split}
\end{equation*}
\sphinxAtStartPar
(Note: you only need check continuity at \(x=0\) and \(x=1\).)

\sphinxAtStartPar
(ii) Does the sequence \((f_n)\) converge  uniformly to a limit? Justify your answer. If you wish, you may assume without proof that each function \(f_n\) is monotone increasing.

\end{enumerate}
\phantomsection\label{\detokenize{Problems:id60}}\begin{enumerate}
\sphinxsetlistlabels{\arabic}{enumi}{enumii}{}{.}%
\setcounter{enumi}{59}
\item {} 
\sphinxAtStartPar
Let \(f_n:[a,b]\to\mathbb{R}\). Suppose that \(\sum_{n=1}^\infty f_n\) converges uniformly to \(f\). Show that the sequence \((f_n)\) converges uniformly to the zero function.

\end{enumerate}
\phantomsection\label{\detokenize{Problems:id61}}\begin{enumerate}
\sphinxsetlistlabels{\arabic}{enumi}{enumii}{}{.}%
\setcounter{enumi}{60}
\item {} 
\sphinxAtStartPar
The functions \(c,s:\mathbb{R}\to\mathbb{R}\) are defined by the infinite series \(s(x) = \sum_{n=0}^\infty\frac{(-1)^nx^{2n+1}}{(2n+1)!}\), \(c(x) = \sum_{n=0}^\infty\frac{(-1)^kx^{2n}}{(2n)!}\).

\sphinxAtStartPar
(i) Prove that these functions are well\sphinxhyphen{}defined as pointwise limits, and that their series converge absolutely.

\sphinxAtStartPar
(ii) Show that when restricted to a closed bounded interval \([-R,R]\), where \(R>0\), both series’ converge uniformly.

\sphinxAtStartPar
(iii) Prove that \(s\) and \(c\) are differentiable everywhere, with \(s'(x)=c(x)\) and \(c'(x)=-s(x)\).
{[}Hint: Mimic the approach of \sphinxhref{https://rosiesb.github.io/Analysis-Notes/5SSF.html\#eg:exp}{Example 5.7} in the notes.{]}

\sphinxAtStartPar
(iv) Prove that \(c^2+s^2=1\). {[}Hint: Differentiate the function \(f:\mathbb{R}\to\mathbb{R}\) given by \(f(x)=c(x)^2+s(x)^2\) for each \(x\).{]}

\sphinxAtStartPar
(v)* Show that \(\exp(ix)=c(x)+is(x)\) for all \(x\in\mathbb{R}\), where \(\exp\) is the function from \sphinxhref{https://rosiesb.github.io/Analysis-Notes/5SSF.html\#eg:exp}{Example 5.7}. You should justify any rearrangements of terms in infinite series (MAS107 Theorem 4.18 may help).

\end{enumerate}
\phantomsection\label{\detokenize{Problems:id62}}\begin{enumerate}
\sphinxsetlistlabels{\arabic}{enumi}{enumii}{}{.}%
\setcounter{enumi}{61}
\item {} 
\sphinxAtStartPar
By using the Weierstrass \(M\)\sphinxhyphen{}test or otherwise, for each of the following series, determine whether it  converges uniformly on \(\mathbb{R}\) and  whether it converges uniformly on \([0,1]\).

\sphinxAtStartPar
(i) \(\displaystyle\sum_{n=1}^\infty \frac{1}{n^2 +x^2}\)

\sphinxAtStartPar
(ii) \(\displaystyle\sum_{n=0}^\infty \frac{(-1)^nx^{2n+1}}{(2n+1)!}\)

\sphinxAtStartPar
(iii) \(\displaystyle\sum_{n=1}^\infty \sin (nx)\)

\sphinxAtStartPar
(iv)* \(\displaystyle\sum_{n=1}^\infty \frac{\sin (nx)}{n}\)

\end{enumerate}
\phantomsection\label{\detokenize{Problems:id63}}\begin{enumerate}
\sphinxsetlistlabels{\arabic}{enumi}{enumii}{}{.}%
\setcounter{enumi}{62}
\item {} 
\sphinxAtStartPar
(i) Show the series \(\sum_{n=1}^\infty x^n\) converges uniformly for \(x\in [0,a]\) whenever \(0<a<1\).

\sphinxAtStartPar
(ii) Does the series converge uniformly on \([0,1)\)\textasciitilde{}? Explain.

\end{enumerate}
\phantomsection\label{\detokenize{Problems:id64}}\begin{enumerate}
\sphinxsetlistlabels{\arabic}{enumi}{enumii}{}{.}%
\setcounter{enumi}{63}
\item {} 
\sphinxAtStartPar
Prove that there is a function \(f\colon [0,2\pi] \rightarrow \mathbb{R}\) defined by

\end{enumerate}
\begin{equation*}
\begin{split}
f(x) = \sum_{n=1}^\infty \frac{\sin n x }{n^2}
\end{split}
\end{equation*}
\sphinxAtStartPar
and that this function is continuous.

\phantomsection\label{\detokenize{Problems:id65}}
\sphinxAtStartPar
65.* This question is about the controversial series presented by Fourier in 1807, when he first attempted to publish his solution to the diffusion equation (for a rectangular lamina).
\begin{equation}\label{equation:Problems:fourier}
\begin{split}\frac{4}{\pi}\left[\cos\left(\frac{\pi x}{2}\right)-\frac{1}{3}\cos\left(\frac{3\pi x}{2}\right)+\frac{1}{5}\cos\left(\frac{5\pi x}{2}\right)-\frac{1}{7}\cos\left(\frac{7\pi x}{2}\right)+\ldots\right]\end{split}
\end{equation}
\sphinxAtStartPar
While a perfectly good series, mathematical understanding in europe at the time was simply not equipped to make any sense of it. Completing this question may give you a sense of why Fourier’s contemporaries found it so disturbing.

\sphinxAtStartPar
\sphinxstylestrong{Disclaimer:} I am not a historian. For more on this fascinating subject, I recommend reading Chapter 1 of \sphinxhref{https://find.shef.ac.uk/permalink/f/1lephdb/44SFD\_ALMA\_DS21193257230001441}{A Radical Approach to Real Analysis by Bressoud}.

\sphinxAtStartPar
(i) Using \sphinxhref{https://www.desmos.com/calculator/bd3xikfhb0}{Desmos} or otherwise, determine (or make an educated guess) what function \(f:\mathbb{R}\to\mathbb{R}\) is represented by the Fourier series {\hyperref[\detokenize{Problems:equation-fourier}]{\sphinxcrossref{(1.2)}}}, and sketch its graph. What is its period?

\sphinxAtStartPar
(ii) Evaluate the series when \(x\) is an odd integer. What do you notice? Do you need to adjust your answer to (i)?

\sphinxAtStartPar
(iii) Calculate \(\text{Dom}(f')\), and write down a formula for \(f'\).

\sphinxAtStartPar
(iv) Now differentiate the series {\hyperref[\detokenize{Problems:equation-fourier}]{\sphinxcrossref{(1.2)}}} term by term. Does this series converge pointwise for any values of \(x\)? You may find Desmos or a similar graphing tool helpful for this part.


\subsection{Integration}
\label{\detokenize{Problems:integration}}\label{\detokenize{Problems:ch6prob}}\phantomsection\label{\detokenize{Problems:id66}}\begin{enumerate}
\sphinxsetlistlabels{\arabic}{enumi}{enumii}{}{.}%
\setcounter{enumi}{65}
\item {} 
\sphinxAtStartPar
Let \(f:[1,4]\to\mathbb{R}\); \(\displaystyle f(x)=\frac{1}{x}\). Let \(P\) be the partition consisting of points \(\left\{1,\frac{3}{2},2,4\right\}\).

\sphinxAtStartPar
(i) Compute \(L(f,P)\), \(U(f,P)\) and \(U(f,P)-L(f,P)\).

\sphinxAtStartPar
(ii) What happens to the value of \(U(f,P)-L(f,P)\) when we add the point \(3\) to the partition?

\sphinxAtStartPar
(iii) Find a partition \(P'\) of \([1,4]\) for which \(U(f,P')-L(f,P')<\frac{2}{5}\). {[}Hint: Mimick the proof of \sphinxhref{https://rosiesb.github.io/Analysis-Notes/6Int.html\#thm:mono}{Theorem 6.1}.{]}

\end{enumerate}
\phantomsection\label{\detokenize{Problems:id67}}\begin{enumerate}
\sphinxsetlistlabels{\arabic}{enumi}{enumii}{}{.}%
\setcounter{enumi}{66}
\item {} 
\sphinxAtStartPar
Let \(g:[0,1]\to\mathbb{R}\); \(\displaystyle g(x)=\left\{\begin{array}{cc} 1 & \text{for } 0\leq x<1 \\ 2 &\text{for } x=1 \end{array}\right.\).

\sphinxAtStartPar
(i) Show that \(L(g,P)=1\) for every partition \(P\) of \([0,1]\).

\sphinxAtStartPar
(ii) Construct a partition \(P\) for which \(U(g,P)<1+\frac{1}{10}\)

\sphinxAtStartPar
(iii) Given \(\varepsilon>0\), construct a partition \(P_\varepsilon\) such that \(U(g,P_\varepsilon)<1+\varepsilon\).

\end{enumerate}
\phantomsection\label{\detokenize{Problems:id68}}\begin{enumerate}
\sphinxsetlistlabels{\arabic}{enumi}{enumii}{}{.}%
\setcounter{enumi}{67}
\item {} 
\sphinxAtStartPar
Complete the proof of \sphinxhref{https://rosiesb.github.io/Analysis-Notes/6Int.html\#int-ind}{Proposition 6.1}. That is, show that for any closed, bounded interval \([a,b]\), the functions \(\mathbb{1}_{[a,b]}\), \(\mathbb{1}_{[a,b]}\), \(\mathbb{1}_{[a,b]}\) and \(\mathbb{1}_{[a,b]}\) are Riemann integrable, and

\end{enumerate}
\begin{equation*}
\begin{split}
\int_a^b\mathbb{1}_{[a,b]}(x)dx = \int_a^b\mathbb{1}_{(a,b)}(x)dx = \int_a^b\mathbb{1}_{[a,b)}(x)dx = \int_a^b\mathbb{1}_{(a,b]}(x)dx = b-a.
\end{split}
\end{equation*}\phantomsection\label{\detokenize{Problems:id69}}\begin{enumerate}
\sphinxsetlistlabels{\arabic}{enumi}{enumii}{}{.}%
\setcounter{enumi}{68}
\item {} 
\sphinxAtStartPar
\sphinxstyleemphasis{(Exam question).} Define step functions \(r,s\colon \mathbb{R} \rightarrow \mathbb{R}\) by
\begin{equation*}
\begin{split}
    r = \mathbb{1}_{[0,1)} + e \mathbb{1}_{[1,2)} + e^4 \mathbb{1}_{[2,3]}, \qquad  s = e \mathbb{1}_{[0,1]} + e^4 \mathbb{1}_{(1,2]} + e^9 \mathbb{1}_{(2,3]}.
    \end{split}
\end{equation*}
\sphinxAtStartPar
(i) Evaluate the integrals \(\displaystyle\int_0^3 r(x)dx\) and \(\displaystyle\int_0^3 s(x)dx\).  You may use results from the course to help you, but should indicate clearly when doing so.

\sphinxAtStartPar
(ii) Why is the function \(f\colon [0,3]\rightarrow \mathbb{R}\) defined by \(f(x)=e^{x^2}\) Riemann integrable?

\sphinxAtStartPar
(iii) Prove that
\begin{equation*}
\begin{split}
    1+e+e^4 \leq \int_0^3 e^{x^2}\ dx \leq e+e^4+e^9.
    \end{split}
\end{equation*}
\sphinxAtStartPar
{[}Hint: there’s no need to calculate the integral in the middle; use the previous parts to prove the inequalities.{]}

\end{enumerate}
\phantomsection\label{\detokenize{Problems:id70}}\begin{enumerate}
\sphinxsetlistlabels{\arabic}{enumi}{enumii}{}{.}%
\setcounter{enumi}{69}
\item {} 
\sphinxAtStartPar
Let \(f,g:[a,b]\to\mathbb{R}\) be integrable functions.

\sphinxAtStartPar
(i) Show that if \(P\) is a partition of \([a,b]\), then
\begin{equation*}
\begin{split}
    U(f+g,P) \leq U(f,P)+U(g,P)
    \end{split}
\end{equation*}
\sphinxAtStartPar
and
\begin{equation*}
\begin{split}
    L(f+g,P) \geq L(f,P)+L(g,P).
    \end{split}
\end{equation*}
\sphinxAtStartPar
Can you think of a particular example where these inequalities are strict?

\sphinxAtStartPar
(ii) Let \(P_1\) and \(P_2\) be partitions of \([a,b]\). Show that
\begin{equation*}
\begin{split}
    U(f+g,P_1\cup P_2) \leq U(f,P_1) + U(g,P_2)
    \end{split}
\end{equation*}
\sphinxAtStartPar
and
\begin{equation*}
\begin{split}
    L(f+g,P_1\cup P_2) \geq L(f,P_1) + L(g,P_2)
    \end{split}
\end{equation*}
\sphinxAtStartPar
{[}Hint: \sphinxhref{https://rosiesb.github.io/Analysis-Notes/6Int.html\#lem:ref}{Lemma 6.1} may be helpful.{]}

\sphinxAtStartPar
(iii) Deduce that
\begin{equation*}
\begin{split}
    L(f+g) \geq \int_a^bf(x)dx + \int_a^bg(x)dx
    \end{split}
\end{equation*}
\sphinxAtStartPar
and
\begin{equation*}
\begin{split}
    U(f+g) \leq \int_a^bf(x)dx + \int_a^bg(x)dx.
    \end{split}
\end{equation*}
\sphinxAtStartPar
Explain why this means \(f+g\) must be integrable, with
\begin{equation*}
\begin{split}
    \int_a^b(f(x)+g(x))dx = \int_a^bf(x)dx + \int_a^bg(x)dx.
    \end{split}
\end{equation*}
\end{enumerate}
\phantomsection\label{\detokenize{Problems:id71}}\begin{enumerate}
\sphinxsetlistlabels{\arabic}{enumi}{enumii}{}{.}%
\setcounter{enumi}{70}
\item {} 
\sphinxAtStartPar
Show that if \(k\in\mathbb{R}\) and \(f:[a,b]\to\mathbb{R}\) is integrable, then so is \(kf:[a,b]\to\mathbb{R}\); \(x\mapsto kf(x)\), and

\end{enumerate}
\begin{equation*}
\begin{split}
\int_a^b kf(x)dx = k\int_a^bf(x).
\end{split}
\end{equation*}
\sphinxAtStartPar
{[}Hint: It may help to treat the cases \(k\geq 0\) and \(k<0\) separately.{]}

\phantomsection\label{\detokenize{Problems:id72}}\begin{enumerate}
\sphinxsetlistlabels{\arabic}{enumi}{enumii}{}{.}%
\setcounter{enumi}{71}
\item {} 
\sphinxAtStartPar
Let \(f\) be bounded on a set \(A\subseteq\mathbb{R}\), let \(M=\sup\{f(x):x\in A\}, \; m=\inf\{f(x):x\in A\}\), and let \(M'=\sup\{|f(x)|:x\in A\}, \; \text{ and } \; m'=\inf\{|f(x)|:x\in A\}\).

\sphinxAtStartPar
(i) Show that \(M-m\geq M'-m'\).

\sphinxAtStartPar
(ii) Show that if \(f\) is integrable on \([a,b]\), then for any partition \(P\) of \([a,b]\),
\begin{equation*}
\begin{split}
    U(|f|,P)-L(|f|,P) \leq U(f,P) - L(f,P).
    \end{split}
\end{equation*}
\sphinxAtStartPar
(iii) Complete the proof of \sphinxhref{https://rosiesb.github.io/Analysis-Notes/6Int.html\#propsint}{Proposition 6.3}(iv) that is, show that \(|f|\) is integrable whenever \(f\) is.

\end{enumerate}
\phantomsection\label{\detokenize{Problems:id73}}\begin{enumerate}
\sphinxsetlistlabels{\arabic}{enumi}{enumii}{}{.}%
\setcounter{enumi}{72}
\item {} 
\sphinxAtStartPar
Let \(f\colon \mathbb{R} \rightarrow \mathbb{R}\) be a continuous function, and let \(a,b\colon \mathbb{R} \rightarrow \mathbb{R}\) be differentiable functions. Prove that

\end{enumerate}
\begin{equation*}
\begin{split}
\frac{d}{dx} \int_{a(x)}^{b(x)} f(t)dt = b'(x)f(b(x))-a'(x)f(a(x)).
\end{split}
\end{equation*}\phantomsection\label{\detokenize{Problems:id74}}\begin{enumerate}
\sphinxsetlistlabels{\arabic}{enumi}{enumii}{}{.}%
\setcounter{enumi}{73}
\item {} 
\sphinxAtStartPar
Define a function \(l:(0,\infty ) \rightarrow \mathbb{R}\) by \(\displaystyle l(x) = \int_1^x \frac{1}{t}dt\).

Show the following directly from the definition of \(l\) via an integral (that is, \sphinxstyleemphasis{without} using any properties of the function \(\ln\)).

\sphinxAtStartPar
(i) \(l\) is differentiable, and \(l'(x) = \frac{1}{x}\).

\sphinxAtStartPar
(ii) \(l(xy) = l(x)+l(y)\) for all \(x,y >0\).

\end{enumerate}
\phantomsection\label{\detokenize{Problems:id75}}
\sphinxAtStartPar
75.* Let \(f\colon [a,b]\rightarrow \mathbb{R}\) be Riemann integrable. Prove that there is a number \(x\in [a,b]\) such that
\begin{equation*}
\begin{split}
\int_a^x f(t)dt = \int_x^b f(t)dt.
\end{split}
\end{equation*}
\sphinxAtStartPar
{[}Hint: apply the intermediate value theorem to the function \(F:[a,b]\to\mathbb{R}\) given by
\begin{equation*}
\begin{split}
F(x) = \int_a^x f(t)dt - \int_x^b f(t)dt .\qquad ]
\end{split}
\end{equation*}\phantomsection\label{\detokenize{Problems:id76}}\begin{enumerate}
\sphinxsetlistlabels{\arabic}{enumi}{enumii}{}{.}%
\setcounter{enumi}{75}
\item {} 
\sphinxAtStartPar
(i) Let \(f\colon [a,b]\rightarrow \mathbb{R}\) be Riemann integrable. Suppose there are \(m,M\in \mathbb{R}\) such that \(m\leq f(x)\leq M\) for all \(x\in [a,b]\). Prove that there is a number \(\mu \in [m,M]\) such that
\begin{equation*}
\begin{split}
    \int_a^b f(x)\ dx = (b-a)\mu .
    \end{split}
\end{equation*}
\sphinxAtStartPar
(ii) Let \(f\colon [a,b]\rightarrow \mathbb{R}\) be continuous. Prove that there is some \(c \in [a,b]\) such that
\begin{equation*}
\begin{split}
    \int_a^b f(x)\ dx = (b-a)f(c) .
    \end{split}
\end{equation*}
\sphinxAtStartPar
{[}Hint: The intermediate value theorem is useful.{]}

\end{enumerate}
\phantomsection\label{\detokenize{Problems:id77}}\begin{enumerate}
\sphinxsetlistlabels{\arabic}{enumi}{enumii}{}{.}%
\setcounter{enumi}{76}
\item {} 
\sphinxAtStartPar
Let \(f\colon [a,b]\rightarrow \mathbb{R}\) be continuous, and let \(g\colon [a,b]\rightarrow [0,\infty )\) be integrable. Prove that there is some \(c \in [a,b]\) such that

\end{enumerate}
\begin{equation*}
\begin{split}
\int_a^b f(x)g(x)\ dx = f(c) \int_a^b g(x)\ dx .
\end{split}
\end{equation*}
\sphinxAtStartPar
Do we need the assumption \(g(x)\geq 0\)? Justify your answer.

\sphinxstepscope


\section{Solutions}
\label{\detokenize{Solutions-upto26:solutions}}\label{\detokenize{Solutions-upto26:sol}}\label{\detokenize{Solutions-upto26::doc}}
\sphinxAtStartPar
Complete up to Q26, excluding some homework questions.

\sphinxAtStartPar
Homework solutions will be released alongside the return of each piece of work.


\subsection{Preliminary problems}
\label{\detokenize{Solutions-upto26:preliminary-problems}}\label{\detokenize{Solutions-upto26:ch1sol}}
\sphinxAtStartPar
{\hyperref[\detokenize{Problems:p1}]{\sphinxcrossref{\DUrole{std,std-ref}{P1.}}}}
The correct statements are (ii) and (v).

\sphinxAtStartPar
In statement (iv), the quantifiers \(\forall\) and \(\exists\) are the wrong way round, while in (i) they are missing altogether. Statement (iii) asserts that last part, \(|x_n-l|<\varepsilon\) \(\forall n\geq N\), should hold for all \(\varepsilon\) and for all \(N\) — so again, there is an issue with the quantifiers.

\sphinxAtStartPar
Bonus exercise: prove that (iv) holds if and only if \((x_n)\) is the constant sequence \(x_n=l\) for all \(n\in\mathbb{N}\).


\bigskip\hrule\bigskip

\phantomsection\label{\detokenize{Solutions-upto26:p2sol}}
\sphinxAtStartPar
{\hyperref[\detokenize{Problems:p2}]{\sphinxcrossref{\DUrole{std,std-ref}{P2.}}}} \sphinxstyleemphasis{(Homework 1 question).}

\sphinxAtStartPar
Let \(\varepsilon>0\). According to the definition, we must prove there is an \(N\in\mathbb{N}\) for which \(\left|\frac{2n}{3n-1}-\frac{2}{3}\right|<\varepsilon\) whenever \(n\geq N\).

\sphinxAtStartPar
Now,
\begin{equation*}
\begin{split}
\left|x_n-\frac{2}{3}\right| = \left|\frac{2n}{3n-1}-\frac{2}{3}\right| = \frac{6n-(6n-2)}{3(3n-1)} = \frac{2}{3(3n-1)}.
\end{split}
\end{equation*}
\sphinxAtStartPar
This will be strictly less than \(\varepsilon\) whenever \(3n-1>\frac{2}{3\varepsilon}\)., or in other words, whenever
\begin{equation*}
\begin{split}
n>\frac{1}{3}\left(\frac{2}{3\varepsilon}+1\right)=\frac{2+3\varepsilon}{9\varepsilon}.
\end{split}
\end{equation*}
\sphinxAtStartPar
Let \(N\) be any integer greater than \(\frac{2+3\varepsilon}{9\varepsilon}\). Then \(\left|x_n-\frac{2}{3}\right|<\varepsilon\) for all \(n\geq N\), and we have proven that \(x_n\rightarrow\frac{2}{3}\) as \(n\rightarrow\infty\) using the definition.


\bigskip\hrule\bigskip

\phantomsection\label{\detokenize{Solutions-upto26:p3sol}}
\sphinxAtStartPar
{\hyperref[\detokenize{Problems:p3}]{\sphinxcrossref{\DUrole{std,std-ref}{P3.}}}} \sphinxstyleemphasis{(Homework 1 question).}

\sphinxAtStartPar
The Bolzano–Weierstrass theorem states that every bounded sequence has a convergent subsequence. Its proof combines the following two results:
\begin{itemize}
\item {} 
\sphinxAtStartPar
The monotone convergence theorem (Theorem 3.10 in the MAS107 notes), which states that bounded monotone sequences must converge.  More specifically, every monotone increasing sequence that is bounded above converges to its supremum, and every monotone decreasing sequence bounded below converges to its infimum.

\item {} 
\sphinxAtStartPar
Theorem 3.13 from MAS107: Every sequence has a monotone subsequence.

\end{itemize}

\sphinxAtStartPar
\sphinxstylestrong{Proof of Bolzano–Weierstrass:} 
If \((x_n)\) is a bounded sequence, then it has a monotone subsequence, \((x_{n_k})\), by Theorem 3.13. This subsequence must also be bounded since \((x_n)\) is bounded, and hence it converges by the monotone convergence theorem.


\bigskip\hrule\bigskip


\sphinxAtStartPar
{\hyperref[\detokenize{Problems:p4}]{\sphinxcrossref{\DUrole{std,std-ref}{P4.}}}}(i) Let \(a, b \geq 0\). Then,
\begin{equation*}
\begin{split}
\left(\sqrt{a+b}\right)^2 = a+b,
\end{split}
\end{equation*}
\sphinxAtStartPar
while,
\begin{equation*}
\begin{split}
\left(\sqrt{a}+\sqrt{b}\right)^2 = a + 2\sqrt{a}\sqrt{b} + b \geq a+b.
\end{split}
\end{equation*}
\sphinxAtStartPar
Therefore
\begin{equation*}
\begin{split}
\left(\sqrt{a}+\sqrt{b}\right)^2 \geq \left(\sqrt{a+b}\right)^2.
\end{split}
\end{equation*}
\sphinxAtStartPar
Since \(\sqrt{a}\), \(\sqrt{b}\) and \(\sqrt{a+b}\) are all non\sphinxhyphen{}negative and the square root function is increasing, we can square root both sides to get
\begin{equation*}
\begin{split}
\sqrt{a}+\sqrt{b}\geq\sqrt{a+b}.
\end{split}
\end{equation*}
\sphinxAtStartPar
(ii)  Note that the inequality we have been asked to prove is symmetric in \(a\) and \(b\), in the sense that exchanging the roles of \(a\) and \(b\) does not affect the value of either side.

\sphinxAtStartPar
This means we can assume without loss of generality that \(\sqrt{|a|}\geq\sqrt{|b|}\). Then,
\begin{equation*}
\begin{split}
\left|\sqrt{|a|} - \sqrt{|b|}\right| = \sqrt{|a|}-\sqrt{|b|},
\end{split}
\end{equation*}
\sphinxAtStartPar
and we need only show that
\begin{equation}\label{equation:Solutions-upto26:eq:sqrta-sqrtb}
\begin{split}\sqrt{|a|} - \sqrt{|b|} \leq \sqrt{|a - b|}.\end{split}
\end{equation}
\sphinxAtStartPar
We use a trick similar to the proof of Corollary 1.1, and write
\begin{equation*}
\begin{split}
\sqrt{|a|} = \sqrt{\big|(a-b) + b\big|}.
\end{split}
\end{equation*}
\sphinxAtStartPar
By the triangle inequality, we have that \(|a| = \big|(a-b) + b\big| \leq |a-b|+|b|\).

\sphinxAtStartPar
Since the square root function is increasing, we can square root both sides to get
\begin{equation*}
\begin{split}
\sqrt{|a|} \leq \sqrt{|a-b|+|b|}.
\end{split}
\end{equation*}
\sphinxAtStartPar
Applying the result from (i) to the right\sphinxhyphen{}hand side, it follows that
\begin{equation*}
\begin{split}
\sqrt{|a|} \leq \sqrt{|a-b|+|b|} \leq \sqrt{|a-b|}+\sqrt{|b|}.
\end{split}
\end{equation*}
\sphinxAtStartPar
Equation {\hyperref[\detokenize{Solutions-upto26:equation-eq-sqrta-sqrtb}]{\sphinxcrossref{(2.1)}}} now follows by subtracting \(\sqrt{|b|}\) from both sides.

\sphinxAtStartPar
(iii) Let \((a_n)\) be a real sequence converging to \(l\in\mathbb{R}\). By (ii),
\begin{equation*}
\begin{split}
\left|\sqrt{|a_{n}|}-\sqrt{|l|}\right| \leq \sqrt{\big|a_n-l\big|}
\end{split}
\end{equation*}
\sphinxAtStartPar
for all \(n\in\mathbb{N}\).

\sphinxAtStartPar
Let \(\varepsilon>0\). Since \(a_n\rightarrow l\), there is \(N\in\mathbb{N}\) such that \(|a_n-l|<\varepsilon^2\) whenever \(n\geq N\). Hence for \(n\geq N\),
\begin{equation*}
\begin{split}
\big|\sqrt{|a_{n}|}-\sqrt{|l|}\big| \leq \sqrt{\varepsilon^2} = \varepsilon.
\end{split}
\end{equation*}
\sphinxAtStartPar
Thus \(\displaystyle\lim_{n\rightarrow\infty}\sqrt{|a_n|} = \sqrt{|l|}\).


\bigskip\hrule\bigskip


\sphinxAtStartPar
{\hyperref[\detokenize{Problems:p5}]{\sphinxcrossref{\DUrole{std,std-ref}{P5.}}}} When it exists, the supremum of a set \(A\subseteq\mathbb{R}\) is defined to be the unique number \(\alpha\in\mathbb{R}\) such that
\begin{itemize}
\item {} 
\sphinxAtStartPar
\(\alpha\) is an upper bound for \(A\), and

\item {} 
\sphinxAtStartPar
if \(M\) is any other upper bound for \(A\), then \(\alpha\leq M\).

\end{itemize}

\sphinxAtStartPar
We write \(\sup A\) for the supremum of \(A\), when it exists.

\sphinxAtStartPar
The definition of the infimum is similar: when it exists, the infimum of \(A\) is the unique number \(\beta\in\mathbb{R}\) such that

\sphinxAtStartPar
(i) \(\beta\) is a lower bound for \(A\), and

\sphinxAtStartPar
(ii) if \(L\) is any other lower bound for \(A\), then \(\alpha\geq L\).

\sphinxAtStartPar
We write \(\inf A\) for the infimum of \(A\), when it exists.

\sphinxAtStartPar
The axiom of completeness for the real numbers says that every non\sphinxhyphen{}empty bounded above subset of \(\mathbb{R}\) has a supremum. Equivalently, it says that every non\sphinxhyphen{}empty bounded below subset of \(\mathbb{R}\) has an infimum.

\sphinxAtStartPar
For more details, see page 32 of your Semester 2 MAS107 notes.


\bigskip\hrule\bigskip


\sphinxAtStartPar
{\hyperref[\detokenize{Problems:p6}]{\sphinxcrossref{\DUrole{std,std-ref}{P6.}}}}(i) \(\displaystyle\sup\left\{\frac{m}{n}:m,n\in\mathbb{N} \text{ s.t } m<n\right\}=1\), \(\displaystyle\inf\left\{\frac{m}{n}:m,n\in\mathbb{N} \text{ s.t } m<n\right\}=0\).

\sphinxAtStartPar
(ii) \(\displaystyle\sup\left\{\frac{(-1)^m}{n}:m,n\in\mathbb{N} \text{ s.t } m<n\right\}=\frac{1}{3}\), \(\displaystyle\inf\left\{\frac{(-1)^m}{n}:m,n\in\mathbb{N} \text{ s.t } m<n\right\}=-\frac{1}{2}\).

\sphinxAtStartPar
(iii) \(\displaystyle\sup\left\{\frac{n}{3n + 1} :n\in\mathbb{N}\right\}=\frac{1}{3}\), \(\displaystyle\inf\left\{\frac{n}{3n + 1} :n\in\mathbb{N}\right\}=\frac{1}{4}\).


\bigskip\hrule\bigskip


\sphinxAtStartPar
{\hyperref[\detokenize{Problems:p7}]{\sphinxcrossref{\DUrole{std,std-ref}{P7.}}}}(i) This is false — for a counter\sphinxhyphen{}example, take any singleton set. Then \(\sup A=\inf A=1\).

\sphinxAtStartPar
A correct version of the statement would be \(\inf A \leq \sup A\).

\sphinxAtStartPar
(ii) True. The analogous statement for \(\sup\) was called the characteristic property of the supremum in your MAS107 lecture notes — see Lemma 2.8 on page 36.

\sphinxAtStartPar
(iii) True. Since \(\sup B\) is an upper bound for \(B\), and \(A\) is a subset of \(B\), \(\sup B\) must be an upper bound for \(A\). But \(\sup A\) is a the least upper bound for \(A\), hence \(\sup A\leq \sup B\). The analogous property for \(\inf\) is that \(\inf A\geq\inf B\) whenever \(A\subseteq B\).

\sphinxAtStartPar
(iv) False. For a counter\sphinxhyphen{}example, take \(B=\left\{\frac{1}{n}:n\in\mathbb{N}\right\}\), for which \(\inf B=0\).

\sphinxAtStartPar
(v) True. Let \(s=\max\{\sup A,\sup B\}\). Then \(s\geq \sup A\) and \(s\geq \sup B\), so \(s\) is an upper bound for both \(A\) and \(B\). Therefore, \(s\) is an upper bound for \(A\cup B\). But \(\sup(A\cup B)\) is the least upper bound of \(A\cup B\), and so \(s\geq\sup(A\cup B)\). Also, since \(A\) and \(B\) are both subsets of \(A\cup B\), we have by statement (iii) that \(s=\max\{\sup A,\sup B)\leq\sup(A\cup B)\). Hence \(\max\{\sup A,\sup B)=\sup(A\cup B)\).


\subsection{Limits of functions}
\label{\detokenize{Solutions-upto26:limits-of-functions}}\label{\detokenize{Solutions-upto26:ch2sol}}
\sphinxAtStartPar
{\hyperref[\detokenize{Problems:id1}]{\sphinxcrossref{\DUrole{std,std-ref}{1.}}}} One has to identify the real numbers where the given formula does not make sense, usually because of a zero in a denominator somewhere, and exclude them.

\sphinxAtStartPar
(i) \(A=\mathbb{R} \setminus \{0, -1\}\).

\sphinxAtStartPar
(ii) \(g_{2}(x) = \displaystyle\frac{(x-1)(x+4)}{(x-1)(x+2)(x+3)}\), so \(A=\mathbb{R} \setminus \{1, -2, -3\}\).

\sphinxAtStartPar
(iii) \(g_{3}(x) = \displaystyle\frac{x+4}{(x+2)(x+3)}\), so \(A=\mathbb{R} \setminus \{-2, -3\}\).

\sphinxAtStartPar
(iv) \(A=\mathbb{R} \setminus \{1\}\).

\sphinxAtStartPar
(v) \(A=\mathbb{R} \setminus \{0\}\).


\bigskip\hrule\bigskip

\phantomsection\label{\detokenize{Solutions-upto26:id1}}
\sphinxAtStartPar
{\hyperref[\detokenize{Problems:id2}]{\sphinxcrossref{\DUrole{std,std-ref}{2.}}}} \sphinxstyleemphasis{(Homework 1 question).}

\begin{sphinxadmonition}{note}{Note:}
\sphinxAtStartPar
This question was ambiguous about whether or not proofs are required. Strictly speaking, “being able to prove rigorously which real numbers are the limit points of a given set” is not a key learning outcome for this module, so the proofs included below are largely for interest. What is important is that you know the definition of a limit point, and understand it well enough that you can identify which real numbers are limit points of a given set, and which are not.
\end{sphinxadmonition}

\sphinxAtStartPar
(i) \(X=(0,1)\cup[2,3)\cup\{4,5\}\). Set of limit points: \(L=[0,1]\cup[2,3]\).
\subsubsection*{Proof (dropdown)}

\sphinxAtStartPar
Let \(L\) be the set of all limit points of \(X\). We prove that \(L=[0,1]\cup[2,3]\) by proving \(L\subseteq[0,1]\cup[2,3]\) and \(L\supseteq[0,1]\cup[2,3]\).

\sphinxAtStartPar
\((\supseteq)\) Let \(a\in[0,1]\cup[2,3]\). We prove \(a\) is a limit point of \(X=(0,1)\cup[2,3)\cup\{4,5\}\) by explicitly writing down a sequence \((x_n)\) in \(X\setminus\{a\}\) with limit \(a\). In fact, \(x_n=a+\frac{1}{n+3}\) works for any choice \(a\in[0,1)\cup[2,3)\). If \(a=0\) or \(a=3\), then use \(x_n=a-\frac{1}{n}\) instead. Either way, we have shown that \(a\in L\).

\sphinxAtStartPar
\((\subseteq)\) If \(a\in L\), then there is a sequence \((x_n)\) in \(X\setminus\{a\}\), with \(a=\lim_{n\rightarrow\infty}x_n\). Since \(x_n\in X\) for all \(n\in\mathbb{N}\), we must have \(0<x_n\leq 5\) for all \(n\in\mathbb{N}\), and so \(0\leq a\leq 5\).

\sphinxAtStartPar
If \(a\in(1,2)\), then the definition of convergence says that for any \(\varepsilon>0\), we can find \(N\in\mathbb{N}\) such that \(x_n\in(a-\varepsilon,a+\varepsilon)\) for all \(n\geq N\). We know this holds for any \(\varepsilon>0\), so in particular we can choose \(\varepsilon>0\) such that \((a-\varepsilon,a+\varepsilon)\subseteq(1,2)\). For this choice of \(\varepsilon\) and resulting \(N\), we have that \(x_n\in(1,2)\) for all \(n\geq N\). But this contradicts \((x_n)\) being a sequence in \(X\). So \(a\notin (1,2)\).

\sphinxAtStartPar
Finally, suppose \(a>3\). By a similar argument, we can choose \(\varepsilon>0\) small enough so that \((a-\varepsilon,a+\varepsilon)\subseteq(3,\infty)\). Since \(x_n\rightarrow a\), we know there exists \(N\in\mathbb{N}\) such that \(|x_n-a|<\varepsilon\) for all \(n\geq N\). Therefore, for all \(n\geq N\), we have \(x_n\in(a-\varepsilon,a+\varepsilon)\subseteq (3,\infty)\). In other words, \(x_n>3\) for all \(n\geq N\). The only elements of \(X\) that are strictly larger than \(3\) are \(4\) and \(5\). This means that \(x_n\in\{4,5\}\) for all \(n\geq N\). But then, since \((x_n)\) converges to \(a\), it must be eventually constant and equal to \(a\), contradicting \(x_n\neq a\) for all \(n\in\mathbb{N}\). So \(a\leq 3\).

\sphinxAtStartPar
It follows that \(a\in[0,1]\cup[2,3]\).\(\square\)

\sphinxAtStartPar
(ii)  \(X=\mathbb{Z}\). Set of limit points: \(L=\emptyset\).
\subsubsection*{Proof (dropdown)}

\sphinxAtStartPar
If \(a\) is a limit point of \(\mathbb{Z}\), then there is a sequence \((x_n)\) in \(\mathbb{Z}\), with \(x_n\neq a\) for all \(n\in\mathbb{N}\), and \(\lim_{n\rightarrow\infty}x_n=a\). But all convergent sequences in \(\mathbb{Z}\) are eventually constant, so this is impossible.  \(\square\)

\sphinxAtStartPar
(iii)\(X=\mathbb{R}\setminus\mathbb{Z}\). Set of limit points:  \(L=\mathbb{R}\).
\subsubsection*{Proof (dropdown)}

\sphinxAtStartPar
If \(a\in\mathbb{Z}\), then \(x_n:=a+\frac{1}{2n}\) defines a sequence in \(X\) with limit \(a\), and clearly \(x_n\neq a\) for all \(n\in\mathbb{N}\). This shows \(\mathbb{Z}\subseteq L\). If \(a\in\mathbb{R}\setminus\mathbb{Z}\), then \(a\) must lie in some open interval \((k,k+1)\), where \(k\in\mathbb{Z}\) (in fact \(k=\lfloor x\rfloor\)).

\begin{figure}[htbp]
\centering
\capstart

\noindent\sphinxincludegraphics[width=500\sphinxpxdimen]{{2iii}.png}
\caption{Sketch to decide/justify choice \(\delta:=\min\{x-a,k+1-a\}\) (Problem 2(iii))}\label{\detokenize{Solutions-upto26:iii}}\end{figure}

\sphinxAtStartPar
Let \(\delta=\min\{x-a,k+1-a\}\) (this choice is strongly motivated by a sketch — see \hyperref[\detokenize{Solutions-upto26:iii}]{Fig.\@ \ref{\detokenize{Solutions-upto26:iii}}}.). Then \(x_n:=a+\frac{\delta}{n}\) defines a sequence in \((k,k+1)\), with limit \(a\), and such that \(x_n\neq a\) for all \(n\in\mathbb{N}\).\(\square\)

\sphinxAtStartPar
(iv) \(X=\{x\in\mathbb{Q}:0<x<1\}\). Set of limit points: \(L=[0,1]\)
\subsubsection*{Proof (dropdown)}

\sphinxAtStartPar
If \((x_n)\) is a convergent sequence in \(X\), then \(0<x_n<1\) for all \(n\in\mathbb{N}\), and so \(0\leq\lim_{n\rightarrow\infty}x_n\leq 1\). This shows that \(L\subseteq [0,1]\). On the other hand, if \(a\in[0,1]\), then by density of the rationals, for all \(n\in\mathbb{N}\) there is a rational number \(x_n\) satisfying \(a<x_n<a+\frac{1}{n}\). Then, \(x_n\neq a\) for all \(n\in\mathbb{N}\), and by the squeeze theorem, \(x_n\rightarrow a\) as \(n\rightarrow\infty\). Also, since \(a\in[0,1]\), with the exception of the case \(a=1\) (treated separately below), we will eventually have \(0<x_n<1\) for all \(n\in\mathbb{N}\) sufficiently large. In other words, removing a finite number of initial terms if necessary, we have found a sequence in \(X\setminus\{a\}\) with limit \(a\).

\sphinxAtStartPar
For the case \(a=1\), use the same argument, but with sequence \((x_n)\) of rational numbers chosen so that  \(1-\frac{1}{n}<x_n<1\), for each \(n\in\mathbb{N}\).\(\square\)

\sphinxAtStartPar
(v) \(X=\displaystyle\left\{\frac{1}{n}:n\in\mathbb{N}\right\}\).  \(L=\{0\}\)
\subsubsection*{Proof (dropdown)}

\sphinxAtStartPar
That \(0\in L\) follows by taking the sequence \(x_n=\frac{1}{n}\). On the other hand, if \(a\in L\), then there is a sequence \((x_n)\) in \(X\setminus\{a\}\) with limit \(a\). All elements of \(X\) are strictly positive, so \(x_n>0\) for all \(n\in\mathbb{N}\), and hence \(a\geq 0\). Suppose that \(a>0\). Then by definition of convergence, there is \(N\in\mathbb{N}\) such that \(|x_n-a|<\frac{a}{2}\) for all \(n\geq N\). But then \(x_n\in\left(\frac{a}{2},\frac{3a}{2}\right)\) for all \(n\geq N\). There are only finitely many elements of \(X\) that lie in this interval, and we know \((x_n)\) converges to \(a\). So, the only option is that \((x_n)\) is eventually constant and equal to \(a\). This is a contradiction, since \((x_n)\) is a sequence in \(X\setminus\{a\}\). It follows that \(a=0\).\(\square\)


\bigskip\hrule\bigskip

\phantomsection\label{\detokenize{Solutions-upto26:id2}}
\sphinxAtStartPar
{\hyperref[\detokenize{Problems:id3}]{\sphinxcrossref{\DUrole{std,std-ref}{3.}}}} \sphinxstyleemphasis{(Homework 1 question).}

\sphinxAtStartPar
(i) We are given \(f:\mathbb{R}\to\mathbb{R}\); \(f(x)=4x+7\). Intuition tells us that
\begin{equation*}
\begin{split}
\lim_{x\rightarrow 2}f(x) = 4\cdot 2+7 = 15.
\end{split}
\end{equation*}
\sphinxAtStartPar
To prove this, let \(\varepsilon>0\). We seek \(\delta>0\) such that \(0<|x-2|<\delta\) implies \(|f(x)-15|<\varepsilon\).

\sphinxAtStartPar
Note that
\begin{equation*}
\begin{split}
|f(x)-15| = |4x+7-15| = |4x-8| = 4|x-2|.
\end{split}
\end{equation*}
\sphinxAtStartPar
Therefore, putting \(\delta=\frac{\varepsilon}{4}\), we get that whenever \(0<|x-2|<\delta\),
\begin{equation*}
\begin{split}
|f(x)-17| = 4|x-2| < 4\cdot\frac{\varepsilon}{4} = \varepsilon.
\end{split}
\end{equation*}
\sphinxAtStartPar
For the limit as \(x\rightarrow 0\), we claim that \(\lim_{x\rightarrow 0}f(x)=7\). To prove this, note that \(|f(x)-7|=4|x|\). This means that given \(\varepsilon>0\), if \(0<|x|<\frac{\varepsilon}{4}\) then
\begin{equation*}
\begin{split}
|f(x)-7|<4\cdot\frac{\varepsilon}{4} = \varepsilon.
\end{split}
\end{equation*}
\sphinxAtStartPar
It follows that \(\lim_{x\rightarrow 0}f(x)\) exists and is equal to \(7\).

\sphinxAtStartPar
(ii) We have \(f:\{0\}\cup[1,3]\to\mathbb{R}\); \(f(x)=3x^2-1\), so using intuition only, \(\lim_{x\rightarrow 2} f(x)=3\cdot 4-1=11\).

\sphinxAtStartPar
Let \(\varepsilon>0\). We seek \(\delta>0\) such that for all \(x\in\{0\}\cup[1,3]\),
\begin{equation*}
\begin{split}
0<|x-2|<\delta \; \Rightarrow \; |f(x)-11|<\varepsilon.
\end{split}
\end{equation*}
\sphinxAtStartPar
Now, for \(x\in\{0\}\cup[1,3]\),
\begin{equation*}
\begin{split}
|f(x)-11| = |3x^2-12| = 3|x^2-4| = 3|x-2||x+2| \leq 15|x-2|.
\end{split}
\end{equation*}
\sphinxAtStartPar
(Here, we have used the fact that \(|x+2| = x+2 \leq 5\), for \(x\in\{0\}\cup[1,3]\).)

\sphinxAtStartPar
Hence we can let \(\delta:=\frac{\varepsilon}{15}\) and conclude that if \(x\in\{0\}\cup[1,3]\) and \(0<|x-2|<\frac{1}{15}\), then
\begin{equation*}
\begin{split}
|f(x)-11|<15\cdot\frac{\varepsilon}{15} =\varepsilon.
\end{split}
\end{equation*}
\sphinxAtStartPar
For this \(f\), \(\lim_{x\rightarrow 0}f(x)\) is not defined, since \(0\) is not a limit point of the domain of \(f\).

\phantomsection\label{\detokenize{Solutions-upto26:iiisol}}
\sphinxAtStartPar
(iii) Finally, let \(f:(0,\infty)\to\mathbb{R}\); \(f(x)=x+\frac{1}{x}\). The limit of \(f(x)\) as \(x\rightarrow 2\) ought to be \(2+\frac{1}{2}=\frac{5}{2}\).

\sphinxAtStartPar
Let \(\varepsilon>0\), and consider
\begin{equation*}
\begin{split}
\left|f(x)-\frac{5}{2}\right| = \left|x+\frac{1}{x}-\frac{5}{2}\right| = \left|\frac{2x^2+2-5x}{2x}\right| = \left|\frac{(x-2)(2x-1)}{2x}\right| = |x-2|\left|1-\frac{1}{2x}\right|.
\end{split}
\end{equation*}
\sphinxAtStartPar
Observe that if \(x\in(0,\infty)\) and \(|x-2|<1\), then \(0<1-\frac{1}{2x}<1\), and so
\begin{equation*}
\begin{split}
\left|f(x)-\frac{5}{2}\right| = |x-2|\left|1-\frac{1}{2x}\right| < |x-2|.
\end{split}
\end{equation*}
\sphinxAtStartPar
To make sure the above inequality holds \sphinxstylestrong{and} is bounded above by \(\varepsilon\), let \(\delta:=\min\{\varepsilon,1\}\).

\sphinxAtStartPar
Suppose that \(x\in(0,\infty)\) with \(0<|x-2|<\delta\). Then by our choice of \(\delta\), \(0<|x-2|<1\) and \(0<|x-2|<\varepsilon\) both hold simultaneously. It follows that
\begin{equation*}
\begin{split}
\left|f(x)-\frac{5}{2}\right|<|x-2|<\varepsilon,
\end{split}
\end{equation*}
\sphinxAtStartPar
and hence \(\lim_{x\rightarrow 2}f(x)=\frac{5}{2}\) is proved.

\sphinxAtStartPar
For the last part, note that \(f(x)=x+\frac{1}{x}\) does not converge to a finite limit as \(x\rightarrow 0\). In fact, \(\lim_{x\rightarrow 0}f(x)=\infty\). One way to prove this is to show that \(f(x)\) surpasses any possible bound as \(x\rightarrow 0\). Note that
\begin{equation*}
\begin{split}
f(x) = x+\frac{1}{x} > \frac{1}{x}.
\end{split}
\end{equation*}
\sphinxAtStartPar
Therefore, given an arbitrary number \(K>0\), we can take \(0<x<\frac{1}{K}\) to ensure that
\begin{equation*}
\begin{split}
f(x) = x+\frac{1}{x} > \frac{1}{x} > K.
\end{split}
\end{equation*}
\sphinxAtStartPar
This proves that \(\lim_{x\rightarrow 0} f(x) = \infty\), for this choice of \(f\).


\bigskip\hrule\bigskip


\sphinxAtStartPar
{\hyperref[\detokenize{Problems:id4}]{\sphinxcrossref{\DUrole{std,std-ref}{4.}}}} We had \(A= \mathbb{R} \setminus \{1, -2, -3\}\), and \(f_2:A\to\mathbb{R}\); \(\displaystyle g_{2}(x)= \frac{(x + 4)}{(x + 2)(x + 3)}\).

\sphinxAtStartPar
It is considerably easier to use the sequential criterion for functional limits (Theorem 2.1) than it is to proceed directly using the Definition 2.1. We include both methods, for completeness.

\sphinxAtStartPar
\sphinxstylestrong{Method 1: Sequences:} 
Let \((x_n)\) be a sequence in \(A\) converging to \(1\). Then, using algebra of limits for real sequences,
\begin{equation*}
\begin{split}
\lim_{n\to\infty}f_2(x_n)= \lim_{n\to\infty} \frac{(x_n + 4)}{(x_n + 2)(x_n + 3)} =  \frac{(1 + 4)}{(1 + 2)(1 + 3)} = \frac{5}{12},
\end{split}
\end{equation*}
\sphinxAtStartPar
where the last equality is using algebra of limits. So \(\displaystyle\lim_{x \rightarrow 1}g_{2}(x) = \frac{5}{12}\).

\sphinxAtStartPar
Let \(x_n=-2 + \frac{1}{n}\), so that \((x_n)\) is a sequence in \(A\) converging to \(-2\). Then we see that \((f_2(x_n))\) diverges to \(+\infty\) and so \(\lim_{x \rightarrow -2}g_{2}(x)\)  does not exist.

\sphinxAtStartPar
Similarly, considering \(x_n=-3 + \frac{1}{n}\),  we see that \(\lim_{x \rightarrow -3}g_{2}(x)\) does not exist.

\sphinxAtStartPar
\sphinxstylestrong{Method 2: \((\varepsilon-\delta)\) criterion}: 
Intuitively speaking, the limit as \(x\rightarrow 1\) “should” be \(\frac{(1 + 4)}{(1 + 2)(1 + 3)}=\frac{5}{12}\).

\sphinxAtStartPar
We prove \(\lim_{x\rightarrow 1}f_2(x) = \frac{5}{12}\).

\sphinxAtStartPar
Let \(\varepsilon>0\). We wish to find \(\delta>0\) so that if \(x\in A\) and \(0<|x-1|<\delta\), then \(\left|f_2(x)-\frac{5}{12}\right| <\varepsilon\).

\sphinxAtStartPar
Now,
\begin{align*}
\left|f_2(x)-\frac{5}{12}\right| &= \left|\frac{(x + 4)}{(x + 2)(x + 3)}-\frac{5}{12}\right| \\
&= \left|\frac{12(x+4)-5(x+2)(x+3)}{(x+2)(x+3)}\right| \\
&= \left|\frac{18-5x^2-13x}{(x+2)(x+3)}\right| \\
&= \left|\frac{(1-x)(5x+18)}{(x+2)(x+3)}\right|  = |x-1|\cdot\frac{|5x+18|}{|x+2||x+3|}.
\end{align*}
\sphinxAtStartPar
We are going to constrain \(x\) to be very close to \(1\). In particular, we can assume \(0<x<2\), so
\begin{equation*}
\begin{split}
\left|f_2(x)-\frac{5}{12}\right| =|x-1|\cdot\frac{|5x+18|}{|x+2||x+3|} < |x-1| \frac{5(2)+18}{(2)(3)} = \frac{14}{3}|x-1|.
\end{split}
\end{equation*}
\sphinxAtStartPar
Finally, we choose \(\delta:= \frac{3}{14}\varepsilon\). Then, for \(0<|x-1|< \delta\),
\begin{equation*}
\begin{split}
\left|f_2(x)-\frac{5}{12}\right| <  \frac{14}{3}|x-1| <  \frac{14}{3}\cdot \frac{3}{14}\varepsilon = \varepsilon.
\end{split}
\end{equation*}

\bigskip\hrule\bigskip


\sphinxAtStartPar
{\hyperref[\detokenize{Problems:id5}]{\sphinxcrossref{\DUrole{std,std-ref}{5.}}}} The first part follows by using the definition of the limit of a function in terms of limits of sequences, and then applying the result of {\hyperref[\detokenize{Problems:p4}]{\sphinxcrossref{\DUrole{std,std-ref}{P4 (iii)}}}}..

To be precise let \((x_{n})\) be any sequence in \(\mathbb{R} \setminus \{a\}\) that converges to \(a\). Then since we are given that \(\lim_{x \rightarrow a} f(x) = l\), we must have that \(\lim_{n\rightarrow\infty} f(x_{n}) = l\). But then by {\hyperref[\detokenize{Problems:p4}]{\sphinxcrossref{\DUrole{std,std-ref}{P4 (iii)}}}}., we have \(\lim_{n\rightarrow\infty} \sqrt{f(x_{n}}) = \sqrt{l}\). So by definition of the limit of a function, \(\lim_{x \rightarrow a} \sqrt{f(x)} = \sqrt{l}\).

Using algebra of limits,  \(\lim_{x \rightarrow 1}\displaystyle\frac{x+1}{x^{2}}=\frac{1+1}{1^2}=2\) and
then by the first part \(\lim_{x \rightarrow 1}\sqrt{\displaystyle\frac{x+1}{x^{2}}} = \sqrt{2}\).


\bigskip\hrule\bigskip


\sphinxAtStartPar
{\hyperref[\detokenize{Problems:id6}]{\sphinxcrossref{\DUrole{std,std-ref}{6.}}}} Let \(f:A\to \mathbb{R}\) and suppose that \(\lim_{x \rightarrow a} f(x) = l\) and \(\lim_{x \rightarrow a} f(x) = l'\). Then given any sequence \((x_{n})\) in \(A \setminus \{a\}\) that converges to \(a\), we have \(\lim_{n\rightarrow\infty} f(x_{n}) = l\) and also \(\lim_{n\rightarrow\infty} f(x_{n}) = l'\). But then \(l = l'\), by uniqueness of limits.


\bigskip\hrule\bigskip


\sphinxAtStartPar
{\hyperref[\detokenize{Problems:id7}]{\sphinxcrossref{\DUrole{std,std-ref}{7.}}}}
When \(x > 0, \displaystyle\frac{|x|}{x} = \displaystyle\frac{x}{|x|} = \displaystyle\frac{x}{x} = 1 = \text{sgn}(x)\), and when \(x < 0, \displaystyle\frac{|x|}{x} = \displaystyle\frac{x}{|x|} = -\displaystyle\frac{x}{x} = -1 = \text{sgn}(x)\). So \(\text{sgn}(x) = \displaystyle\frac{|x|}{x} = \displaystyle\frac{x}{|x|}\), when \(x\neq 0\).

Consider the sequences \(x_n=\frac{1}{n}\) and \(y_n=-\frac{1}{n}\). If \(\lim_{x\rightarrow 0}\text{sgn}(x)\) existed, then we would have
\begin{equation*}
\begin{split}
\lim_{n\rightarrow\infty}\text{sgn}(x_n)=\lim_{n\rightarrow\infty}\text{sgn}(y_n)=\lim_{x\rightarrow 0}\text{sgn}(x).
\end{split}
\end{equation*}
\sphinxAtStartPar
But in fact, \(\text{sgn}(x_n)=1\) and \(\text{sgn}(y_n)=-1\) for all \(n\in\mathbb{N}\), so
\begin{equation*}
\begin{split}
\lim_{n\rightarrow\infty}\text{sgn}(x_n)=1\neq-1=\lim_{n\rightarrow\infty}\text{sgn}(y_n).
\end{split}
\end{equation*}
\sphinxAtStartPar
So \(\text{sgn}\) does not converge as \(x\rightarrow 0\).

\sphinxAtStartPar
For the left and right limits at \(0\):
\begin{itemize}
\item {} 
\sphinxAtStartPar
The left limit is \(\displaystyle\lim_{x \rightarrow 0^-} \text{sgn}(x) = -1\), since for any sequence \((x_n)\) approaching \(0\) from the left, we have \(\text{sgn}(x_n) = -1\) for all \(n\).

\item {} 
\sphinxAtStartPar
The right limit is \(\displaystyle\lim_{x \rightarrow 0^+} \text{sgn}(x) = 1\),  since for any sequence \((x_n)\) approaching \(0\) from the right, we have \(\text{sgn}(x_n) = 1\) for all \(n\).

\end{itemize}


\bigskip\hrule\bigskip

\phantomsection\label{\detokenize{Solutions-upto26:id3}}
\sphinxAtStartPar
{\hyperref[\detokenize{Problems:id8}]{\sphinxcrossref{\DUrole{std,std-ref}{8.}}}} To appear (Homework 2 question)


\bigskip\hrule\bigskip

\phantomsection\label{\detokenize{Solutions-upto26:id4}}
\sphinxAtStartPar
{\hyperref[\detokenize{Problems:id9}]{\sphinxcrossref{\DUrole{std,std-ref}{9.}}}} To appear (Homework 2 question)


\bigskip\hrule\bigskip


\sphinxAtStartPar
{\hyperref[\detokenize{Problems:id10}]{\sphinxcrossref{\DUrole{std,std-ref}{10.}}}} The largest subset of \(\mathbb{R}\) for which the formula \(f(x)=x\sin\left(\frac{1}{x}\right)\) makes sense is \(A = \mathbb{R} \setminus \{0\}\).

We claim that \(\displaystyle\lim_{x \rightarrow 0} x \sin\left(\frac{1}{x}\right) = 0\).

To see this, let \((x_{n})\) be an arbitrary sequence in \(A\) that converges to \(0\). The sine function takes values only in the interval \([-1,1]\), and so
\begin{equation*}
\begin{split}
\left|x_n\sin\left(\frac{1}{x_n}\right)\right| \leq |x_n|
\end{split}
\end{equation*}
\sphinxAtStartPar
for each \(n\in\mathbb{N}\). Put another way,
\begin{equation*}
\begin{split}
-|x_{n}| \leq x_{n}\sin\left(\frac{1}{x_n}\right) \leq |x_{n}|
\end{split}
\end{equation*}
\sphinxAtStartPar
for all \(n\in\mathbb{N}\). Since \(\displaystyle\lim_{n\to \infty} x_n=0\), we have \( \displaystyle\lim_{n\to \infty} |x_n| =0\). Therefore, but the sandwich rule, \(\displaystyle\lim_{n\rightarrow 0}x_n\sin\left(\frac{1}{x_n}\right) = 0\).

Since \((x_n)\) was arbitrary, \(\displaystyle\lim_{x \rightarrow 0} x \sin\left(\frac{1}{x}\right) = 0\).

You should contrast \hyperref[\detokenize{Solutions-upto26:xs1x}]{Fig.\@ \ref{\detokenize{Solutions-upto26:xs1x}}} with \sphinxcode{\sphinxupquote{s1x}} that for the previous question.

\begin{figure}[htbp]
\centering
\capstart

\noindent\sphinxincludegraphics[width=700\sphinxpxdimen]{{xsin(1,x)}.png}
\caption{Graph of the function \(f:\mathbb{R}\to\mathbb{R}\); \(f(x)=x\sin\left(\frac{1}{x}\right)\) (Problem 10).}\label{\detokenize{Solutions-upto26:xs1x}}\end{figure}


\bigskip\hrule\bigskip


\sphinxAtStartPar
{\hyperref[\detokenize{Problems:id11}]{\sphinxcrossref{\DUrole{std,std-ref}{11.}}}} We’ll just do \(\displaystyle\lim_{x \rightarrow \infty}f(x)\) here, as \(\displaystyle\lim_{x \rightarrow -\infty}f(x)\) is so similar.

\sphinxAtStartPar
(i) Let \(X\subset\mathbb{R}\) and \(f:X\to\mathbb{R}\). We say that \(\lim_{x \rightarrow \infty}f(x) = l\) if for all \(\varepsilon>0\) there exists \(K>0\) such that for all \(x\in X\), \(x>K\) implies \(|f(x)-l|<\varepsilon\).

\sphinxAtStartPar
(ii) The analogue of the sequential criterion is:
\begin{quote}

\sphinxAtStartPar
\(\lim_{x \rightarrow \infty}f(x) = l\) if for any sequence \((x_{n})\) in \(X\) that diverges to infinity, we have \(\lim_{x \rightarrow \infty}f(x_{n}) = l\).
\end{quote}

\sphinxAtStartPar
For the proof,  imitate the proof of \sphinxhref{https://rosiesb.github.io/Analysis-Notes/2LoF.html\#ed}{Theorem 2.1}. Suppose the \((\varepsilon- K)\) criterion stated in part (i) holds. Let \(\varepsilon>0\), and choose \(K>0\) such that for all \(x\in X\), \(x>K\) implies \(|f(x)-l|<\varepsilon\).  Let \((x_n)\) be any sequence in \(X\) that diverges to infinity. Then using the same \(K>0\), there exists \(N\in\mathbb{N}\) such that if \(n\geq N\), then \(x_{n} > K\).  But then, for all \(n\geq N\), \(x_n>K\), which in turn implies \(|f(x_{n}) - l| < \varepsilon\). Hence \(\lim_{x \rightarrow \infty}f(x_{n}) = l\), as required.

\sphinxAtStartPar
For the converse, we prove by contrapositive. Suppose the \((\varepsilon-K)\) criterion does not hold. Then there exists \(\varepsilon>0\) such that for all \(K>0\), there is \(x\in X\) for which \(x>K\) and \(|f(x)-l|\geq \varepsilon\). Apply this successively to \(K = 1, 2, 3, \ldots\). For \(K=n\), we get \(x_n\in X\) with \(x_{n} > n\), and \(|f(x_{n}) - l| \geq \varepsilon\) for each \(n\in\mathbb{N}\). But then, \((x_n)\) is a sequence in \(X\) diverging to infinity, and such that \(f(x_n)\nrightarrow l\). This shows that the sequential criterion fails.

\sphinxAtStartPar
(iii) Using the definition: Given any \(\varepsilon > 0\), choose \(K = \frac{1}{\varepsilon}\). Then \(x > K \Rightarrow \frac{1}{x} < \varepsilon\), and so \(\lim_{x\to\infty} \frac{1}{x}=0\).

\sphinxAtStartPar
Using sequences: Let \((x_n)\) be any sequence in \(\mathbb{R}\setminus\{0\}\) diverging to infinity as \(n\rightarrow\infty\). Then, by algebra of limits, \(\frac{1}{x_n}\rightarrow 0\). Since the sequence was chosen arbitrarily, it follows that \(\lim_{x\rightarrow\infty}\frac{1}{x}=0\).

\sphinxAtStartPar
The case where \(x\to-\infty\) is similar.


\bigskip\hrule\bigskip


\sphinxAtStartPar
{\hyperref[\detokenize{Problems:id12}]{\sphinxcrossref{\DUrole{std,std-ref}{12.}}}}
(i) Let \(f:X\to\mathbb{R}\) be a function, where \(X\subseteq\mathbb{R}\). We say that \(\lim_{x \rightarrow \infty} f(x) = \infty\) if:
\begin{quote}

\sphinxAtStartPar
Given any \(M > 0\), there exists \(K > 0\) such that \(x > K\) implies \(f(x) > M\).
\end{quote}

\sphinxAtStartPar
The analogue of the sequential criterion is:
\begin{quote}

\sphinxAtStartPar
For any sequence \((x_{n})\) in \(X\) which diverges to infinity, we also have that \((f(x_{n}))\) diverges to infinity.
\end{quote}

\sphinxAtStartPar
The other cases are similar.

\sphinxAtStartPar
(ii) Let \(f:\mathbb{R}\to[0,\infty)\), \(g:\mathbb{R}\to[0,\infty)\), and suppose \(\displaystyle\lim_{x\rightarrow\infty}f(x)=\infty\) and \(\displaystyle\lim_{x\rightarrow\infty}g(x)=l\), where \(l>0\).

Let \(M > 0\). Since \(\lim_{x\rightarrow\infty}f(x)=\infty\), there exists \(K_1 > 0\) such that
\begin{equation*}
\begin{split}
f(x)>\frac{2M}{l} \hspace{2em} \forall x > K_1.
\end{split}
\end{equation*}
\sphinxAtStartPar
Since \(\lim_{x\rightarrow\infty}g(x)=l\), for all \(\varepsilon>0\) we can find \(K_2>0\) such that \(x>K_2\) implies \(|g(x)-l|<\varepsilon\). In other words,
\begin{equation*}
\begin{split}
l-\varepsilon < g(x) < l+\varepsilon, \hspace{2em} \forall x>K_2.
\end{split}
\end{equation*}
\sphinxAtStartPar
This is true for all \(\varepsilon\). We apply it specifically to \(\varepsilon = \frac{l}{2}\). Then, there is \(K_2>0\) such that
\begin{equation*}
\begin{split}
\frac{l}{2} < g(x) < \frac{3l}{2}, \hspace{2em} \forall x>K_2.
\end{split}
\end{equation*}
\sphinxAtStartPar
Let \(K=\max\{K_1,K_2\}\). If \(x>K\), then \(x>K_1\) and \(x>K_2\), so \(f(x)>\frac{2M}{l}\) and \(\frac{l}{2} < g(x) < \frac{3l}{2}\).

Hence for \(x>K\),
\begin{equation*}
\begin{split}
f(x)g(x) > \frac{2M}{l}\frac{l}{2} = M.
\end{split}
\end{equation*}
\sphinxAtStartPar
That is, \(\lim_{x\rightarrow\infty}f(x)g(x) = \infty\).

\sphinxAtStartPar
(iii) Let \(p:\mathbb{R}\to\mathbb{R}\) be a polynomial of even degree \(m\), with positive leading coefficient. Write \(m = 2n\) and let
\begin{align*}
p(x) &= a_{2n}x^{2n} + a_{2n-1}x^{2n-1} + \cdots + a_{1}x + a_{0}\\
&= x^{2n}\left(a_{2n} + \frac{a_{2n-1}}{x} + \cdots + \frac{a_{1}}{x^{2n-1}} + \frac{a_{0}}{x^{2n}}\right). 
\end{align*}
\sphinxAtStartPar
We use part (ii). Take \(f(x) = x^{2n}\) and \(\displaystyle g(x) = a_{2n} + \frac{a_{2n-1}}{x} + \cdots + \frac{a_{1}}{x^{2n-1}} + \frac{a_{0}}{x^{2n}}\).

Observe that \(\displaystyle\lim_{x \rightarrow \infty}f(x) = \infty\) and \(\displaystyle\lim_{x \rightarrow \infty}g(x) = a_{2n}\). Hence by part (ii),
\begin{equation*}
\begin{split}
\lim_{x \rightarrow \infty}p(x) = \lim_{x \rightarrow \infty}f(x)g(x) = \infty.
\end{split}
\end{equation*}
\sphinxAtStartPar
If \(n\) is odd, \(\lim_{x \rightarrow \infty}p(x) = \infty\), but \(\lim_{x \rightarrow -\infty}p(x) = -\infty\).


\subsection{Continuity}
\label{\detokenize{Solutions-upto26:continuity}}\label{\detokenize{Solutions-upto26:ch3sol}}
\sphinxAtStartPar
{\hyperref[\detokenize{Problems:id13}]{\sphinxcrossref{\DUrole{std,std-ref}{13.}}}} For each of (i) to (v), the function is continuous at each point of its domain. For (i), (ii) and (iii), we have rational functions, which are continuous on all points of \(A\) (which is the subset of \(\mathbb{R}\) where the denominator is non\sphinxhyphen{}zero). For (iv) and (v), we have compositions of rational functions with the exponential and cosine functions, which are continuous on the whole of \(\mathbb{R}\). Again, the functions are continuous on all points of \(A\) (which is the subset of \(\mathbb{R}\) where the denominator of the rational function is non\sphinxhyphen{}zero).


\bigskip\hrule\bigskip


\sphinxAtStartPar
{\hyperref[\detokenize{Problems:id14}]{\sphinxcrossref{\DUrole{std,std-ref}{14.}}}} If \((x_{n})\) is any sequence that converges to \(a\), we know that \(f(x_n)\) converges to \(f(a)\) as \(f\) is continuous at \(a\), and we need to show that \((|f|(x_n))\) converges to \(|f|(a)\).

By Corollary 1.1, if \((x_{n})\) is any sequence that converges to \(a\),
\begin{align*}
0 \leq ||f|(a)| -|f|(x_{n})| &= ||f(a) - |f(x_{n})|| \\
&\leq |f(a) - f(x_{n})| \rightarrow 0~\mbox{as}~n \rightarrow \infty, 
\end{align*}
\sphinxAtStartPar
as \(f\) is continuous. So by the sandwich rule, \(\lim_{n\rightarrow\infty} |f|(x_{n}) = |f|(a)\), and so \(f\) is continuous at \(a\).


\bigskip\hrule\bigskip


\sphinxAtStartPar
{\hyperref[\detokenize{Problems:id15}]{\sphinxcrossref{\DUrole{std,std-ref}{15.}}}} Let \((x_{n})\) be a sequence in \(A\) that converges to \(a\). Since \(f\) is continuous at \(a\) we have \(\lim_{n\rightarrow\infty} f(x_{n}) = f(a)\). And then, since \(g\) is continuous at \(f(a)\), we have \(\lim_{n\rightarrow\infty} g(f(x_{n})) = g(f(a))\), and the result follows.


\bigskip\hrule\bigskip


\sphinxAtStartPar
{\hyperref[\detokenize{Problems:id16}]{\sphinxcrossref{\DUrole{std,std-ref}{16.}}}} \(f \circ g:\mathbb{R}\to \mathbb{R}\) given by \((f \circ g)(x) = \frac{1}{1  + x^{2}}\). It is continuous on \(\mathbb{R}\) by Theorem 3.2(iv).

\(g \circ f:\mathbb{R} \setminus \{0\} \to \mathbb{R} \setminus \{0\}\) given by \((g \circ f)(x) = {1  + \frac{1}{x^2}}\). It is continuous on \(\mathbb{R} \setminus \{0\}\) by Theorem 3.2(iv).


\bigskip\hrule\bigskip


\sphinxAtStartPar
{\hyperref[\detokenize{Problems:id17}]{\sphinxcrossref{\DUrole{std,std-ref}{17.}}}}
The proofs for each of these answers are immediate by {\hyperref[\detokenize{Problems:id8}]{\sphinxcrossref{\DUrole{std,std-ref}{Problem 8}}}}.

\sphinxAtStartPar
(i) \(f:\mathbb{R}\to\mathbb{R}\); \(f(x) = \begin{cases} 1 -x & \text{if }x < 1\\ x^{2}& \text{if }x \geq 1. \end{cases}\)

\sphinxAtStartPar
This function is continuous on \(\mathbb{R} \setminus \{1\}\), with a jump discontinuity at \(x=1\) of size \(1\).

\sphinxAtStartPar
(ii) \(g:\mathbb{R}\to\mathbb{R}\); \(g(x) = [x] = \left\{\begin{array}{cl} \lfloor x\rfloor & \text{ if } x\geq 0 \\ \lceil x \rceil & \text{ if } x<0 \end{array}\right.\)

\sphinxAtStartPar
This function is continuous on \(\mathbb{R}\setminus\{\pm k: k\in\mathbb{N}\}\). Jump discontinuity at \(n\) with size \(1\) for all \(n\in\mathbb{Z}_+\).

\sphinxAtStartPar
(iii) \(h:\mathbb{R}\to\mathbb{R}\); \(h(x) =3 - 5\mathbb{1}_{(0, 1]}(x) + 7\mathbb{1}_{(1, 2]}(x)\)

\sphinxAtStartPar
This is continuous at \(\mathbb{R} \setminus \{0,1,2\}\), with jump discontinuities at \(0\), \(1\), and \(2\). The jump sizes are, respectively, \(0\), \(-5\), and \(-7\).


\bigskip\hrule\bigskip

\phantomsection\label{\detokenize{Solutions-upto26:id5}}
\sphinxAtStartPar
{\hyperref[\detokenize{Problems:id18}]{\sphinxcrossref{\DUrole{std,std-ref}{18.}}}} To appear (Homework 3 question)


\bigskip\hrule\bigskip

\phantomsection\label{\detokenize{Solutions-upto26:id6}}
\sphinxAtStartPar
{\hyperref[\detokenize{Problems:id19}]{\sphinxcrossref{\DUrole{std,std-ref}{19.}}}} To appear (Homework 2 question).


\bigskip\hrule\bigskip


\sphinxAtStartPar
{\hyperref[\detokenize{Problems:id20}]{\sphinxcrossref{\DUrole{std,std-ref}{20.}}}}
(i)
\begin{equation*}
\begin{split}
\max\{a, b\} = \left\{\begin{array}{c c} a & ~\mbox{if}~a \geq b\\
b & ~\mbox{if}~a < b.\\ \end{array} \right.
\end{split}
\end{equation*}
\sphinxAtStartPar
On the other hand,
\begin{equation*}
\begin{split}
\frac{1}{2}( a+b) + \frac{1}{2}|a-b| = \left\{\begin{array}{c c c} \frac{1}{2}(a+b) + \frac{1}{2}(a- b) & = a & ~\mbox{if}~a \geq b\\[.5em]
\frac{1}{2}(a+b) + \frac{1}{2}(b- a) &=  b & ~\mbox{if}~a < b,\\ \end{array} \right.
\end{split}
\end{equation*}
\sphinxAtStartPar
and the result follows.

\sphinxAtStartPar
Then for all \(x \in A\cap B,\)
\begin{equation*}
\begin{split}
\max\{f, g\}(x) = \frac{1}{2}(f(x) + g(x)) + \frac{1}{2}|f(x) - g(x)|,
\end{split}
\end{equation*}
\sphinxAtStartPar
and continuity follows by Theorem  3.2(i) and (iii) and {\hyperref[\detokenize{Problems:id14}]{\sphinxcrossref{\DUrole{std,std-ref}{Problem 14}}}}.

\sphinxAtStartPar
(ii) You can check that for all \(a, b \in \mathbb{R}\),
\begin{equation*}
\begin{split}
\min\{a, b\} = \frac{1}{2}(a + b) - \frac{1}{2}|a - b|,
\end{split}
\end{equation*}
\sphinxAtStartPar
and then argue as in (i).

\sphinxAtStartPar
Alternatively, one could derive (ii) from (i) by using \(\min\{f,g\} = - \max\{-f, -g\}\).


\bigskip\hrule\bigskip


\sphinxAtStartPar
{\hyperref[\detokenize{Problems:id21}]{\sphinxcrossref{\DUrole{std,std-ref}{21.}}}} Let \(f: \mathbb{R} \rightarrow \mathbb{R}\) be such that
\begin{equation}\label{equation:Solutions-upto26:linear}
\begin{split}f(x+y) = f(x) + f(y) \; \forall x,y \in \mathbb{R}.\end{split}
\end{equation}
\sphinxAtStartPar
Then:

\sphinxAtStartPar
(i) \(f(0) = f(0 + 0) = f(0) + f(0) = 2f(0)\), hence \(f(0) = 0\).

\sphinxAtStartPar
(ii) For all \(x\in\mathbb{R}\), we have by (i), \(0 = f(0) = f(x + -x) = f(x) + f(-x)\). So \(f(-x)=-f(x)\).

\sphinxAtStartPar
(iii) Suppose \(f\) is continuous at zero. Then, if \(x \neq 0\) then any sequence \((x_{n})\) which converges to \(x\) can be written as \(x_{n} = x + y_{n}\) where \((y_{n})\) converges to zero. Since \(f\) is assumed to be continuous at \(0\), we have that \( \lim_{n\rightarrow\infty} f(y_{n})\) exists and equals \(f(0)\). By (i), \(f(0)=0\), hence
\begin{equation*}
\begin{split}
\lim_{n\rightarrow\infty} f(x_{n}) = \lim_{n\rightarrow\infty} f(a + y_{n}) = f(a) + \lim_{n\rightarrow\infty} f(y_{n}) = f(a) + f(0) = f(a)+0 = f(a).
\end{split}
\end{equation*}
\sphinxAtStartPar
(iv) Let \(k=f(1)\). We use induction on \(n\in\mathbb{N}\). If \(n=1\), then \(f(n)=f(1)=k=k\cdot 1\), so the statement holds. Assume it holds for some \(n\in\mathbb{N}\). Then by {\hyperref[\detokenize{Solutions-upto26:equation-linear}]{\sphinxcrossref{(2.2)}}},
\begin{equation*}
\begin{split}
f(n+1) = f(n) + f(1) = nk + k = (n+1)k.
\end{split}
\end{equation*}
\sphinxAtStartPar
Hence by induction, \(f(n)=nk\), for all \(n\in\mathbb{N}\). Combine this with part (ii) to extend to all \(n\in \mathbb{Z}\).

\sphinxAtStartPar
(v) Still writing \(k=f(1)\), consider \(\frac{p}{q}\in\mathbb{Q}\), where \(p\in \mathbb{Z}\) and \(q\in \mathbb{N}\). By (iv),
\begin{equation*}
\begin{split}
pk = f(p) = f\left(q\cdot \frac{p}{q}\right) = qf\left(\frac{p}{q}\right),
\end{split}
\end{equation*}
\sphinxAtStartPar
and the result follows.

\sphinxAtStartPar
(vi) Finally, let \(k=f(1)\), and assume \(f\) is continuous at \(0\), and let \(x\in\mathbb{R}\). If \(x\in\mathbb{Q}\), then we have already shown \(f(x)=kx\) in part (v). So assume \(x\) is irrational. By denseness of the rationals, there is a sequence \((r_n)\) of rational numbers converging to \(x\). By (v), \(f(r_n)=kr_n\) for all \(n\in\mathbb{N}\). By (iii), \(f\) is continuous at \(x\), so using algebra of limits, we have
\begin{equation*}
\begin{split}
f(x) = \lim_{n\rightarrow\infty} f(r_n) =   \lim_{n\rightarrow\infty} kr_n=k \lim_{n\rightarrow\infty} r_n = kx.
\end{split}
\end{equation*}

\bigskip\hrule\bigskip

\phantomsection\label{\detokenize{Solutions-upto26:id7}}
\sphinxAtStartPar
{\hyperref[\detokenize{Problems:id22}]{\sphinxcrossref{\DUrole{std,std-ref}{22.}}}} To appear (Homework 2 question).


\bigskip\hrule\bigskip


\sphinxAtStartPar
{\hyperref[\detokenize{Problems:id23}]{\sphinxcrossref{\DUrole{std,std-ref}{23.}}}} The function \(\mathbb{1}_{(a, b)}\) is left continuous at \(a\) (but not right continuous), and right continuous at \(b\) (but not left continuous).

\sphinxAtStartPar
To prove the left continuity at \(a\), let \((x_{n})\) be any sequence in \(\mathbb{R}\) which converges to \(a\) with \(x_{n} < a\) for all \(n\in\mathbb{N}\). Then \(\lim_{n\rightarrow\infty} \mathbb{1}_{(a, b)}(x_{n}) = 0 = \mathbb{1}_{(a, b)}(a).\) On the other hand to see that it is not right continuous at \(a\), let \((y_{n})\) be any sequence in \(\mathbb{R}\) which converges to \(a\) with \(a < y_{n} < b\) for all \(n\in\mathbb{N}\). Then
\( \lim_{n\rightarrow\infty} \mathbb{1}_{(a, b)}(y_{n}) = 1 \neq \mathbb{1}_{(a, b)}(a). \) The other assertion is proved similarly.

{[}Contrast this with \(\mathbb{1}_{[a, b]}\), which was discussed in the notes.{]}


\bigskip\hrule\bigskip


\sphinxAtStartPar
{\hyperref[\detokenize{Problems:id24}]{\sphinxcrossref{\DUrole{std,std-ref}{24.}}}} Since \(f\) is continuous on \([a, b]\), so is \(g\). We have \(g(a) = f(a) - \gamma < 0\) and \(g(b) = f(b) - \gamma > 0\). Hence by \sphinxhref{https://rosiesb.github.io/Analysis-Notes/3Cty.html\#ivt-sc}{Proposition 3.1}, there exists \(c \in (a, b)\) with \(g(c) = 0\), i.e. \(f(c) = \gamma\), as was required.


\bigskip\hrule\bigskip

\phantomsection\label{\detokenize{Solutions-upto26:id8}}
\sphinxAtStartPar
{\hyperref[\detokenize{Problems:id25}]{\sphinxcrossref{\DUrole{std,std-ref}{25.}}}} To appear (Homework 3 question).


\bigskip\hrule\bigskip


\sphinxAtStartPar
{\hyperref[\detokenize{Problems:id26}]{\sphinxcrossref{\DUrole{std,std-ref}{26.}}}} Define \(\gamma = \inf_{x \in [a, b]}f(x)\) and assume that it is not attained, so \(\gamma < f(x)\) for all \(x \in [a,b]\). Then consider the function \(h:[a,b]\to \mathbb{R}\) given by \(h(x) = \displaystyle\frac{1}{f(x) - \gamma}\). This is continuous, and hence bounded on \([a, b]\). So there exists \(K \geq 0\) such that \(|h(x)| \leq K\) for all \(x \in [a, b]\). By Problem 16(ii), given any \(\varepsilon > 0\), there exists \(x \in [a, b]\) such that \(f(x) < \gamma + \varepsilon\). Now take \(\varepsilon = \frac{1}{K}\) to deduce that \(h(x) > K\), which yields the required contradiction.

\sphinxstepscope


\section{Homework feedback}
\label{\detokenize{HW-feedback:homework-feedback}}\label{\detokenize{HW-feedback:hw-feedback}}\label{\detokenize{HW-feedback::doc}}
\sphinxAtStartPar
Here is some general feedback comments about each piece of homework for MAS2004/9. Individual feedback can be viewed on Crowdmark — let me know if you have any difficulty finding it.

\sphinxAtStartPar
Responding to individual feedback is the fastest way to improve as a mathematician. Hopefully your feedback will feel useful and make sense to you, but if it ever doesn’t, please get in touch and I am more than happy to look at it with you.


\subsection{Feedback for Homework 1}
\label{\detokenize{HW-feedback:feedback-for-homework-1}}
\sphinxAtStartPar
\sphinxstylestrong{Problems:} {\hyperref[\detokenize{Problems:p2}]{\sphinxcrossref{\DUrole{std,std-ref}{P2}}}}, {\hyperref[\detokenize{Problems:p3}]{\sphinxcrossref{\DUrole{std,std-ref}{P3}}}}, {\hyperref[\detokenize{Problems:id2}]{\sphinxcrossref{\DUrole{std,std-ref}{2}}}}, {\hyperref[\detokenize{Problems:id3}]{\sphinxcrossref{\DUrole{std,std-ref}{3}}}};  \sphinxstylestrong{Solutions:} {\hyperref[\detokenize{Solutions-upto26:p2sol}]{\sphinxcrossref{\DUrole{std,std-ref}{P2}}}}, {\hyperref[\detokenize{Solutions-upto26:p3sol}]{\sphinxcrossref{\DUrole{std,std-ref}{P3}}}}, {\hyperref[\detokenize{Solutions-upto26:id1}]{\sphinxcrossref{\DUrole{std,std-ref}{2}}}}, {\hyperref[\detokenize{Solutions-upto26:id2}]{\sphinxcrossref{\DUrole{std,std-ref}{3}}}}.

\sphinxAtStartPar
Well done for completing the first analysis homework! Here are some general comments.


\subsubsection{General comments on mathematical writing}
\label{\detokenize{HW-feedback:general-comments-on-mathematical-writing}}\begin{itemize}
\item {} 
\sphinxAtStartPar
The absolute best thing you can do is look at the individual comments on your work on Crowdmark, and \sphinxhref{https://calendar.app.google/vqdgru29fnbycVoY6}{get help} interpreting them as needed. Analysis proofs are notoriously picky, and all mathematicians go through this at some stage or other, so please don’t feel embarassed about needing help with your proof writing.

\item {} 
\sphinxAtStartPar
When using a definition to solve a problem, it’s fine to write out the general definition, but make sure you also explicitly link it to the question at hand. So, for example, when solving {\hyperref[\detokenize{Problems:id3}]{\sphinxcrossref{\DUrole{std,std-ref}{Problem 3}}}}, you could (though don’t have to) begin by copying out \sphinxhref{https://rosiesb.github.io/Analysis-Notes/2LoF.html\#functionlimit}{Definition 2.2} from the notes. If you do that, make sure you then say something like “Therefore, given \(\varepsilon>0\), we seek \(\delta>0\) such that …” and link it to the specific functions and limits you are working with.

\item {} 
\sphinxAtStartPar
Remember to only use notation/variables after you have introduced them. This sometimes relates to the above point — Definition 2.2 has a general limit \(L\) in it, but there’s no need to carry it around in your solution if you know the actual numerical limit you’re aiming for (e.g. \(5/2\)).

\end{itemize}


\subsubsection{Problems P2 and P3}
\label{\detokenize{HW-feedback:problems-p2-and-p3}}
\sphinxAtStartPar
These were generally answered well, and most people were able to find the intended theorems from their MAS107/MAS117 notes in order to re\sphinxhyphen{}prove the Bolzano\sphinxhyphen{}Weierstrass theorem. We’ll be using this theorem more than once this semester, so this revision will come in handy.

\sphinxAtStartPar
In {\hyperref[\detokenize{Problems:p2}]{\sphinxcrossref{\DUrole{std,std-ref}{P2}}}}, there were some issues with mathematical writing, so make sure you go and look at the individual comments on your work and see what comments you can use to improve this.


\subsubsection{Problem 2}
\label{\detokenize{HW-feedback:problem-2}}
\sphinxAtStartPar
For this question, it’s fine to just write down what the limit points are, but of course we like proofs in this module. If you did include a proof, well done! And take a look for comments about its structure and presentation. The {\hyperref[\detokenize{Solutions-upto26:id1}]{\sphinxcrossref{\DUrole{std,std-ref}{solutions}}}} do now include proofs, if you would like to see how they go.


\subsubsection{Problem 3}
\label{\detokenize{HW-feedback:problem-3}}
\sphinxAtStartPar
The three parts of this question were not of equal difficulty, and this was intentional. I would say part (iii) is definitely the hardest, so well done if you got that one out. If you didn’t, take a look at the individual feedback and see if you can make some progress, and/or, bring it to an office hour. You may find it easier to tackle now you have had more time to practise.

\sphinxAtStartPar
One of the more common issues people had in {\hyperref[\detokenize{Problems:id3}]{\sphinxcrossref{\DUrole{std,std-ref}{Problem 3(iii)}}}} was defining a \(\delta\) that depended on \(x\) as well as \(\varepsilon\). This is an understandable error: once you get to the stage
\begin{equation}\label{equation:HW-feedback:hw1}
\begin{split}\left|f(x)-\frac{5}{2}\right|=\left|1-\frac{1}{2x}\right||x-2|\end{split}
\end{equation}
\sphinxAtStartPar
it can be hard to know how to deal with the \(\left|1-\frac{1}{2x}\right|\) factor. However, letting \(\delta=\frac{\varepsilon}{|1-\frac{1}{2x}|}\) does \sphinxstylestrong{not} work here — we need a \(\delta\) that is independent of \(x\) (though it is allowed to depend on \(\varepsilon\)). One way to see why is to look at the order that different variables appear in the definition of the limit:
\begin{equation*}
\begin{split}
\forall\varepsilon>0\;\exists\delta>0\;\text{s.t. }\forall x\in X \; |x-a|<\delta\Rightarrow|f(x)-L|<\varepsilon.
\end{split}
\end{equation*}
\sphinxAtStartPar
Here, \(\varepsilon\) appears first, and so cannot depend on anything, and \(\delta\) appears second, so can only depend on \(\varepsilon\). In particular, \(\varepsilon\) and \(\delta\) appear \sphinxstyleemphasis{before} \(x\), so neither are allowed to depend on \(x\).

\sphinxAtStartPar
To progress past the point of {\hyperref[\detokenize{HW-feedback:equation-hw1}]{\sphinxcrossref{(3.1)}}} with a \(\delta\) \sphinxstylestrong{independent} of \(x\), the thing to do take a \(\delta\) that is the minimum of two values, with on value controlling each factor. For example, if \(|x-2|<1\), then \(0<1-\frac{1}{2x}<1\), so put \(\delta=\min\{1,\text{?}\}\). Now find “\(\text{?}\)” (or have a look at the {\hyperref[\detokenize{Solutions-upto26:iiisol}]{\sphinxcrossref{\DUrole{std,std-ref}{solutions}}}}).







\renewcommand{\indexname}{Index}
\printindex
\end{document}